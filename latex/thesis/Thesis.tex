%\documentclass[german, twoside, parskip]{VRThesis} % dies ist eine deutsche Abschlussarbeit
\documentclass[twoside, parskip]{VRThesis} % this is an english bachelor/master thesis

% replace this by the title of the thesis
\title{Pre-operative Planning in Virtual Reality with Head Mounted Displays for Oral and Maxillofacial Surgery}
% replace this by the authors name
\author{Filip Kajzer}
% replace this by the date of issue (Abgabedatum), for PhD thesis date of the exam
\date{\today}                 

% options for master thesis and Diplomarbeit
% Matrikelnummer, student ID
\studentid{380 428}
\fieldofstudy{Informatik}
%\fieldofstudy{Software Systems Engineering}
\firstsupervisor{Prof. Dr. Torsten W. Kuhlen \\ Visual Computing Institute, RWTH Aachen University}
%\firstsupervisor{Prof. C. Bischof, Ph.D. \\ Institute for Scientific Computing}
\secondsupervisor{Priv.-Doz. Dr. med. Dr. med. dent. Ali Modabber, MBA \\ Department for Oral and Maxillofacial Surgery, University Hospital RWTH Aachen}
\tutor{Andrea Bönsch, Behrus Puladi}

\addtolength{\oddsidemargin}{+.2in}
\addtolength{\evensidemargin}{-.2in}

% use compactenum to save space
\usepackage{enumitem}
\newlist{compactenum}{enumerate}{4}
\setlist[compactenum,1]{nolistsep}
% avoid orphans and widows
\clubpenalty = 10000
\widowpenalty = 10000
%\displaywidowpenalty = 10000

%Generate the environment for the abstract:
\newcommand\summaryname{Abstract}
\newenvironment{Abstract}%
    {\small\begin{center}%
    \bfseries{\summaryname} \end{center}}
\maketitle

\begin{document}

\maketitle
%\makecoverMaster % this is the thesis cover (a shorter version of the title page)
\maketitleMaster % this is the thesis title page

% generates the statement (Erkl�rung) page for master thesis and Diplomarbeit
\makestatement

% Insert bastract
\begin{Abstract}
    \textbf{Introduction}: Virtual reality (VR) technology is first depicted in the scientific literature as early as 1963.
The upcoming of head-mounted devices (HMDs) is described since the early 1960s.
With the emergence of consumer HMDs such as the Oculus Rift in 2013, a widespread use of VR HMDs has followed.
The use of VR in surgical training is described since 1995 and has proven to bring advantages over classic surgical training.
However, the use of VR-based surgical scenarios is still limited duo to two reasons.
First, most current scenarios are application-specific instead of generalizable.
This requires costly and customized development for each medical procedure simulated in VR.
Second, most applications focus on expensive room-mounted displays and custom input devices instead of affordable HMDs.
Through the use of HMDs, an immersive and cost effective solution for VR surgical training could be provided.
By using consumer devices, the barrier of entry for such VR surgial training software is rather low.
Furthermore, there exist no surgical workflow for VR-based surgical planning and training and augmented reality (AR) glasses based intra-operational navigation.
By choosing a common data format, exchanging data from the planning process in VR can be reused for intra-operational guidance with AR glasses.
In the course of this master thesis, an open-source software application was developed for an VR scenario with HMDs embedded in an VR/AR surgical workflow in the field of oral and maxillofacial surgery (OMFS). 
(\textbf{TODO ANDREA}: Explain what OMFS is and what the goal of the medical procedures is. Remember: the abstract is the first thing readers read.)


\textbf{Material and Methods}: Unity 3D was used together with the Open VR Software Development Kit to develop an open-source software for a VR-scenario with the HTC Vive.
The software was coded in C\#. 
Additionally, the recently released Valve Index controller, which allow for tracking of the user hands in VR, were used for a more natural and immersive experience.
A real operation room was captured with a 360 degree camera and used for more immersion in the developed VR software.
Corresponding 3D models of true surgical instruments and a wide variety of surgical material like osteosynthesis plates were implemented.
The developed VR scenario was evaluated by 5 OMFS trainees in a system usability scale study.

\textbf{Results}: (\textbf{TODO ANDREA}: ...Here i would expect statements of how well your 5 participants could handle the system, whether they liked the options, whether they were able to plan the procedures for their cases etc.)
Furthermore, training with VR was percieved XXX. (TODO results of study)

\textbf{Conclusion}: The use of an open-source VR software with VR HMDs from the consumer market is a cost-effective application in the field of OMFS.

\end{Abstract}

% generates the table of contents (Inhaltsverzeichnis)
\tableofcontents

\chapter{\label{chap::Introduction}Introduction}
TODO: Jeden Absatz zitieren
As previously described, the user interface consists of a GUI, VUI and natural hand interactions such as grabbing.
In the following, how users interact with the different kind of interfaces will be described. 

\subsection{\label{sec::GraphicalUserInterface}Graphical User Interface}
The first graphical interface which the user will see when pressing the button is depicted in Figure \ref{fig::UIProjectCase}
\begin{figure}[ht]
    \centering
    \includegraphics[width=200px]{images/implementation/user_interface/project_cases.png}
    \caption{\label{fig::UIProjectCase}User Interface for the Selection of Project Cases}
\end{figure}
The buttons for loading a case are created dynamically and depend on the number of project case JSON files found in the specified folder.
As will be explained later, the evaluation consists of five distinct scenarios in which distinct procedures have to be performed via the available tools.
By pressing the button by selecting it with the representation of the users hand in the virtual operating room, the project case will be loaded and the patient and information written down in the project case will be display from within the virtual operting room.
On the left side, different submenus can be selected.
Visualization tools can be selected via clicking on 'patient', as depicted in Figure \ref{fig::UIPatient}.
Here, tools which are the same for any project case can be used for visualization purposes.
Users have the option to scale the patient, look at the unprocessed patient for comparison and the ability to explode the 3D-Model to help with visualization.
Project cases can also be saved from this menu (Figure \ref{fig::UIPatient}).

\begin{figure}[h]
    \centering
    \begin{minipage}{.5\textwidth}
      \centering
      \includegraphics[width=0.95\linewidth]{images/implementation/user_interface/patient.png}
      \caption{\label{fig::UIPatient}User Interface for Patient Visualization Tools}
    \end{minipage}%
    \begin{minipage}{.5\textwidth}
      \centering
      \includegraphics[width=0.95\linewidth]{images/implementation/user_interface/segments.png}
      \caption{\label{fig::UIPatientSegments}User Interface for Patient specific Segments}
    \end{minipage}
  \end{figure}
\subsection{\label{sec::VoiceUserInterface}Voice User Interface}
\begin{center}
    \label{table::VoiceCommands}
    \begin{tabular}{ | l | p{4cm} | p{8cm} |}
    \hline
    Command & Action & Comment \\ \hline
    "Start" & Show first step of procedure & This command also disables the "Show all" command. \\ \hline
    "Show all" & Show all steps enumerated chronologicly & Is disabled via "Start" command. \newline
    Disables the use of 'step next' or 'step back'. \\ \hline
    "Step next" & Show next step of the procedure & \\ \hline
    "Step back" & Show previous step of the procedure & \\ \hline
    "Undo" & Reverts current step & Last undid step gets cached and can be redone via "Redo" command. \\ \hline
    "Redo" & Redo last reverted step & Only the last reverted step can be redone. \\ \hline
    "Train start" & Start train mode & Start at step 1 and navigate steps by following instructions and performing procedures in the correct manner.
    It also disables every other command except "Train stop" \\ \hline
    "Train stop" & Stop train mode & \\ \hline
    \end{tabular}
\end{center}

Table \ref{table::VoiceCommands} shows all Voice Commands.

\chapter{\label{chap::RelatedWork}Related Work}
As previously described, the user interface consists of a GUI, VUI and natural hand interactions such as grabbing.
In the following, how users interact with the different kind of interfaces will be described. 

\subsection{\label{sec::GraphicalUserInterface}Graphical User Interface}
The first graphical interface which the user will see when pressing the button is depicted in Figure \ref{fig::UIProjectCase}
\begin{figure}[ht]
    \centering
    \includegraphics[width=200px]{images/implementation/user_interface/project_cases.png}
    \caption{\label{fig::UIProjectCase}User Interface for the Selection of Project Cases}
\end{figure}
The buttons for loading a case are created dynamically and depend on the number of project case JSON files found in the specified folder.
As will be explained later, the evaluation consists of five distinct scenarios in which distinct procedures have to be performed via the available tools.
By pressing the button by selecting it with the representation of the users hand in the virtual operating room, the project case will be loaded and the patient and information written down in the project case will be display from within the virtual operting room.
On the left side, different submenus can be selected.
Visualization tools can be selected via clicking on 'patient', as depicted in Figure \ref{fig::UIPatient}.
Here, tools which are the same for any project case can be used for visualization purposes.
Users have the option to scale the patient, look at the unprocessed patient for comparison and the ability to explode the 3D-Model to help with visualization.
Project cases can also be saved from this menu (Figure \ref{fig::UIPatient}).

\begin{figure}[h]
    \centering
    \begin{minipage}{.5\textwidth}
      \centering
      \includegraphics[width=0.95\linewidth]{images/implementation/user_interface/patient.png}
      \caption{\label{fig::UIPatient}User Interface for Patient Visualization Tools}
    \end{minipage}%
    \begin{minipage}{.5\textwidth}
      \centering
      \includegraphics[width=0.95\linewidth]{images/implementation/user_interface/segments.png}
      \caption{\label{fig::UIPatientSegments}User Interface for Patient specific Segments}
    \end{minipage}
  \end{figure}
\subsection{\label{sec::VoiceUserInterface}Voice User Interface}
\begin{center}
    \label{table::VoiceCommands}
    \begin{tabular}{ | l | p{4cm} | p{8cm} |}
    \hline
    Command & Action & Comment \\ \hline
    "Start" & Show first step of procedure & This command also disables the "Show all" command. \\ \hline
    "Show all" & Show all steps enumerated chronologicly & Is disabled via "Start" command. \newline
    Disables the use of 'step next' or 'step back'. \\ \hline
    "Step next" & Show next step of the procedure & \\ \hline
    "Step back" & Show previous step of the procedure & \\ \hline
    "Undo" & Reverts current step & Last undid step gets cached and can be redone via "Redo" command. \\ \hline
    "Redo" & Redo last reverted step & Only the last reverted step can be redone. \\ \hline
    "Train start" & Start train mode & Start at step 1 and navigate steps by following instructions and performing procedures in the correct manner.
    It also disables every other command except "Train stop" \\ \hline
    "Train stop" & Stop train mode & \\ \hline
    \end{tabular}
\end{center}

Table \ref{table::VoiceCommands} shows all Voice Commands.
\section{\label{sec::MedicalImaging}Medical Imaging}
As previously described, the user interface consists of a GUI, VUI and natural hand interactions such as grabbing.
In the following, how users interact with the different kind of interfaces will be described. 

\subsection{\label{sec::GraphicalUserInterface}Graphical User Interface}
The first graphical interface which the user will see when pressing the button is depicted in Figure \ref{fig::UIProjectCase}
\begin{figure}[ht]
    \centering
    \includegraphics[width=200px]{images/implementation/user_interface/project_cases.png}
    \caption{\label{fig::UIProjectCase}User Interface for the Selection of Project Cases}
\end{figure}
The buttons for loading a case are created dynamically and depend on the number of project case JSON files found in the specified folder.
As will be explained later, the evaluation consists of five distinct scenarios in which distinct procedures have to be performed via the available tools.
By pressing the button by selecting it with the representation of the users hand in the virtual operating room, the project case will be loaded and the patient and information written down in the project case will be display from within the virtual operting room.
On the left side, different submenus can be selected.
Visualization tools can be selected via clicking on 'patient', as depicted in Figure \ref{fig::UIPatient}.
Here, tools which are the same for any project case can be used for visualization purposes.
Users have the option to scale the patient, look at the unprocessed patient for comparison and the ability to explode the 3D-Model to help with visualization.
Project cases can also be saved from this menu (Figure \ref{fig::UIPatient}).

\begin{figure}[h]
    \centering
    \begin{minipage}{.5\textwidth}
      \centering
      \includegraphics[width=0.95\linewidth]{images/implementation/user_interface/patient.png}
      \caption{\label{fig::UIPatient}User Interface for Patient Visualization Tools}
    \end{minipage}%
    \begin{minipage}{.5\textwidth}
      \centering
      \includegraphics[width=0.95\linewidth]{images/implementation/user_interface/segments.png}
      \caption{\label{fig::UIPatientSegments}User Interface for Patient specific Segments}
    \end{minipage}
  \end{figure}
\subsection{\label{sec::VoiceUserInterface}Voice User Interface}
\begin{center}
    \label{table::VoiceCommands}
    \begin{tabular}{ | l | p{4cm} | p{8cm} |}
    \hline
    Command & Action & Comment \\ \hline
    "Start" & Show first step of procedure & This command also disables the "Show all" command. \\ \hline
    "Show all" & Show all steps enumerated chronologicly & Is disabled via "Start" command. \newline
    Disables the use of 'step next' or 'step back'. \\ \hline
    "Step next" & Show next step of the procedure & \\ \hline
    "Step back" & Show previous step of the procedure & \\ \hline
    "Undo" & Reverts current step & Last undid step gets cached and can be redone via "Redo" command. \\ \hline
    "Redo" & Redo last reverted step & Only the last reverted step can be redone. \\ \hline
    "Train start" & Start train mode & Start at step 1 and navigate steps by following instructions and performing procedures in the correct manner.
    It also disables every other command except "Train stop" \\ \hline
    "Train stop" & Stop train mode & \\ \hline
    \end{tabular}
\end{center}

Table \ref{table::VoiceCommands} shows all Voice Commands.
\section{\label{sec::VirtualRealityInMedicine}Virtual Reality in Medicine}
As previously described, the user interface consists of a GUI, VUI and natural hand interactions such as grabbing.
In the following, how users interact with the different kind of interfaces will be described. 

\subsection{\label{sec::GraphicalUserInterface}Graphical User Interface}
The first graphical interface which the user will see when pressing the button is depicted in Figure \ref{fig::UIProjectCase}
\begin{figure}[ht]
    \centering
    \includegraphics[width=200px]{images/implementation/user_interface/project_cases.png}
    \caption{\label{fig::UIProjectCase}User Interface for the Selection of Project Cases}
\end{figure}
The buttons for loading a case are created dynamically and depend on the number of project case JSON files found in the specified folder.
As will be explained later, the evaluation consists of five distinct scenarios in which distinct procedures have to be performed via the available tools.
By pressing the button by selecting it with the representation of the users hand in the virtual operating room, the project case will be loaded and the patient and information written down in the project case will be display from within the virtual operting room.
On the left side, different submenus can be selected.
Visualization tools can be selected via clicking on 'patient', as depicted in Figure \ref{fig::UIPatient}.
Here, tools which are the same for any project case can be used for visualization purposes.
Users have the option to scale the patient, look at the unprocessed patient for comparison and the ability to explode the 3D-Model to help with visualization.
Project cases can also be saved from this menu (Figure \ref{fig::UIPatient}).

\begin{figure}[h]
    \centering
    \begin{minipage}{.5\textwidth}
      \centering
      \includegraphics[width=0.95\linewidth]{images/implementation/user_interface/patient.png}
      \caption{\label{fig::UIPatient}User Interface for Patient Visualization Tools}
    \end{minipage}%
    \begin{minipage}{.5\textwidth}
      \centering
      \includegraphics[width=0.95\linewidth]{images/implementation/user_interface/segments.png}
      \caption{\label{fig::UIPatientSegments}User Interface for Patient specific Segments}
    \end{minipage}
  \end{figure}
\subsection{\label{sec::VoiceUserInterface}Voice User Interface}
\begin{center}
    \label{table::VoiceCommands}
    \begin{tabular}{ | l | p{4cm} | p{8cm} |}
    \hline
    Command & Action & Comment \\ \hline
    "Start" & Show first step of procedure & This command also disables the "Show all" command. \\ \hline
    "Show all" & Show all steps enumerated chronologicly & Is disabled via "Start" command. \newline
    Disables the use of 'step next' or 'step back'. \\ \hline
    "Step next" & Show next step of the procedure & \\ \hline
    "Step back" & Show previous step of the procedure & \\ \hline
    "Undo" & Reverts current step & Last undid step gets cached and can be redone via "Redo" command. \\ \hline
    "Redo" & Redo last reverted step & Only the last reverted step can be redone. \\ \hline
    "Train start" & Start train mode & Start at step 1 and navigate steps by following instructions and performing procedures in the correct manner.
    It also disables every other command except "Train stop" \\ \hline
    "Train stop" & Stop train mode & \\ \hline
    \end{tabular}
\end{center}

Table \ref{table::VoiceCommands} shows all Voice Commands.
\section{\label{sec::SurgicalSimulations}Surgical Simulations}
As previously described, the user interface consists of a GUI, VUI and natural hand interactions such as grabbing.
In the following, how users interact with the different kind of interfaces will be described. 

\subsection{\label{sec::GraphicalUserInterface}Graphical User Interface}
The first graphical interface which the user will see when pressing the button is depicted in Figure \ref{fig::UIProjectCase}
\begin{figure}[ht]
    \centering
    \includegraphics[width=200px]{images/implementation/user_interface/project_cases.png}
    \caption{\label{fig::UIProjectCase}User Interface for the Selection of Project Cases}
\end{figure}
The buttons for loading a case are created dynamically and depend on the number of project case JSON files found in the specified folder.
As will be explained later, the evaluation consists of five distinct scenarios in which distinct procedures have to be performed via the available tools.
By pressing the button by selecting it with the representation of the users hand in the virtual operating room, the project case will be loaded and the patient and information written down in the project case will be display from within the virtual operting room.
On the left side, different submenus can be selected.
Visualization tools can be selected via clicking on 'patient', as depicted in Figure \ref{fig::UIPatient}.
Here, tools which are the same for any project case can be used for visualization purposes.
Users have the option to scale the patient, look at the unprocessed patient for comparison and the ability to explode the 3D-Model to help with visualization.
Project cases can also be saved from this menu (Figure \ref{fig::UIPatient}).

\begin{figure}[h]
    \centering
    \begin{minipage}{.5\textwidth}
      \centering
      \includegraphics[width=0.95\linewidth]{images/implementation/user_interface/patient.png}
      \caption{\label{fig::UIPatient}User Interface for Patient Visualization Tools}
    \end{minipage}%
    \begin{minipage}{.5\textwidth}
      \centering
      \includegraphics[width=0.95\linewidth]{images/implementation/user_interface/segments.png}
      \caption{\label{fig::UIPatientSegments}User Interface for Patient specific Segments}
    \end{minipage}
  \end{figure}
\subsection{\label{sec::VoiceUserInterface}Voice User Interface}
\begin{center}
    \label{table::VoiceCommands}
    \begin{tabular}{ | l | p{4cm} | p{8cm} |}
    \hline
    Command & Action & Comment \\ \hline
    "Start" & Show first step of procedure & This command also disables the "Show all" command. \\ \hline
    "Show all" & Show all steps enumerated chronologicly & Is disabled via "Start" command. \newline
    Disables the use of 'step next' or 'step back'. \\ \hline
    "Step next" & Show next step of the procedure & \\ \hline
    "Step back" & Show previous step of the procedure & \\ \hline
    "Undo" & Reverts current step & Last undid step gets cached and can be redone via "Redo" command. \\ \hline
    "Redo" & Redo last reverted step & Only the last reverted step can be redone. \\ \hline
    "Train start" & Start train mode & Start at step 1 and navigate steps by following instructions and performing procedures in the correct manner.
    It also disables every other command except "Train stop" \\ \hline
    "Train stop" & Stop train mode & \\ \hline
    \end{tabular}
\end{center}

Table \ref{table::VoiceCommands} shows all Voice Commands.
\section{\label{sec::RelatedWorkDiscussion}Discussion - Limitations}
As previously described, the user interface consists of a GUI, VUI and natural hand interactions such as grabbing.
In the following, how users interact with the different kind of interfaces will be described. 

\subsection{\label{sec::GraphicalUserInterface}Graphical User Interface}
The first graphical interface which the user will see when pressing the button is depicted in Figure \ref{fig::UIProjectCase}
\begin{figure}[ht]
    \centering
    \includegraphics[width=200px]{images/implementation/user_interface/project_cases.png}
    \caption{\label{fig::UIProjectCase}User Interface for the Selection of Project Cases}
\end{figure}
The buttons for loading a case are created dynamically and depend on the number of project case JSON files found in the specified folder.
As will be explained later, the evaluation consists of five distinct scenarios in which distinct procedures have to be performed via the available tools.
By pressing the button by selecting it with the representation of the users hand in the virtual operating room, the project case will be loaded and the patient and information written down in the project case will be display from within the virtual operting room.
On the left side, different submenus can be selected.
Visualization tools can be selected via clicking on 'patient', as depicted in Figure \ref{fig::UIPatient}.
Here, tools which are the same for any project case can be used for visualization purposes.
Users have the option to scale the patient, look at the unprocessed patient for comparison and the ability to explode the 3D-Model to help with visualization.
Project cases can also be saved from this menu (Figure \ref{fig::UIPatient}).

\begin{figure}[h]
    \centering
    \begin{minipage}{.5\textwidth}
      \centering
      \includegraphics[width=0.95\linewidth]{images/implementation/user_interface/patient.png}
      \caption{\label{fig::UIPatient}User Interface for Patient Visualization Tools}
    \end{minipage}%
    \begin{minipage}{.5\textwidth}
      \centering
      \includegraphics[width=0.95\linewidth]{images/implementation/user_interface/segments.png}
      \caption{\label{fig::UIPatientSegments}User Interface for Patient specific Segments}
    \end{minipage}
  \end{figure}
\subsection{\label{sec::VoiceUserInterface}Voice User Interface}
\begin{center}
    \label{table::VoiceCommands}
    \begin{tabular}{ | l | p{4cm} | p{8cm} |}
    \hline
    Command & Action & Comment \\ \hline
    "Start" & Show first step of procedure & This command also disables the "Show all" command. \\ \hline
    "Show all" & Show all steps enumerated chronologicly & Is disabled via "Start" command. \newline
    Disables the use of 'step next' or 'step back'. \\ \hline
    "Step next" & Show next step of the procedure & \\ \hline
    "Step back" & Show previous step of the procedure & \\ \hline
    "Undo" & Reverts current step & Last undid step gets cached and can be redone via "Redo" command. \\ \hline
    "Redo" & Redo last reverted step & Only the last reverted step can be redone. \\ \hline
    "Train start" & Start train mode & Start at step 1 and navigate steps by following instructions and performing procedures in the correct manner.
    It also disables every other command except "Train stop" \\ \hline
    "Train stop" & Stop train mode & \\ \hline
    \end{tabular}
\end{center}

Table \ref{table::VoiceCommands} shows all Voice Commands.


\chapter{\label{chap::Approach}Approach}
As previously described, the user interface consists of a GUI, VUI and natural hand interactions such as grabbing.
In the following, how users interact with the different kind of interfaces will be described. 

\subsection{\label{sec::GraphicalUserInterface}Graphical User Interface}
The first graphical interface which the user will see when pressing the button is depicted in Figure \ref{fig::UIProjectCase}
\begin{figure}[ht]
    \centering
    \includegraphics[width=200px]{images/implementation/user_interface/project_cases.png}
    \caption{\label{fig::UIProjectCase}User Interface for the Selection of Project Cases}
\end{figure}
The buttons for loading a case are created dynamically and depend on the number of project case JSON files found in the specified folder.
As will be explained later, the evaluation consists of five distinct scenarios in which distinct procedures have to be performed via the available tools.
By pressing the button by selecting it with the representation of the users hand in the virtual operating room, the project case will be loaded and the patient and information written down in the project case will be display from within the virtual operting room.
On the left side, different submenus can be selected.
Visualization tools can be selected via clicking on 'patient', as depicted in Figure \ref{fig::UIPatient}.
Here, tools which are the same for any project case can be used for visualization purposes.
Users have the option to scale the patient, look at the unprocessed patient for comparison and the ability to explode the 3D-Model to help with visualization.
Project cases can also be saved from this menu (Figure \ref{fig::UIPatient}).

\begin{figure}[h]
    \centering
    \begin{minipage}{.5\textwidth}
      \centering
      \includegraphics[width=0.95\linewidth]{images/implementation/user_interface/patient.png}
      \caption{\label{fig::UIPatient}User Interface for Patient Visualization Tools}
    \end{minipage}%
    \begin{minipage}{.5\textwidth}
      \centering
      \includegraphics[width=0.95\linewidth]{images/implementation/user_interface/segments.png}
      \caption{\label{fig::UIPatientSegments}User Interface for Patient specific Segments}
    \end{minipage}
  \end{figure}
\subsection{\label{sec::VoiceUserInterface}Voice User Interface}
\begin{center}
    \label{table::VoiceCommands}
    \begin{tabular}{ | l | p{4cm} | p{8cm} |}
    \hline
    Command & Action & Comment \\ \hline
    "Start" & Show first step of procedure & This command also disables the "Show all" command. \\ \hline
    "Show all" & Show all steps enumerated chronologicly & Is disabled via "Start" command. \newline
    Disables the use of 'step next' or 'step back'. \\ \hline
    "Step next" & Show next step of the procedure & \\ \hline
    "Step back" & Show previous step of the procedure & \\ \hline
    "Undo" & Reverts current step & Last undid step gets cached and can be redone via "Redo" command. \\ \hline
    "Redo" & Redo last reverted step & Only the last reverted step can be redone. \\ \hline
    "Train start" & Start train mode & Start at step 1 and navigate steps by following instructions and performing procedures in the correct manner.
    It also disables every other command except "Train stop" \\ \hline
    "Train stop" & Stop train mode & \\ \hline
    \end{tabular}
\end{center}

Table \ref{table::VoiceCommands} shows all Voice Commands.
\section{\label{sec::RequirementAnalysis}Requirement Analysis (Scope of the thesis?)}
As previously described, the user interface consists of a GUI, VUI and natural hand interactions such as grabbing.
In the following, how users interact with the different kind of interfaces will be described. 

\subsection{\label{sec::GraphicalUserInterface}Graphical User Interface}
The first graphical interface which the user will see when pressing the button is depicted in Figure \ref{fig::UIProjectCase}
\begin{figure}[ht]
    \centering
    \includegraphics[width=200px]{images/implementation/user_interface/project_cases.png}
    \caption{\label{fig::UIProjectCase}User Interface for the Selection of Project Cases}
\end{figure}
The buttons for loading a case are created dynamically and depend on the number of project case JSON files found in the specified folder.
As will be explained later, the evaluation consists of five distinct scenarios in which distinct procedures have to be performed via the available tools.
By pressing the button by selecting it with the representation of the users hand in the virtual operating room, the project case will be loaded and the patient and information written down in the project case will be display from within the virtual operting room.
On the left side, different submenus can be selected.
Visualization tools can be selected via clicking on 'patient', as depicted in Figure \ref{fig::UIPatient}.
Here, tools which are the same for any project case can be used for visualization purposes.
Users have the option to scale the patient, look at the unprocessed patient for comparison and the ability to explode the 3D-Model to help with visualization.
Project cases can also be saved from this menu (Figure \ref{fig::UIPatient}).

\begin{figure}[h]
    \centering
    \begin{minipage}{.5\textwidth}
      \centering
      \includegraphics[width=0.95\linewidth]{images/implementation/user_interface/patient.png}
      \caption{\label{fig::UIPatient}User Interface for Patient Visualization Tools}
    \end{minipage}%
    \begin{minipage}{.5\textwidth}
      \centering
      \includegraphics[width=0.95\linewidth]{images/implementation/user_interface/segments.png}
      \caption{\label{fig::UIPatientSegments}User Interface for Patient specific Segments}
    \end{minipage}
  \end{figure}
\subsection{\label{sec::VoiceUserInterface}Voice User Interface}
\begin{center}
    \label{table::VoiceCommands}
    \begin{tabular}{ | l | p{4cm} | p{8cm} |}
    \hline
    Command & Action & Comment \\ \hline
    "Start" & Show first step of procedure & This command also disables the "Show all" command. \\ \hline
    "Show all" & Show all steps enumerated chronologicly & Is disabled via "Start" command. \newline
    Disables the use of 'step next' or 'step back'. \\ \hline
    "Step next" & Show next step of the procedure & \\ \hline
    "Step back" & Show previous step of the procedure & \\ \hline
    "Undo" & Reverts current step & Last undid step gets cached and can be redone via "Redo" command. \\ \hline
    "Redo" & Redo last reverted step & Only the last reverted step can be redone. \\ \hline
    "Train start" & Start train mode & Start at step 1 and navigate steps by following instructions and performing procedures in the correct manner.
    It also disables every other command except "Train stop" \\ \hline
    "Train stop" & Stop train mode & \\ \hline
    \end{tabular}
\end{center}

Table \ref{table::VoiceCommands} shows all Voice Commands.
\section{\label{sec::Concept}Concept}
As previously described, the user interface consists of a GUI, VUI and natural hand interactions such as grabbing.
In the following, how users interact with the different kind of interfaces will be described. 

\subsection{\label{sec::GraphicalUserInterface}Graphical User Interface}
The first graphical interface which the user will see when pressing the button is depicted in Figure \ref{fig::UIProjectCase}
\begin{figure}[ht]
    \centering
    \includegraphics[width=200px]{images/implementation/user_interface/project_cases.png}
    \caption{\label{fig::UIProjectCase}User Interface for the Selection of Project Cases}
\end{figure}
The buttons for loading a case are created dynamically and depend on the number of project case JSON files found in the specified folder.
As will be explained later, the evaluation consists of five distinct scenarios in which distinct procedures have to be performed via the available tools.
By pressing the button by selecting it with the representation of the users hand in the virtual operating room, the project case will be loaded and the patient and information written down in the project case will be display from within the virtual operting room.
On the left side, different submenus can be selected.
Visualization tools can be selected via clicking on 'patient', as depicted in Figure \ref{fig::UIPatient}.
Here, tools which are the same for any project case can be used for visualization purposes.
Users have the option to scale the patient, look at the unprocessed patient for comparison and the ability to explode the 3D-Model to help with visualization.
Project cases can also be saved from this menu (Figure \ref{fig::UIPatient}).

\begin{figure}[h]
    \centering
    \begin{minipage}{.5\textwidth}
      \centering
      \includegraphics[width=0.95\linewidth]{images/implementation/user_interface/patient.png}
      \caption{\label{fig::UIPatient}User Interface for Patient Visualization Tools}
    \end{minipage}%
    \begin{minipage}{.5\textwidth}
      \centering
      \includegraphics[width=0.95\linewidth]{images/implementation/user_interface/segments.png}
      \caption{\label{fig::UIPatientSegments}User Interface for Patient specific Segments}
    \end{minipage}
  \end{figure}
\subsection{\label{sec::VoiceUserInterface}Voice User Interface}
\begin{center}
    \label{table::VoiceCommands}
    \begin{tabular}{ | l | p{4cm} | p{8cm} |}
    \hline
    Command & Action & Comment \\ \hline
    "Start" & Show first step of procedure & This command also disables the "Show all" command. \\ \hline
    "Show all" & Show all steps enumerated chronologicly & Is disabled via "Start" command. \newline
    Disables the use of 'step next' or 'step back'. \\ \hline
    "Step next" & Show next step of the procedure & \\ \hline
    "Step back" & Show previous step of the procedure & \\ \hline
    "Undo" & Reverts current step & Last undid step gets cached and can be redone via "Redo" command. \\ \hline
    "Redo" & Redo last reverted step & Only the last reverted step can be redone. \\ \hline
    "Train start" & Start train mode & Start at step 1 and navigate steps by following instructions and performing procedures in the correct manner.
    It also disables every other command except "Train stop" \\ \hline
    "Train stop" & Stop train mode & \\ \hline
    \end{tabular}
\end{center}

Table \ref{table::VoiceCommands} shows all Voice Commands.

\chapter{\label{chap::Implementation}Implementation}
As previously described, the user interface consists of a GUI, VUI and natural hand interactions such as grabbing.
In the following, how users interact with the different kind of interfaces will be described. 

\subsection{\label{sec::GraphicalUserInterface}Graphical User Interface}
The first graphical interface which the user will see when pressing the button is depicted in Figure \ref{fig::UIProjectCase}
\begin{figure}[ht]
    \centering
    \includegraphics[width=200px]{images/implementation/user_interface/project_cases.png}
    \caption{\label{fig::UIProjectCase}User Interface for the Selection of Project Cases}
\end{figure}
The buttons for loading a case are created dynamically and depend on the number of project case JSON files found in the specified folder.
As will be explained later, the evaluation consists of five distinct scenarios in which distinct procedures have to be performed via the available tools.
By pressing the button by selecting it with the representation of the users hand in the virtual operating room, the project case will be loaded and the patient and information written down in the project case will be display from within the virtual operting room.
On the left side, different submenus can be selected.
Visualization tools can be selected via clicking on 'patient', as depicted in Figure \ref{fig::UIPatient}.
Here, tools which are the same for any project case can be used for visualization purposes.
Users have the option to scale the patient, look at the unprocessed patient for comparison and the ability to explode the 3D-Model to help with visualization.
Project cases can also be saved from this menu (Figure \ref{fig::UIPatient}).

\begin{figure}[h]
    \centering
    \begin{minipage}{.5\textwidth}
      \centering
      \includegraphics[width=0.95\linewidth]{images/implementation/user_interface/patient.png}
      \caption{\label{fig::UIPatient}User Interface for Patient Visualization Tools}
    \end{minipage}%
    \begin{minipage}{.5\textwidth}
      \centering
      \includegraphics[width=0.95\linewidth]{images/implementation/user_interface/segments.png}
      \caption{\label{fig::UIPatientSegments}User Interface for Patient specific Segments}
    \end{minipage}
  \end{figure}
\subsection{\label{sec::VoiceUserInterface}Voice User Interface}
\begin{center}
    \label{table::VoiceCommands}
    \begin{tabular}{ | l | p{4cm} | p{8cm} |}
    \hline
    Command & Action & Comment \\ \hline
    "Start" & Show first step of procedure & This command also disables the "Show all" command. \\ \hline
    "Show all" & Show all steps enumerated chronologicly & Is disabled via "Start" command. \newline
    Disables the use of 'step next' or 'step back'. \\ \hline
    "Step next" & Show next step of the procedure & \\ \hline
    "Step back" & Show previous step of the procedure & \\ \hline
    "Undo" & Reverts current step & Last undid step gets cached and can be redone via "Redo" command. \\ \hline
    "Redo" & Redo last reverted step & Only the last reverted step can be redone. \\ \hline
    "Train start" & Start train mode & Start at step 1 and navigate steps by following instructions and performing procedures in the correct manner.
    It also disables every other command except "Train stop" \\ \hline
    "Train stop" & Stop train mode & \\ \hline
    \end{tabular}
\end{center}

Table \ref{table::VoiceCommands} shows all Voice Commands.
\section{\label{sec::Features}Features}
As previously described, the user interface consists of a GUI, VUI and natural hand interactions such as grabbing.
In the following, how users interact with the different kind of interfaces will be described. 

\subsection{\label{sec::GraphicalUserInterface}Graphical User Interface}
The first graphical interface which the user will see when pressing the button is depicted in Figure \ref{fig::UIProjectCase}
\begin{figure}[ht]
    \centering
    \includegraphics[width=200px]{images/implementation/user_interface/project_cases.png}
    \caption{\label{fig::UIProjectCase}User Interface for the Selection of Project Cases}
\end{figure}
The buttons for loading a case are created dynamically and depend on the number of project case JSON files found in the specified folder.
As will be explained later, the evaluation consists of five distinct scenarios in which distinct procedures have to be performed via the available tools.
By pressing the button by selecting it with the representation of the users hand in the virtual operating room, the project case will be loaded and the patient and information written down in the project case will be display from within the virtual operting room.
On the left side, different submenus can be selected.
Visualization tools can be selected via clicking on 'patient', as depicted in Figure \ref{fig::UIPatient}.
Here, tools which are the same for any project case can be used for visualization purposes.
Users have the option to scale the patient, look at the unprocessed patient for comparison and the ability to explode the 3D-Model to help with visualization.
Project cases can also be saved from this menu (Figure \ref{fig::UIPatient}).

\begin{figure}[h]
    \centering
    \begin{minipage}{.5\textwidth}
      \centering
      \includegraphics[width=0.95\linewidth]{images/implementation/user_interface/patient.png}
      \caption{\label{fig::UIPatient}User Interface for Patient Visualization Tools}
    \end{minipage}%
    \begin{minipage}{.5\textwidth}
      \centering
      \includegraphics[width=0.95\linewidth]{images/implementation/user_interface/segments.png}
      \caption{\label{fig::UIPatientSegments}User Interface for Patient specific Segments}
    \end{minipage}
  \end{figure}
\subsection{\label{sec::VoiceUserInterface}Voice User Interface}
\begin{center}
    \label{table::VoiceCommands}
    \begin{tabular}{ | l | p{4cm} | p{8cm} |}
    \hline
    Command & Action & Comment \\ \hline
    "Start" & Show first step of procedure & This command also disables the "Show all" command. \\ \hline
    "Show all" & Show all steps enumerated chronologicly & Is disabled via "Start" command. \newline
    Disables the use of 'step next' or 'step back'. \\ \hline
    "Step next" & Show next step of the procedure & \\ \hline
    "Step back" & Show previous step of the procedure & \\ \hline
    "Undo" & Reverts current step & Last undid step gets cached and can be redone via "Redo" command. \\ \hline
    "Redo" & Redo last reverted step & Only the last reverted step can be redone. \\ \hline
    "Train start" & Start train mode & Start at step 1 and navigate steps by following instructions and performing procedures in the correct manner.
    It also disables every other command except "Train stop" \\ \hline
    "Train stop" & Stop train mode & \\ \hline
    \end{tabular}
\end{center}

Table \ref{table::VoiceCommands} shows all Voice Commands.
\section{\label{sec::Workflow}Workflow}
As previously described, the user interface consists of a GUI, VUI and natural hand interactions such as grabbing.
In the following, how users interact with the different kind of interfaces will be described. 

\subsection{\label{sec::GraphicalUserInterface}Graphical User Interface}
The first graphical interface which the user will see when pressing the button is depicted in Figure \ref{fig::UIProjectCase}
\begin{figure}[ht]
    \centering
    \includegraphics[width=200px]{images/implementation/user_interface/project_cases.png}
    \caption{\label{fig::UIProjectCase}User Interface for the Selection of Project Cases}
\end{figure}
The buttons for loading a case are created dynamically and depend on the number of project case JSON files found in the specified folder.
As will be explained later, the evaluation consists of five distinct scenarios in which distinct procedures have to be performed via the available tools.
By pressing the button by selecting it with the representation of the users hand in the virtual operating room, the project case will be loaded and the patient and information written down in the project case will be display from within the virtual operting room.
On the left side, different submenus can be selected.
Visualization tools can be selected via clicking on 'patient', as depicted in Figure \ref{fig::UIPatient}.
Here, tools which are the same for any project case can be used for visualization purposes.
Users have the option to scale the patient, look at the unprocessed patient for comparison and the ability to explode the 3D-Model to help with visualization.
Project cases can also be saved from this menu (Figure \ref{fig::UIPatient}).

\begin{figure}[h]
    \centering
    \begin{minipage}{.5\textwidth}
      \centering
      \includegraphics[width=0.95\linewidth]{images/implementation/user_interface/patient.png}
      \caption{\label{fig::UIPatient}User Interface for Patient Visualization Tools}
    \end{minipage}%
    \begin{minipage}{.5\textwidth}
      \centering
      \includegraphics[width=0.95\linewidth]{images/implementation/user_interface/segments.png}
      \caption{\label{fig::UIPatientSegments}User Interface for Patient specific Segments}
    \end{minipage}
  \end{figure}
\subsection{\label{sec::VoiceUserInterface}Voice User Interface}
\begin{center}
    \label{table::VoiceCommands}
    \begin{tabular}{ | l | p{4cm} | p{8cm} |}
    \hline
    Command & Action & Comment \\ \hline
    "Start" & Show first step of procedure & This command also disables the "Show all" command. \\ \hline
    "Show all" & Show all steps enumerated chronologicly & Is disabled via "Start" command. \newline
    Disables the use of 'step next' or 'step back'. \\ \hline
    "Step next" & Show next step of the procedure & \\ \hline
    "Step back" & Show previous step of the procedure & \\ \hline
    "Undo" & Reverts current step & Last undid step gets cached and can be redone via "Redo" command. \\ \hline
    "Redo" & Redo last reverted step & Only the last reverted step can be redone. \\ \hline
    "Train start" & Start train mode & Start at step 1 and navigate steps by following instructions and performing procedures in the correct manner.
    It also disables every other command except "Train stop" \\ \hline
    "Train stop" & Stop train mode & \\ \hline
    \end{tabular}
\end{center}

Table \ref{table::VoiceCommands} shows all Voice Commands.
\section{\label{sec::InteractionFlow}Interaction Flow}
As previously described, the user interface consists of a GUI, VUI and natural hand interactions such as grabbing.
In the following, how users interact with the different kind of interfaces will be described. 

\subsection{\label{sec::GraphicalUserInterface}Graphical User Interface}
The first graphical interface which the user will see when pressing the button is depicted in Figure \ref{fig::UIProjectCase}
\begin{figure}[ht]
    \centering
    \includegraphics[width=200px]{images/implementation/user_interface/project_cases.png}
    \caption{\label{fig::UIProjectCase}User Interface for the Selection of Project Cases}
\end{figure}
The buttons for loading a case are created dynamically and depend on the number of project case JSON files found in the specified folder.
As will be explained later, the evaluation consists of five distinct scenarios in which distinct procedures have to be performed via the available tools.
By pressing the button by selecting it with the representation of the users hand in the virtual operating room, the project case will be loaded and the patient and information written down in the project case will be display from within the virtual operting room.
On the left side, different submenus can be selected.
Visualization tools can be selected via clicking on 'patient', as depicted in Figure \ref{fig::UIPatient}.
Here, tools which are the same for any project case can be used for visualization purposes.
Users have the option to scale the patient, look at the unprocessed patient for comparison and the ability to explode the 3D-Model to help with visualization.
Project cases can also be saved from this menu (Figure \ref{fig::UIPatient}).

\begin{figure}[h]
    \centering
    \begin{minipage}{.5\textwidth}
      \centering
      \includegraphics[width=0.95\linewidth]{images/implementation/user_interface/patient.png}
      \caption{\label{fig::UIPatient}User Interface for Patient Visualization Tools}
    \end{minipage}%
    \begin{minipage}{.5\textwidth}
      \centering
      \includegraphics[width=0.95\linewidth]{images/implementation/user_interface/segments.png}
      \caption{\label{fig::UIPatientSegments}User Interface for Patient specific Segments}
    \end{minipage}
  \end{figure}
\subsection{\label{sec::VoiceUserInterface}Voice User Interface}
\begin{center}
    \label{table::VoiceCommands}
    \begin{tabular}{ | l | p{4cm} | p{8cm} |}
    \hline
    Command & Action & Comment \\ \hline
    "Start" & Show first step of procedure & This command also disables the "Show all" command. \\ \hline
    "Show all" & Show all steps enumerated chronologicly & Is disabled via "Start" command. \newline
    Disables the use of 'step next' or 'step back'. \\ \hline
    "Step next" & Show next step of the procedure & \\ \hline
    "Step back" & Show previous step of the procedure & \\ \hline
    "Undo" & Reverts current step & Last undid step gets cached and can be redone via "Redo" command. \\ \hline
    "Redo" & Redo last reverted step & Only the last reverted step can be redone. \\ \hline
    "Train start" & Start train mode & Start at step 1 and navigate steps by following instructions and performing procedures in the correct manner.
    It also disables every other command except "Train stop" \\ \hline
    "Train stop" & Stop train mode & \\ \hline
    \end{tabular}
\end{center}

Table \ref{table::VoiceCommands} shows all Voice Commands.
\section{\label{sec::UserInterface}User Interface}
As previously described, the user interface consists of a GUI, VUI and natural hand interactions such as grabbing.
In the following, how users interact with the different kind of interfaces will be described. 

\subsection{\label{sec::GraphicalUserInterface}Graphical User Interface}
The first graphical interface which the user will see when pressing the button is depicted in Figure \ref{fig::UIProjectCase}
\begin{figure}[ht]
    \centering
    \includegraphics[width=200px]{images/implementation/user_interface/project_cases.png}
    \caption{\label{fig::UIProjectCase}User Interface for the Selection of Project Cases}
\end{figure}
The buttons for loading a case are created dynamically and depend on the number of project case JSON files found in the specified folder.
As will be explained later, the evaluation consists of five distinct scenarios in which distinct procedures have to be performed via the available tools.
By pressing the button by selecting it with the representation of the users hand in the virtual operating room, the project case will be loaded and the patient and information written down in the project case will be display from within the virtual operting room.
On the left side, different submenus can be selected.
Visualization tools can be selected via clicking on 'patient', as depicted in Figure \ref{fig::UIPatient}.
Here, tools which are the same for any project case can be used for visualization purposes.
Users have the option to scale the patient, look at the unprocessed patient for comparison and the ability to explode the 3D-Model to help with visualization.
Project cases can also be saved from this menu (Figure \ref{fig::UIPatient}).

\begin{figure}[h]
    \centering
    \begin{minipage}{.5\textwidth}
      \centering
      \includegraphics[width=0.95\linewidth]{images/implementation/user_interface/patient.png}
      \caption{\label{fig::UIPatient}User Interface for Patient Visualization Tools}
    \end{minipage}%
    \begin{minipage}{.5\textwidth}
      \centering
      \includegraphics[width=0.95\linewidth]{images/implementation/user_interface/segments.png}
      \caption{\label{fig::UIPatientSegments}User Interface for Patient specific Segments}
    \end{minipage}
  \end{figure}
\subsection{\label{sec::VoiceUserInterface}Voice User Interface}
\begin{center}
    \label{table::VoiceCommands}
    \begin{tabular}{ | l | p{4cm} | p{8cm} |}
    \hline
    Command & Action & Comment \\ \hline
    "Start" & Show first step of procedure & This command also disables the "Show all" command. \\ \hline
    "Show all" & Show all steps enumerated chronologicly & Is disabled via "Start" command. \newline
    Disables the use of 'step next' or 'step back'. \\ \hline
    "Step next" & Show next step of the procedure & \\ \hline
    "Step back" & Show previous step of the procedure & \\ \hline
    "Undo" & Reverts current step & Last undid step gets cached and can be redone via "Redo" command. \\ \hline
    "Redo" & Redo last reverted step & Only the last reverted step can be redone. \\ \hline
    "Train start" & Start train mode & Start at step 1 and navigate steps by following instructions and performing procedures in the correct manner.
    It also disables every other command except "Train stop" \\ \hline
    "Train stop" & Stop train mode & \\ \hline
    \end{tabular}
\end{center}

Table \ref{table::VoiceCommands} shows all Voice Commands.

\chapter{\label{chap::Evaluation}Evaluation}
As previously described, the user interface consists of a GUI, VUI and natural hand interactions such as grabbing.
In the following, how users interact with the different kind of interfaces will be described. 

\subsection{\label{sec::GraphicalUserInterface}Graphical User Interface}
The first graphical interface which the user will see when pressing the button is depicted in Figure \ref{fig::UIProjectCase}
\begin{figure}[ht]
    \centering
    \includegraphics[width=200px]{images/implementation/user_interface/project_cases.png}
    \caption{\label{fig::UIProjectCase}User Interface for the Selection of Project Cases}
\end{figure}
The buttons for loading a case are created dynamically and depend on the number of project case JSON files found in the specified folder.
As will be explained later, the evaluation consists of five distinct scenarios in which distinct procedures have to be performed via the available tools.
By pressing the button by selecting it with the representation of the users hand in the virtual operating room, the project case will be loaded and the patient and information written down in the project case will be display from within the virtual operting room.
On the left side, different submenus can be selected.
Visualization tools can be selected via clicking on 'patient', as depicted in Figure \ref{fig::UIPatient}.
Here, tools which are the same for any project case can be used for visualization purposes.
Users have the option to scale the patient, look at the unprocessed patient for comparison and the ability to explode the 3D-Model to help with visualization.
Project cases can also be saved from this menu (Figure \ref{fig::UIPatient}).

\begin{figure}[h]
    \centering
    \begin{minipage}{.5\textwidth}
      \centering
      \includegraphics[width=0.95\linewidth]{images/implementation/user_interface/patient.png}
      \caption{\label{fig::UIPatient}User Interface for Patient Visualization Tools}
    \end{minipage}%
    \begin{minipage}{.5\textwidth}
      \centering
      \includegraphics[width=0.95\linewidth]{images/implementation/user_interface/segments.png}
      \caption{\label{fig::UIPatientSegments}User Interface for Patient specific Segments}
    \end{minipage}
  \end{figure}
\subsection{\label{sec::VoiceUserInterface}Voice User Interface}
\begin{center}
    \label{table::VoiceCommands}
    \begin{tabular}{ | l | p{4cm} | p{8cm} |}
    \hline
    Command & Action & Comment \\ \hline
    "Start" & Show first step of procedure & This command also disables the "Show all" command. \\ \hline
    "Show all" & Show all steps enumerated chronologicly & Is disabled via "Start" command. \newline
    Disables the use of 'step next' or 'step back'. \\ \hline
    "Step next" & Show next step of the procedure & \\ \hline
    "Step back" & Show previous step of the procedure & \\ \hline
    "Undo" & Reverts current step & Last undid step gets cached and can be redone via "Redo" command. \\ \hline
    "Redo" & Redo last reverted step & Only the last reverted step can be redone. \\ \hline
    "Train start" & Start train mode & Start at step 1 and navigate steps by following instructions and performing procedures in the correct manner.
    It also disables every other command except "Train stop" \\ \hline
    "Train stop" & Stop train mode & \\ \hline
    \end{tabular}
\end{center}

Table \ref{table::VoiceCommands} shows all Voice Commands.
\section{\label{sec::Methology}System Usability Study}
As previously described, the user interface consists of a GUI, VUI and natural hand interactions such as grabbing.
In the following, how users interact with the different kind of interfaces will be described. 

\subsection{\label{sec::GraphicalUserInterface}Graphical User Interface}
The first graphical interface which the user will see when pressing the button is depicted in Figure \ref{fig::UIProjectCase}
\begin{figure}[ht]
    \centering
    \includegraphics[width=200px]{images/implementation/user_interface/project_cases.png}
    \caption{\label{fig::UIProjectCase}User Interface for the Selection of Project Cases}
\end{figure}
The buttons for loading a case are created dynamically and depend on the number of project case JSON files found in the specified folder.
As will be explained later, the evaluation consists of five distinct scenarios in which distinct procedures have to be performed via the available tools.
By pressing the button by selecting it with the representation of the users hand in the virtual operating room, the project case will be loaded and the patient and information written down in the project case will be display from within the virtual operting room.
On the left side, different submenus can be selected.
Visualization tools can be selected via clicking on 'patient', as depicted in Figure \ref{fig::UIPatient}.
Here, tools which are the same for any project case can be used for visualization purposes.
Users have the option to scale the patient, look at the unprocessed patient for comparison and the ability to explode the 3D-Model to help with visualization.
Project cases can also be saved from this menu (Figure \ref{fig::UIPatient}).

\begin{figure}[h]
    \centering
    \begin{minipage}{.5\textwidth}
      \centering
      \includegraphics[width=0.95\linewidth]{images/implementation/user_interface/patient.png}
      \caption{\label{fig::UIPatient}User Interface for Patient Visualization Tools}
    \end{minipage}%
    \begin{minipage}{.5\textwidth}
      \centering
      \includegraphics[width=0.95\linewidth]{images/implementation/user_interface/segments.png}
      \caption{\label{fig::UIPatientSegments}User Interface for Patient specific Segments}
    \end{minipage}
  \end{figure}
\subsection{\label{sec::VoiceUserInterface}Voice User Interface}
\begin{center}
    \label{table::VoiceCommands}
    \begin{tabular}{ | l | p{4cm} | p{8cm} |}
    \hline
    Command & Action & Comment \\ \hline
    "Start" & Show first step of procedure & This command also disables the "Show all" command. \\ \hline
    "Show all" & Show all steps enumerated chronologicly & Is disabled via "Start" command. \newline
    Disables the use of 'step next' or 'step back'. \\ \hline
    "Step next" & Show next step of the procedure & \\ \hline
    "Step back" & Show previous step of the procedure & \\ \hline
    "Undo" & Reverts current step & Last undid step gets cached and can be redone via "Redo" command. \\ \hline
    "Redo" & Redo last reverted step & Only the last reverted step can be redone. \\ \hline
    "Train start" & Start train mode & Start at step 1 and navigate steps by following instructions and performing procedures in the correct manner.
    It also disables every other command except "Train stop" \\ \hline
    "Train stop" & Stop train mode & \\ \hline
    \end{tabular}
\end{center}

Table \ref{table::VoiceCommands} shows all Voice Commands.
\section{\label{sec::Results}Results}
As previously described, the user interface consists of a GUI, VUI and natural hand interactions such as grabbing.
In the following, how users interact with the different kind of interfaces will be described. 

\subsection{\label{sec::GraphicalUserInterface}Graphical User Interface}
The first graphical interface which the user will see when pressing the button is depicted in Figure \ref{fig::UIProjectCase}
\begin{figure}[ht]
    \centering
    \includegraphics[width=200px]{images/implementation/user_interface/project_cases.png}
    \caption{\label{fig::UIProjectCase}User Interface for the Selection of Project Cases}
\end{figure}
The buttons for loading a case are created dynamically and depend on the number of project case JSON files found in the specified folder.
As will be explained later, the evaluation consists of five distinct scenarios in which distinct procedures have to be performed via the available tools.
By pressing the button by selecting it with the representation of the users hand in the virtual operating room, the project case will be loaded and the patient and information written down in the project case will be display from within the virtual operting room.
On the left side, different submenus can be selected.
Visualization tools can be selected via clicking on 'patient', as depicted in Figure \ref{fig::UIPatient}.
Here, tools which are the same for any project case can be used for visualization purposes.
Users have the option to scale the patient, look at the unprocessed patient for comparison and the ability to explode the 3D-Model to help with visualization.
Project cases can also be saved from this menu (Figure \ref{fig::UIPatient}).

\begin{figure}[h]
    \centering
    \begin{minipage}{.5\textwidth}
      \centering
      \includegraphics[width=0.95\linewidth]{images/implementation/user_interface/patient.png}
      \caption{\label{fig::UIPatient}User Interface for Patient Visualization Tools}
    \end{minipage}%
    \begin{minipage}{.5\textwidth}
      \centering
      \includegraphics[width=0.95\linewidth]{images/implementation/user_interface/segments.png}
      \caption{\label{fig::UIPatientSegments}User Interface for Patient specific Segments}
    \end{minipage}
  \end{figure}
\subsection{\label{sec::VoiceUserInterface}Voice User Interface}
\begin{center}
    \label{table::VoiceCommands}
    \begin{tabular}{ | l | p{4cm} | p{8cm} |}
    \hline
    Command & Action & Comment \\ \hline
    "Start" & Show first step of procedure & This command also disables the "Show all" command. \\ \hline
    "Show all" & Show all steps enumerated chronologicly & Is disabled via "Start" command. \newline
    Disables the use of 'step next' or 'step back'. \\ \hline
    "Step next" & Show next step of the procedure & \\ \hline
    "Step back" & Show previous step of the procedure & \\ \hline
    "Undo" & Reverts current step & Last undid step gets cached and can be redone via "Redo" command. \\ \hline
    "Redo" & Redo last reverted step & Only the last reverted step can be redone. \\ \hline
    "Train start" & Start train mode & Start at step 1 and navigate steps by following instructions and performing procedures in the correct manner.
    It also disables every other command except "Train stop" \\ \hline
    "Train stop" & Stop train mode & \\ \hline
    \end{tabular}
\end{center}

Table \ref{table::VoiceCommands} shows all Voice Commands.

\chapter{\label{chap::Discussion}Discussion}
As previously described, the user interface consists of a GUI, VUI and natural hand interactions such as grabbing.
In the following, how users interact with the different kind of interfaces will be described. 

\subsection{\label{sec::GraphicalUserInterface}Graphical User Interface}
The first graphical interface which the user will see when pressing the button is depicted in Figure \ref{fig::UIProjectCase}
\begin{figure}[ht]
    \centering
    \includegraphics[width=200px]{images/implementation/user_interface/project_cases.png}
    \caption{\label{fig::UIProjectCase}User Interface for the Selection of Project Cases}
\end{figure}
The buttons for loading a case are created dynamically and depend on the number of project case JSON files found in the specified folder.
As will be explained later, the evaluation consists of five distinct scenarios in which distinct procedures have to be performed via the available tools.
By pressing the button by selecting it with the representation of the users hand in the virtual operating room, the project case will be loaded and the patient and information written down in the project case will be display from within the virtual operting room.
On the left side, different submenus can be selected.
Visualization tools can be selected via clicking on 'patient', as depicted in Figure \ref{fig::UIPatient}.
Here, tools which are the same for any project case can be used for visualization purposes.
Users have the option to scale the patient, look at the unprocessed patient for comparison and the ability to explode the 3D-Model to help with visualization.
Project cases can also be saved from this menu (Figure \ref{fig::UIPatient}).

\begin{figure}[h]
    \centering
    \begin{minipage}{.5\textwidth}
      \centering
      \includegraphics[width=0.95\linewidth]{images/implementation/user_interface/patient.png}
      \caption{\label{fig::UIPatient}User Interface for Patient Visualization Tools}
    \end{minipage}%
    \begin{minipage}{.5\textwidth}
      \centering
      \includegraphics[width=0.95\linewidth]{images/implementation/user_interface/segments.png}
      \caption{\label{fig::UIPatientSegments}User Interface for Patient specific Segments}
    \end{minipage}
  \end{figure}
\subsection{\label{sec::VoiceUserInterface}Voice User Interface}
\begin{center}
    \label{table::VoiceCommands}
    \begin{tabular}{ | l | p{4cm} | p{8cm} |}
    \hline
    Command & Action & Comment \\ \hline
    "Start" & Show first step of procedure & This command also disables the "Show all" command. \\ \hline
    "Show all" & Show all steps enumerated chronologicly & Is disabled via "Start" command. \newline
    Disables the use of 'step next' or 'step back'. \\ \hline
    "Step next" & Show next step of the procedure & \\ \hline
    "Step back" & Show previous step of the procedure & \\ \hline
    "Undo" & Reverts current step & Last undid step gets cached and can be redone via "Redo" command. \\ \hline
    "Redo" & Redo last reverted step & Only the last reverted step can be redone. \\ \hline
    "Train start" & Start train mode & Start at step 1 and navigate steps by following instructions and performing procedures in the correct manner.
    It also disables every other command except "Train stop" \\ \hline
    "Train stop" & Stop train mode & \\ \hline
    \end{tabular}
\end{center}

Table \ref{table::VoiceCommands} shows all Voice Commands.

\chapter{\label{chap::Conclusion}Conclusion}
As previously described, the user interface consists of a GUI, VUI and natural hand interactions such as grabbing.
In the following, how users interact with the different kind of interfaces will be described. 

\subsection{\label{sec::GraphicalUserInterface}Graphical User Interface}
The first graphical interface which the user will see when pressing the button is depicted in Figure \ref{fig::UIProjectCase}
\begin{figure}[ht]
    \centering
    \includegraphics[width=200px]{images/implementation/user_interface/project_cases.png}
    \caption{\label{fig::UIProjectCase}User Interface for the Selection of Project Cases}
\end{figure}
The buttons for loading a case are created dynamically and depend on the number of project case JSON files found in the specified folder.
As will be explained later, the evaluation consists of five distinct scenarios in which distinct procedures have to be performed via the available tools.
By pressing the button by selecting it with the representation of the users hand in the virtual operating room, the project case will be loaded and the patient and information written down in the project case will be display from within the virtual operting room.
On the left side, different submenus can be selected.
Visualization tools can be selected via clicking on 'patient', as depicted in Figure \ref{fig::UIPatient}.
Here, tools which are the same for any project case can be used for visualization purposes.
Users have the option to scale the patient, look at the unprocessed patient for comparison and the ability to explode the 3D-Model to help with visualization.
Project cases can also be saved from this menu (Figure \ref{fig::UIPatient}).

\begin{figure}[h]
    \centering
    \begin{minipage}{.5\textwidth}
      \centering
      \includegraphics[width=0.95\linewidth]{images/implementation/user_interface/patient.png}
      \caption{\label{fig::UIPatient}User Interface for Patient Visualization Tools}
    \end{minipage}%
    \begin{minipage}{.5\textwidth}
      \centering
      \includegraphics[width=0.95\linewidth]{images/implementation/user_interface/segments.png}
      \caption{\label{fig::UIPatientSegments}User Interface for Patient specific Segments}
    \end{minipage}
  \end{figure}
\subsection{\label{sec::VoiceUserInterface}Voice User Interface}
\begin{center}
    \label{table::VoiceCommands}
    \begin{tabular}{ | l | p{4cm} | p{8cm} |}
    \hline
    Command & Action & Comment \\ \hline
    "Start" & Show first step of procedure & This command also disables the "Show all" command. \\ \hline
    "Show all" & Show all steps enumerated chronologicly & Is disabled via "Start" command. \newline
    Disables the use of 'step next' or 'step back'. \\ \hline
    "Step next" & Show next step of the procedure & \\ \hline
    "Step back" & Show previous step of the procedure & \\ \hline
    "Undo" & Reverts current step & Last undid step gets cached and can be redone via "Redo" command. \\ \hline
    "Redo" & Redo last reverted step & Only the last reverted step can be redone. \\ \hline
    "Train start" & Start train mode & Start at step 1 and navigate steps by following instructions and performing procedures in the correct manner.
    It also disables every other command except "Train stop" \\ \hline
    "Train stop" & Stop train mode & \\ \hline
    \end{tabular}
\end{center}

Table \ref{table::VoiceCommands} shows all Voice Commands.

\bibliography{ms_thesis,ms_endnote}
%\bibliographystyle{apalike}
\bibliographystyle{ieeetr}

\end{document} 