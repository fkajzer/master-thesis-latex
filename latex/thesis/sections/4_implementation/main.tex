After the thorough requirement analysis in Section \ref{sec::RequirementAnalysis} on which a concept for the software was derived in \ref{sec::Concept}, 
we can now discuss the implementation of the system components.
We will start with the work- and interaction flow of the system.
Afterwards, we will shortly discuss architectural components as well as utilities provided by SteamVR.
Later, available commands and the concrete implementation details of surgical procedures and visualization tools will be discussed.

1.) Functionality
FEATURES
Visualization
Procedures
Training
1.1) What can this software do? (Features -> siehe Chapter Approach)
1.2) How was this implemented? (Technisch)

2.) Workflow
JSON UML Foto
3.) User Interface
3.1) Graphical
JEDES menü 1 FOTO
3.1) Voice
Tabelle mit allen Befehlen + beschreibung was wie wo

Architecture Kapitel 4
Software mit Objektdiagrammen darstellen und textlich beschreiben
 SteamVR -> Architecture

