\begin{center}
    \label{table::VoiceCommands}
    \begin{tabular}{ | l | p{4cm} | p{8cm} |}
    \hline
    Command & Action & Comment \\ \hline
    "Start" & Show first step of procedure & This command also disables the "Show all" command. \\ \hline
    "Show all" & Show all steps enumerated chronologicly & Is disabled via "Start" command. \newline
    Disables the use of 'step next' or 'step back'. \\ \hline
    "Step next" & Show next step of the procedure & \\ \hline
    "Step back" & Show previous step of the procedure & \\ \hline
    "Undo" & Reverts current step & Last undid step gets cached and can be redone via "Redo" command. \\ \hline
    "Redo" & Redo last reverted step & Only the last reverted step can be redone. \\ \hline
    "Train start" & Start train mode & Start at step 1 and navigate steps by following instructions and performing procedures in the correct manner.
    It also disables every other command except "Train stop" \\ \hline
    "Train stop" & Stop train mode & \\ \hline
    \end{tabular}
\end{center}

The voice commands have been implemented in the context of users having surgical instruments in their hands \ref{table::VoiceCommands}.
While performing procedures, the user can most importantly directly undo current actions.
This way, a natural flow of working on the procedures is guaranteed.
The voice user interface is strongly adapted to the AR and VR workflow, meaning the registrated commands and their functions are the same, even though the implementation might differ.
For the VR part, undo, redo and train start and train stop is unique since they are not part of the workflow but rather special to the VR application.
With the start command, the procedure is started in planning mode and the first procedure is shown.
This is only relevant if a previously edited case is being loaded, since new cases will not have any procedure steps in the project case.
By saying show all, an enumerated overview over all steps will be shown. This overview is by saying start once again to resume planning the procedure.
By using the step next and step back commands, users can navigate through the procedure and trace the chronological procedure sequence.
At any point of the planned procedure, users can use the undo command to remove the currently shown step from the procedure.
Additonally, the last undid step will be saved if a user decides that the deletion of the step was an error.
Undid steps can be redone via the "redo" command.
Important to note is that only the last undid step can be redone at the moment.
Via Train start, train mode as described in detail in Section \ref{sec::ImplementationTraining} will start.
Train mode is disabled either by finishing the whole procedure in train mode, or by revoking the "train stop" command to end it prematurely.