At this point, six procedures are implemented in the system.
However, as described in Section \ref{sec::Architecture}, the system was designed with extensibility in mind, so that new instruments can be implemented seemlessly.
Currently, the implemented instruments are based on the workflow of OMF surgeons of the UHA: Users can perform drilling, hammer and chisel, bonesaw and milling operations (Requirement \ref{req::F3.6}).
Additionally, users can place markings and osteosynthesis plates on the virtual patient.
\\ To perform procedures, a project case has to be loaded first, as described in Section \ref{sec::GraphicalUserInterface}.
In the sense of SteamVR's interaction system described in Section \ref{sec::Architecture}, for each procedure there is an 'indicate' action when the user touches the 
'perform' button, as well as a 'perform' action when the user presses the button all the way.
This way, users get visual feedback, when they are about to perform a procedure, as well as having visual indications of where the procedure will start (i.e. tip of the surgical instrument) (Requirement \ref{req::N3}).
\\ Each procedure will add a step to the project case.
Procedures consist of either a copy of the object concerning the current procedure, meaning the geometry of i.e. the drill bit, chisel, or osteosynthesis plate.
In other cases, simple geometry indicates to the user where a procedure is to be done. This is the case for the sawing, marker and milling procedures, as will be elaborated upon 
when describing procedure specifics later.
The steps are added to the hierarchy of the patient's 3D model and are identified as a step by their name.
\\ In the sense of the VR-AR-based workflow, project cases can then be saved in the VR part, and then loaded into both parts of the workflow (Requirement \ref{req::F6}).
Through using this kind of approach, extensibility is guaranteed as each new instrument simply has to add some kind of geometry as a step to the project case (Requirements \ref{req::N8},\ref{req::F3.7}).
It follows that any kind of procedure can then be imported into the AR workflow, even without touching the application (Requirement \ref{req::F6}).
\\ When a procedure is performed, users get voice feedback in the form of "step next", confirming that a step has been added to the project case (Requirement \ref{req::N3}).
Users can also navigate the project case's steps by using the VUIs commands (Section \ref{sec::VoiceUserInterface}), so that navigation through the steps of the procedure can be done 
while holding surgical instruments (Requirements \ref{req::N1},\ref{req::F3.7}).
\\ For some of the surgical instruments, the user representation of the hand will be shown while holding the instrument, for others the hand is hidden.
When a surgical instrument is placed into the virtual hand, the position of the surgical instrument is always fixed.
This way, it is guaranteed that the instrument will always be positioned in the same way when it is grabbed, meaning the handgrip will always be the same.
However, in some cases, for example sawing with the bonesaw, this feature would prevent users from switching the handgrip of the surgical instrument (Requirement \ref{req::N5}). 
Therefore, for some instruments, this feature was removed.
Users will not see their virtual hands on the surgical instrument, but therefore can choose to hold it with any grip they see fit.
The virtual hands will be hidden while holding the instrument.
However, the instrument still represents the users hand position.
The decision if hands should be hidden when grabbing an instrument was made by a trial and error approach with the help of a physician's opinion on whether 
this features was useful (Requirement \ref{req::N5}).
The procedure specific implementation will be thoroughly described in the following.

\begin{figure}[ht]
    \centering
    \includegraphics[width=200px]{images/implementation/features/procedures/drilling.png}
    \caption{\label{fig::FeatureDrilling} Drilling Procedure}
\end{figure}

The drilling operation is performed by picking up the bonesaw (Figure \ref{fig::FeatureDrilling}).
In total, there are fifteen attachments which can be used for the drilling procedure.

\begin{figure}[ht]
    \centering
    \includegraphics[width=200px]{images/implementation/features/procedures/drilling_attachment.png}
    \caption{\label{fig::FeatureDrillingAttachments} Attaching an Attachment to the Surgical Drill}
\end{figure}

Attachments get attached to the virtual drill be holding the indicate button of the controller in which hand the drill is currently located and moving the attachment to the indicated circle \ref{fig::FeatureDrillingAttachments}.
To perform the procedure, an attachment must be currently attached to the virtual drill.
Performing the virtual procedure will create a copy of the current attachment with a different material to indicate that is is part of the project case.
To increase performance in the AR and VR workflow, since the AR hardware is not as capable as high end computers which are needed for VR, the resolution of the attachments has been decreased at the processing step as described at the start of this section.
\paragraph{Chiseling}

The \textbf{chiseling} procedure has two parts to it.
First, with one hand a chisel has to be chosen.
Users have a choice between a small, medium, large and extra large chisel to perform the procedure.
With the other hand, users then have to pick up the hammer.
Since this procedure requires users to hold two surgical instruments at the same time, this procedure can get cumbersome.
However, users can easily avoid this by placing the instruments on the operating table in the middle of the OT and repositioning afterwards (Figure \ref{fig::ChiselPrepare}).

\begin{figure}[ht]
    \centering
    \includegraphics[width=\linewidth]{images/implementation/features/procedures/chisel_prepare.png}
    \caption{\label{fig::ChiselPrepare}The user prepares for the chiseling procedure by placing the instruments on the OT where the patient is located.
    indicators with the hammer in the other hand to perform the procedure.}
\end{figure}

When users have the perfect viewpoint, they can take up both instruments once again and start the procedure.
By pressing the indicate button on the hand where the chisel is located, rectangular indications at the top and bottom end of the chisel are shown to the user.
While these indications are active, the user has to perform a hammering motion with the hand holding the hammer.

\begin{figure}[ht]
    \centering
    \begin{minipage}{.5\textwidth}
      \centering
      \includegraphics[width=0.99\linewidth]{images/implementation/features/procedures/chisel_1.png}
    \end{minipage}%
    \begin{minipage}{.5\textwidth}
      \centering
      \includegraphics[width=0.99\linewidth]{images/implementation/features/procedures/chisel_2.png}
    \end{minipage}
    \caption{\label{fig::ChiselProcedure}Process of the chiseling procedure. Left: The users uses the indicate action with his left hand in preparation for the procedure. Right: After hammering on the indication on the chisel, the procedure has been performed and a step is generated.}
\end{figure}

When they hit the rectangular indicators located on the chisel, the chiseling procedure step is added to the project case in form of a modified copy of the hold chisel (Figure \ref{fig::ChiselProcedure}).
Here, information about the performed step is also included in the form of chisel size used for the procedure. 
\paragraph{Sawing}

\begin{figure}[ht]
    \centering
    \includegraphics[width=200px]{images/implementation/features/procedures/bonesaw.png}
    \caption{\label{fig::FeatureBoneSaw}Bonesaw procedure. Left: The procedure is started by pressing down the "perform" button. Indications show two starting points which are 
    used to create the plane when letting go of the button. Right: The button has been released and as plane visualizing the procedure has been created.}
\end{figure}

The \textbf{sawing} procedure is performed by picking up the bonesaw.
The indications to the user consist of two spheres indicating the start and end of the cutting plane (Figure \ref{fig::FeatureBoneSaw}).
These indicators are placed at the top and bottom end of the cutting portion of the bonesaws blade.
The procedure is triggered by first pressing down the "perform" button and holding it.
When letting go of the button, a two dimensional plane is created in the three dimensional space by using four points.
Two of these points are stored internally when pressing down, and the other two when letting go of the trigger button.
A plane is then created and added to the project case.

\begin{figure}[ht]
    \centering
    \includegraphics[width=200px]{images/todo.png}
    \caption{\label{fig::MultipleSawing}Cutting of the lower jaw simulated by performing two sawing procedures in succession.}
\end{figure}

Through following the plane with the bonesaw, the user can reproduce the way in which the bonesaw has been moved.
More complex cutting procedures can be simulated by breaking them down into two-dimensional shapes and performing multiple operational steps as depicted in Figure \ref{fig::MultipleSawing}.
\paragraph{Milling}

\begin{figure}[ht]
    \centering
    \includegraphics[width=200px]{images/implementation/features/procedures/piezo.png}
    \caption{\label{fig::FeaturePiezo}Milling procedure. Holding down the trigger button will draw little spheres until the button is released. The resulting object represents the volumetric space which is to be milled.}
\end{figure}

The \textbf{milling} operation is performed with the by grabbing the piezo instrument.
The indicator for the piezo is at the tip of the instrument, indicating which area will be milled.
While the "perform" button is being held down, little spheres are being drawn at the tip of the instrument \ref{fig::FeaturePiezo}.
When the button is released, the shapes are combined into a single 3D model and added as a project step.
The resulting object represents the volumetric space which has to be milled in the procedure.
The procedure can be reconstructed by "milling" the same 3D space in the virtual operating room.
\paragraph{Osteosynthesis plates}

\begin{figure}[ht]
    \centering
    \includegraphics[width=\linewidth]{images/implementation/features/procedures/osteo_overview.png}
    \caption{\label{fig::FeatureMetalPlate} Over view of all available steosynthesis plates. Users can choose from four reconstructional plates, 29 1.5mm plates and 20 2mm plates to apply in the procedure.}
\end{figure}

The \textbf{osteosynthesis plates} procedure consists of two steps before adding it as a step to the project case.
First, users have to chose which plate to use.
The user can chose from four reconstruction plates, 29 1.5mm plates and 20 2.0mm plates (Figure \ref{fig::FeatureMetalPlate}).
The optimal plates to use vary due to the pathology of the patient and the previously performed procedures.
After selecting the proper plate, a number of indicators will appear to the user (Figure \ref{fig::FeatureMetalPlate2}).

\begin{figure}
  \centering
  \begin{minipage}{.5\textwidth}
    \centering
    \includegraphics[width=0.997\linewidth]{images/implementation/features/procedures/osteo_1.png}
  \end{minipage}%
  \begin{minipage}{.5\textwidth}
    \centering
    \includegraphics[width=0.997\linewidth]{images/implementation/features/procedures/osteo_2.png}
  \end{minipage}
  \begin{minipage}{.5\textwidth}
    \centering
    \includegraphics[width=0.997\linewidth]{images/implementation/features/procedures/osteo_3.png}
  \end{minipage}%
  \begin{minipage}{.5\textwidth}
    \centering
    \includegraphics[width=0.997\linewidth]{images/implementation/features/procedures/osteo_4.png}
  \end{minipage}
  \caption{\label{fig::FeatureMetalPlate2}Osteosynthesis Plates modifications. User can move and rotate control points to perform modifications to the plate's geometry. Here, the user applies a number of modifications by altering the position and rotation of the control points. After the plate has been adjusted to the patient's anatomy, it is placed onto the patient.} 
\end{figure}

In the context of the osteosynthesis plates, these indicators are "control points", with which the user can bend and twist the plates.
These control points were set manually according to each of the osteosynthesises plates after investigating how different amounts of control points would affect the geometry.
The control points act as knots in a spline based calculation of the geometries mesh.
Bending and twisting is performed by choosing a control point via hovering over them with the user's free hand and grabbing them.
Then, the user has to move and rotate the control point in the desired manner (Figure \ref{fig::FeatureMetalPlate2}).

\begin{figure}[ht!]
    \centering
    \includegraphics[width=\linewidth]{images/implementation/features/procedures/marker.png}
    \caption{\label{fig::FeatureDrilling} Marker Procedure}
\end{figure}