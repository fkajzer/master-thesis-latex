\begin{figure}[ht!]
    \centering
    \includegraphics[width=\linewidth]{images/project_plan.png}
    \caption{\label{fig::ProjectPlan} Complete VR and AR workflow for surgeons concept}
\end{figure}

This VR-application is in the context of a AR/VR-based surgical workflow as depicted in Figure \ref{fig::ProjectPlan}.
OMFS-Surgeons can plan procedures with familiar surgical instruments on realistic representations of patients.
Workflows can be imported and exported via a file in JSON-notation.
The system uses a mixture of natural hand gestures like grabbing objects with voice commands for navigation.
The patient can be freely scaled, rotated and moved for visualization purposes.
Patients can be resetted to the default position on the operating desk at any time.

\begin{enumerate}
    \item Acquisition of medical imaging (CT, DCT, MRT\ldots)
    \item Segmentation of volumetric data (DICOM) into 3D model data (STL, OBJ\ldots)
    \item Definition of project case in JSON-notation
    \item Import of project case
    \item Planning procedures in the context of project case
    \item Optional: Simulate procedure for learning and training purposes
    \item Save project case for use in AR or share with other surgeons
\end{enumerate}

The workflow of the software is as depicted in the enumeration above.
Important to note is that the first two steps of the workflow are not part of this thesis, as they are kindly provided by the Oral and Maxillofacial Department of the UHA.
Step one and two of the workflow are especially critical, as medical imaging techniques have their own advantages and disadvantages.
Users performing the medical imaging acquisition have make expert decisions about which techniques to use to highlight the patient specific tissue.
Also, while segmenting the medical imaging in step two, the decision about which tissue has which color and which tissue will be transparant is made.
Specific features of the software are dependant on decision made in these two steps.

The project case is at the core of the workflow since it acts as the interface between the AR- and VR-components of the workflow.
The specifics about the JSON-notation have already been agreed and decided upon in the conceptualization phase of the project, since it is critical for a smooth interaction between AR and VR. 

The case info contains patient specific information which may be important for the procedure as depicted in the listing above.
These informations, aswell as the location of the patients 3D data on the hard drive has to be manually input by the user.
Information about the steps will be automatically generated while perfoming procedures inside of the application, however the user can then later add additional information as he sees fit.
The automatically generated information about the steps only contains information about which instrument was used at which step of the procedure.

When a project case is saved from within the application, a copy of the current patient with all steps which were performed is made withing the root folder of the patient model.
This way, the patients model is never lost and we can always start the procedure from scratch if desired.
Also, the timestamp of the updated field inside of the case info will be set.

By simply adding the planned procedures as additional 3D-Data to the patient, extensibility and robustness of the software is ensured.
New instruments and thus procedures can easily be added into the software without touching the project cases and everything will continue working as expected.
In comparison, if an approach where procedures have to be defined from within the project case was used, project cases would have to be updated every time we want to add new instruments to the application.

After setting up the project case, it can be loaded up at runtime inside of the VR-application.
Users can then view, add and edit existing procedures or create new ones.
If users wish to run through the planned procedure step by step, they can either navigate using the voice user interface or by enabling 'train mode' in which the user will be guided step by step through the procedure.
Here, users will be visually guided by the outlines of a surgical instrument to know where a procedure has to be performed.
Additionally, users will be guided via text inside of the application.
The text which is shown will be automatically generated with information about the current surgical instrument when performing a new procedure.
However, users are also free to edit the textual guidance manually via the JSON file in any way they see fit.
To confirm if the current procedure is carried out in the desired manner, voice feedback will notify the user if procedures are carried out correctly or if errors are being made.
The user can decide to save the current procedure for later use at any time of the process.