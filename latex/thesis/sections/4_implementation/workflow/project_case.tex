The project case is at the core of the workflow since it acts as the interface between the AR- and VR-components of the workflow.
The specifics about the JSON-notation have already been agreed and decided upon in the conceptualization phase of the project, since it is critical for a smooth interaction between AR and VR. 

The case info contains patient specific information which may be important for the procedure as depicted in the listing above.
These informations, aswell as the location of the patients 3D data on the hard drive has to be manually input by the user.
Information about the steps will be automatically generated while perfoming procedures inside of the application, however the user can then later add additional information as he sees fit.
The automatically generated information about the steps only contains information about which instrument was used at which step of the procedure.

When a project case is saved from within the application, a copy of the current patient with all steps which were performed is made withing the root folder of the patient model.
This way, the patients model is never lost and we can always start the procedure from scratch if desired.
Also, the timestamp of the updated field inside of the case info will be set.

By simply adding the planned procedures as additional 3D-Data to the patient, extensibility and robustness of the software is ensured.
New instruments and thus procedures can easily be added into the software without touching the project cases and everything will continue working as expected.
In comparison, if an approach where procedures have to be defined from within the project case was used, project cases would have to be updated every time we want to add new instruments to the application.