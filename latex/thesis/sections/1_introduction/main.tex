Through the well established field of medical imaging, surgeons can get a very detailed view of patient’s specific anatomy and pathology today. 
It is an essential part of preparing for surgery.
Most common medical 3D image acquisition techniques (not exlusive) are computed tomography (CT), cone-beam computed tomography (CBCT) and magnetic resonance imaging (MRI).
CT / CBCT makes use of x-ray measurements from different angles to produce cross-sectional (tomographic) "slices" of the scanned site.
With this technique, bone structure and soft tissue can be displayed in medical imaging.
The disadvantage of these techniques is the exposure to carcinogenic x-rays.
In MRI, strong magnetic fields, magnetic field gradients and ultrasound are used to create tomography of the patients tissue.
Since this technique makes use of hydrogen atoms, which is predominantly present in patient's soft tissue, bone structure is not imaged well.
However, when studying mandibular joints for example, MRI is able to outperform CT \cite{RN65}.
The most recent one, CBCT, lowers radiation dosage of traditional CT and continually contributes to the accuracy of diagnostic tasks of the viscerocranium.
It is able to produce images with isotropic submillimeter spatial resolution, which is ideally suited for isolated viscerocranium scans. 
The radiation dosage of CBCT is less than traditional CT and thus helps optimize health-to-risk ratio \cite{WHITE2008689}.

As discussed, there are a variety of ways to acquire medical imaging.
However, the displaying methods of 3D medical imaging data is very limited for clinicians.
After acquiring raw data via aformentioned modalities, volume images are generated. 
Generally, data is reconstruced in three planes (axial, sagittal and coronal).
Each plane is represented in "slices" which are 2D images of the volume image in an axis.
The resolution of aquired slices is described in the \textit{SpacingBetweenImages} and \textit{SliceThickness} attributes of the DICOM standard \cite{DICOMStandard}.

OMFS is very diverse. It has to handle a complex arrangement of bones, teeth, vessels, cartilage, nerves, muscles, skin and gland tissue.
These structures can be deeply complex, even more so in the viscerocranium.
Therefore, OMF surgeons rely heavily on accurate 3D medical imaging to plan procedures.
Even though three-dimensional objects are being analysed and also generated, they are viewed in a two-dimensional format on conventional computer screens.
The generated slices from medical imaging are viewed in the mentioned planes to get volumentric understanding of patient's underlying anatomy and pathology.
The problem with slices is that they are generally unsegmented and it is up to the viewer of the medical imaging to interpret them correctly.


Virtual reality (VR) at its core is a computer-generated, immersive, multi-sensory information program which tracks a user in real time \cite{burdea2003virtual}.
Its main compoenents are immersion, interaction, and imagination, which is broadly reffered to as \begin{equation}I^3\end{equation}.
Even though the term "Virtual Reality" was found in 1989 by Jaron Lanier, its first appearance goes back to as early as 1962.
Since then, VR has hit many milestones which were important for the development and success of the technology \cite{burdea2003virtual}.
The first VR head-mounted-device (HMD) was created by Ivan Sutherland in 1966.

The major applications of VR in surgery can be divided into three areas:
virtual humans for training, the fusion of virtual humans with real humans for performing surgery, 
and virtual telemedicine shared decision environments for training of multiple players \cite{486713}.
The first medical application of VR goes back to 1965, where R. Mann et. al. evaluated the use of VR as a mobility aid for the blind by creating a virtual environment in which to test hypothetical auditory display device  \cite{mann1965evaluation}.
In general, VR in medicine can be split into generic models and patient specific models.
Generic models are based on non-specific "average" data \cite{486713}.
\newline
Patient specific models on the other hand are modeled after individual patients.
Throughout the years, a number of such "virtual human" applications have been developed and had left positive impressions.

Data fusion is the term used to address the capability of a system to integrate data from different sources.
In the past decade, imaging techniques became very powerful and provided high resolution data.
However, the surgeon must still be able to run 2D images delivered by techniques such as MRI or CT in front of his mental eyes to build a 3D model of the structures.
Many of these images show the region of interest from different perspectives than the surgeon’s during the operation.
The surgeon not only has to create a mental 3D model but must also be able to fit the orientation of that model to the orientation of the structures observed during the operation.
Creation of 3D models using existing data from MRI or CT is easily done by the computer, allowing the surgeon to concentrate on more complex tasks \cite{486713}.

VR promises major improvements in this area.
Especially with the use of HMDs, surgeons can get a more natural and realistic feeling environment for education and training purposes.

For this thesis, three characteristic problems which surgeons today face have been identified.

First, to prepare for medical procedures, a number of preparational options are at the surgeons disposal.
Each of those common practices will be discussed based on how well it is suited for the individuality of patients, the realism and the cost (time and resources) associated with preparing for the operation.

A classical approach is the use of operational textbooks and videos.
They are not patient specific at all, but due to real operations being depicted the realism is generally low to medium.
The cost is also almost zero since the hospital will generally own a number of textbooks and videos for training.

Another common technique is the mental simulation in which the surgeon uses his imagination to get a mental image of the surgical site.
Surgeons draw out treatment plans in a papersimulation to complement the mental exercise.
This is solely dependant on the experience and spatial imagination of the surgeon. 
Hence it can be very patient specific, which is the goal in such an excercise.
This is usually little patient specific and not realistic, however the costs are very low too.
Since the costs are dependant on the expercience and skill of the surgeon himself, experienced surgeons however have an advantage here.

Another, yet costly technique is the phantom simulation.
Surgeons use a "phantom" of the patient to simulate the procedure.
Phantoms are real world representations of the patients anatomy and pathology, which can be e.g. 3D-printed.
This allows surgeons to more realisticly model the planned procedure and even try out certain procedures like drilling and milling, although the phantom will be corrupted in the process.
Since patient data is used here, it is an highly patient specific and highly realistic simulation.
However, recreating specific structures is prone to error, and once created, phantoms cannot be easy scaled or modified.
Another drawback is cost in form of the corresponding materials and the time to create the phantoms.

The last technique presented are planning software on traditional 2D monitors.
Several tools already exist, reaching from pre-operational planning tools to intra operation navigation and assitance \cite{HASSFELD20012}.
A drawback of such tools is that they are often complex software systems, which are in discrepandy with the requirements of the surgeon.
Hassfeld et. al. point out the need for user friendly and ergomic software solutions for surgeons.
Proposed advantages are precise and fast entry of the planned surgical procedure in theplanning and simulation phase \cite{HASSFELD20012}.

Each of theses techniques has major disadvantages.
They all cannot actively reflect the underlying specific anatomy and pathology of the patient and lack spatial perception.
Also, the 3D medical imaging, which gets viewed on 2D computer screens, has to be translated back onto the patient via the surgeons imaginative power. 
As discussed, there does not seem to be a trivial technique which all surgeons should use.
Surgeons will often use a combination of mentioned techniques to get the best results.
In the preparation stage, it is crucial that the operator gets a well defined mental image by 3D medical imaging data of the patient's anatomy and pathology.

Second, surgical training has to be done on patients.
This can not be fully mitigated.

Since patient procedures are non-reproducible, trainees have limited opportunities to train for specific procedures.
In VR, through saving states when training and being able to reproduce specific procedures, trainees have the ability to repeat specific procedures as often as needed.
\newline
Another disadvantage of non-reproducibility comes in the form of difficulty of measurability.
By saving planned procedures, trainees can redo and peers can review the procedures.
Saving has to be implemented with reproducibility in mind.
In each step of the procedure, how steps are to be performed has to be clear.

Lastly, the surgical intervention itself and the thereto related operating room time is expensive.
Often times, surgeries are immediate.
The surgeon finds something and must act immediately and quickly.
For theses reasons, the surgical environment is stressful while being high risk-reward based.