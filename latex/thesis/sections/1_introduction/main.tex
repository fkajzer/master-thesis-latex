Through the well established field of medical imaging, surgeons can get a very detailed view of patient’s specific anatomy and pathology today. 
It is an essential part of preparing for surgery \cite{Steinbacher.2015}.
Most commonly used medical imaging modalities (not exlusive) are computed tomography (CT), cone-beam computed tomography (CBCT) and magnetic resonance imaging (MRI).
CT makes use of x-ray measurements from different angles to produce cross-sectional (tomographic) "slices" of the scanned site \cite{Handels.2009}.
With this technique, bone structure and soft tissue can be displayed in medical imaging.
The disadvantage of these techniques is the exposure to carcinogenic x-rays.
\\
In MRI, strong magnetic fields and magnetic field gradients are used to create tomography of the patients tissue \cite{Handels.2009}.
Since this technique makes use of hydrogen atoms, which is predominantly present in patient's soft tissue, bone structure is not imaged well.
When studying mandibular joints for example, MRI is able to outperform CT \cite{RN65}.
\\
The most recent one, CBCT, lowers radiation dosage of traditional CT and continually contributes to the accuracy of diagnostic tasks of the viscerocranium \cite{Vos.2009}.
CBCT excels in imaging of bone tissue, while other tissues are not imaged well \cite{Vos.2009}.
It is able to produce images with isotropic submillimeter spatial resolution, which is ideally suited for isolated viscerocranium scans. 
The radiation dosage of CBCT is less than traditional CT and thus helps optimize health-to-risk ratio \cite{WHITE2008689}.

As discussed, there are a variety of ways to acquire medical imaging.
However, the displaying methods of 3D medical imaging data is very limited for surgeons.
After acquiring raw data via aformentioned modalities, volume images are generated in Digital Imaging and Communications in Medicine (DICOM) format \cite{DICOMStandard}.
DICOM is the standard format for the storage of volumetric image data.
Generally, data is reconstructed in three planes (axial, sagittal and coronal).
Each plane is represented in "slices" of varying thickness, which are represented as volumetric voxels.
The resulting voxels are then displayed as pixels on monitors.
The voxels resolution is described in the \textit{SpacingBetweenImages} and \textit{SliceThickness} attributes of the slices in the DICOM standard \cite{DICOMStandard}.

Oral and maxillofacial surgery (OMFS) is a very diverse surgical discipline.
It has to handle an arrangement of bones, teeth, vessels, cartilage, nerves, muscles, skin and gland tissue.
These structures can be deeply complex, even more so in the viscerocranium.
Therefore, OMF surgeons rely on accurate 3D medical imaging to plan procedures \cite{Fonseca.2018}.
Even though three-dimensional objects are being analysed and also generated, they are viewed in a two-dimensional format on conventional computer screens \cite{brewster1984interactive}.
The generated slices from medical imaging are viewed in the mentioned planes to get volumentric understanding of patient's underlying anatomy and pathology.
The challenge with raw DICOM images is that they are generally unsegmented and it is up to the viewer of the medical imaging to interpret them correctly \cite{Handels.2009}.


% Virtual reality (VR) at its core is a computer-generated, immersive, multi-sensory information technology which tracks a user in real time \cite{burdea2003virtual}.
% Its main compoenents are immersion, interaction, and intuition, which is broadly reffered to as $I^3$.
% Even though the term "Virtual Reality" was found in 1989 by Jaron Lanier, its first appearance goes back to as early as 1962 in form of the Sensorama, the first 
% VR device which simulated a motorcycle ride through New York\cite{heilig1962sensorama}.
% Since then, VR has hit many milestones which were important for the development and success of the technology \cite{burdea2003virtual}.
The first VR head-mounted display (HMD) was described in the scientific literature by Ivan Sutherland in 1968 \cite{sutherland1968head}.
Since then, they have become more and more relevant and increasingly popular in various fields of research.
\\
The first medical application of VR (without HMD) goes back to 1965, where R. Mann et al. evaluated the use of VR as a mobility aid for the blind 
by creating a virtual environment in which to test an hypothetical auditory display device \cite{mann1965evaluation}.
In general, VR in medicine can be split into generic models and patient specific models.
Generic models are based on non-specific "average" data \cite{486713}.
Patient specific models on the other hand are created for each individual patients.
Throughout the years, a number of such "virtual human" applications have been developed and had left positive impressions \cite{486713}.

Data fusion is the term used to address the capability of a system to integrate data from different sources.
In the past decade, imaging techniques became very powerful and provided high resolution data \cite{RN2}.
However, the surgeon must still be able to run 2D images delivered by techniques such as MRI or CT in front of his mental eyes to build a 3D model of the structures \cite{Fonseca.2018}.
Many of these images show the region of interest (ROI) from different perspectives than the surgeons will view it during the operation.
The surgeon not only has to create a mental 3D model, but must also be able to fit the orientation of that model onto the observed structures during the operation.
Creation of 3D models using existing data from MRI or CT is easily done by the computer, allowing the surgeon to concentrate on more complex tasks \cite{486713}.

VR promises major improvements in this area.
Through the stereoscopic vision of VR, surgeons can get a more natural and realistic feeling environment for education and training purposes.
This can be achieved either by room-mounted displays or HMDs, however, consumer HMDs are especially affordable.

For this thesis, a number of problems which surgeons today face have been identified.
These problems can be categorized into planning and training for procedures.
To prepare for medical procedures, a number of preparational options are at the surgeons disposal.
Each of those common practices will be described based on how well it is suited for the individuality of patients, the realism and the cost (time and resources) 
associated with preparing for the operation.
\\ A classical approach is the use of operational textbooks and videos.
They are not patient specific at all, but due to real operations being depicted, the realism is generally low to medium \cite{Ellis.2005}.
\\ Another common technique is the mental simulation in which the surgeon uses his imagination to get a mental image of the surgical site.
Surgeons draw out treatment plans in a paper simulation to complement the mental exercise.
This is solely dependant on the experience and spatial imagination of the surgeon. 
Hence, it can be very patient specific, which is the goal in such an excercise.
This is usually not realistic, however the costs are very low.
Since the costs are dependant on the experience and skill of the surgeon himself, experienced surgeons however have an advantage here.
\\ Another technique is the phantom simulation.
Surgeons use a "phantom" of the patient to simulate the procedure.
Phantoms are real world representations of the patients anatomy and pathology, which can be e.g. 3D-printed.
This allows surgeons to more realisticly model the planned procedure and even try out certain procedures like drilling and milling, although the phantom will be corrupted in the process.
Since patient data is used here, it is a highly patient specific and highly realistic simulation.
However, recreating specific structures is prone to error, and once created, phantoms cannot be easily scaled or modified \cite{TejoOtero.2020}.
They can be produced cost-effectively and timely though, since corresponding materials are commonly available.
One drawback is the non reproducibility of procedures \cite{richardson2015cost}.
\\ The last technique presented is planning software on traditional 2D monitors.
Several tools already exist, reaching from pre-operational planning tools to intra-operation navigation and assistance \cite{HASSFELD20012}.
A drawback of such tools is that they are often complex software systems, which are in discrepancy with the requirements of the surgeon.
Hassfeld et. al. point out the need for user friendly and ergonomic software solutions for surgeons.
Proposed advantages are precise and fast entry of the planned surgical procedure in the planning and simulation phase \cite{HASSFELD20012}.
\\ Each of the presented techniques has major disadvantages.
They all lack critical spatial perception, which is up to the surgeon's imagination.
Additionally, most techniques are not patient specific in regards to anatomy and pathology.
Also, the 3D medical imaging which gets viewed on 2D computer screens has to be translated back onto the patient via the surgeons imaginative power.
\\ As described, there does not seem to be a trivial technique which all surgeons should use.
Surgeons will often use a combination of the mentioned techniques to get the best results.
In the preparation stage, it is crucial that the operator gets a well defined mental image by 3D medical imaging data of the patient's anatomy and pathology.

Surgical training has to be done on patients in the operating room.
An analysis of the scientific literature shows that this can not be fully mitigated \cite{mcgaghie2011does}.
\\ Since patient procedures are non-reproducible, trainees have limited opportunities to train for specific procedures.
In VR, through saving states when training and being able to reproduce specific procedures, trainees have the ability to repeat specific procedures as often as needed.
\\ Another disadvantage of non-reproducibility comes in the form of difficulty of measurability.
By saving planned procedures, trainees can redo and peers can review the procedures.
Saving has to be implemented with reproducibility in mind.
In each step of the procedure, how steps are to be performed has to be clear.

Lastly, surgical intervention itself and the thereto related operating room time is expensive \cite{Barber.2020}.
Often times, surgeries are immediate.
The surgeon finds previously unknown circumstances in the patient and must act immediately and quickly.
For theses reasons, the surgical environment is stressful especially for inexperienced surgeons \cite{schuetz2008three}.

Even though some techniques already have been established, operations are still planned on two-dimensional images for three-dimensional surgical sites \cite{Steinbacher.2015}.
As HMDs become cheaper, they are more widely used and more often applied for surgical purposes.
This thesis aims to improve medical imaging and pre-operational planning by using a modern approach with HMDs.
By using acquired 3D medical imaging in a VR application, a very patient specific, highly realistic and only moderately costly technique with which surgeons can prepare and 
train for operations is pursued.
With the emergence of new HMD such as the Oculus Rift S in 2019, HMDs become more and more affordable.
Based on this trend, this thesis will be developed for consumer HMDs and will be compatible with all currently available HMDs to utilize the immersive power of stereoscopic imaging.
Additionally, open-source software will be utilized to make the developed software as easily accessible as possible. 
\\ Although having major utility for the surgical application, haptic feedback devices and tissue simulation, as will be described in Section \ref{chap::RelatedWork}, are specific to the procedure and usually costly.
Additionally, some users have difficulties adjusting to the input devices, since they have somewhat of a learning curve, especially for non-technically experienced users.
Therefore, in this thesis, realistic physical behaviour of tissue is not considered.
The focus will be on providing visualization, planning and training tools for surgeons.
\\ The developed software will be evaluated by surgical trainees for the preparation of a release version of the developed software.
With the help of VR, this thesis is aiming to investigate, whether stress and action times of surgeons can be improved by planning and training in a safe and risk-free environment.

The following advantages of VR surgery simulations over conventional methods for planning and training purposes are being considered:
\begin{compactenum}[label=(\alph*)]
    \item Familiarize the operator with the patient specific anatomy and pathology before operating
    \item The ability to simulate important operation steps
    \item Allow revision of the virtual operation as often as needed
    \item Recording and analysis of users and others virtual operations
    \item Test out procedures
\end{compactenum}

In this thesis, however, the focus is not on realistic simulation of a surgical procedure and tissue involved.
Here, abstract simulation of surgical procedures are planned.
The aim of this thesis is to familiarize surgeons with the patient specific anatomy and pathology.
By providing an immersive VR environment, surgeons should feel present in the surgical environment.
It will be investigated, whether this can reduce pre-operative scress for trainees.
Surgical procedures will be abstracted in such a way, that a huge number of procedures can be simulated.
This way, this thesis could provide a platform for any kind of surgical procedure in the future. 
Surgeons could both prepare for patient specific surgical procedures as well as train the process of specific surgical procedures on generic patient models.  
However, to get the correct motor skills necessary for procedures, training has to be done on live patients, as mentioned before, or even through sophisticated, procedure specific surgical 
simulations with haptic feedback.

Through providing the mentioned aspects of a VR simulation, the hope is to reduce the rate on which training has to be done live on patients through familiarizing trainees with 
surgical procedures and processes involved beforehand.
Additionally, a sophisticated visualization, planning and training tool is aspired.
\\ In a VR environment, procedures can be planned, edited and saved as users see fit.
Through VR's stereoscopic nature, spatial relationships in the patients anatomy and pathology can be more accurately represented.
Additionally, procedures can be planned collaboratively and improved incrementally, although this advantage is not exclusive to VR.

For OMF surgeons, especially the imaging of volumetric objects is mentally demanding. 
By the use of VR technology, the area of visualization of medical imaging has already been proven to be advanced, as will be shown in Section \ref{chap::RelatedWork}.
This thesis aims to aid trainees by providing realistic 3D medical imaging in VR.
By playing through the somewhat abstract procedures which can be planned beforehand, trainees have the opportunity to remember the correct processes of specific procedures.
They can visualize themselves, the patient and surgical instruments inside of a virtual operating theatre (OT) and go through procedures step by step as often as needed.

The remainder of this thesis is structured as follows. 
In Section \ref{chap::RelatedWork} related research in the field of VR applied to surgical workflows and existing surgical simulations will be presented.
Used hard- and software, as well as input devices (with and without haptic feedback) will be presented.
A number of surgical simulations which utilized HMDs will be described in detail.
Different visualization techniques and levels of presence will also be presented via the related work.
In Section \ref{chap::Approach}, the developed requirements from the aforementioned related works, which are used as a guideline to design the system, will be described 
together with a broad concept of the software.
The specific implementation details will then be extensively described in Section \ref{chap::Implementation}.
Section \ref{chap::Evaluation} covers the methology behind the conducted evaluation with the developed software, as well as presenting the results.
Lastly, a thorough discussion followed by a conclusion with prospects of future work will finish this thesis.