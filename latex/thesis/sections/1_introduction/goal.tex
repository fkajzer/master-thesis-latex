Even though some techniques already have been established, operations are still planned on two-dimensional images for three-dimensional surgical sites \cite{Steinbacher.2015}.
As HMDs become cheaper, they are more widely used and more often applied for surgical purposes.
This thesis aims to improve medical imaging and pre-operational planning by using a modern approach with HMDs.
By using acquired 3D medical imaging in a VR application, a very patient specific, highly realistic and only moderately costly technique with which surgeons can prepare and 
train for operations is pursued.
With the emergence of new HMD such as the Oculus Rift S in 2019, HMDs become more and more affordable.
Based on this trend, this thesis will be developed for consumer HMDs and will be compatible with all currently available HMDs to utilize the immersive power of stereoscopic imaging.
Additionally, open-source software will be utilized to make the developed software as easily accessible as possible. 
\\ Although having major utility for the surgical application, haptic feedback devices and tissue simulation, as will be described in Section \ref{chap::RelatedWork}, are specific to the procedure and usually costly.
Additionally, some users have difficulties adjusting to the input devices, since they have somewhat of a learning curve, especially for non-technically experienced users.
Therefore, in this thesis, realistic physical behaviour of tissue is not considered.
The focus will be on providing visualization, planning and training tools for surgeons.
\\ The developed software will be evaluated by surgical trainees for the preparation of a release version of the developed software.
With the help of VR, this thesis is aiming to investigate, whether stress and action times of surgeons can be improved by planning and training in a safe and risk-free environment.

The following advantages of VR surgery simulations over conventional methods for planning and training purposes are being considered:
\begin{compactenum}[label=(\alph*)]
    \item Familiarize the operator with the patient specific anatomy and pathology before operating
    \item The ability to simulate important operation steps
    \item Allow revision of the virtual operation as often as needed
    \item Recording and analysis of users and others virtual operations
    \item Test out procedures
\end{compactenum}

In this thesis, however, the focus is not on realistic simulation of a surgical procedure and tissue involved.
Here, abstract simulation of surgical procedures are planned.
The aim of this thesis is to familiarize surgeons with the patient specific anatomy and pathology.
By providing an immersive VR environment, surgeons should feel present in the surgical environment.
It will be investigated, whether this can reduce pre-operative scress for trainees.
Surgical procedures will be abstracted in such a way, that a huge number of procedures can be simulated.
This way, this thesis could provide a platform for any kind of surgical procedure in the future. 
Surgeons could both prepare for patient specific surgical procedures as well as train the process of specific surgical procedures on generic patient models.  
However, to get the correct motor skills necessary for procedures, training has to be done on live patients, as mentioned before, or even through sophisticated, procedure specific surgical 
simulations with haptic feedback.

Through providing the mentioned aspects of a VR simulation, the hope is to reduce the rate on which training has to be done live on patients through familiarizing trainees with 
surgical procedures and processes involved beforehand.
Additionally, a sophisticated visualization, planning and training tool is aspired.
\\ In a VR environment, procedures can be planned, edited and saved as users see fit.
Through VR's stereoscopic nature, spatial relationships in the patients anatomy and pathology can be more accurately represented.
Additionally, procedures can be planned collaboratively and improved incrementally, although this advantage is not exclusive to VR.

For OMF surgeons, especially the imaging of volumetric objects is mentally demanding. 
By the use of VR technology, the area of visualization of medical imaging has already been proven to be advanced, as will be shown in Section \ref{chap::RelatedWork}.
This thesis aims to aid trainees by providing realistic 3D medical imaging in VR.
By playing through the somewhat abstract procedures which can be planned beforehand, trainees have the opportunity to remember the correct processes of specific procedures.
They can visualize themselves, the patient and surgical instruments inside of a virtual operating theatre (OT) and go through procedures step by step as often as needed.