
This thesis aims to improve medical imaging and pre-operational planning by using a modern approach with head-mounted displays.
By using acquired 3D medical imaging in a virtual reality application, a very patient specific, highly realistic and only moderately costly technique with which surgeons can prepare for operations is pursued.
\newline

The following advantages over conventional methods are being considered:
\begin{compactenum}[label=(\alph*)]
    \item Familiarize the operator with the patient specific anatomy and pathalogy before operating
    \item The ability to simulate important operation steps
    \item Allow revision of the virtual operation as often as needed
    \item Recording and analysis of users and others virtual operations
    \item Test out procedures
\end{compactenum}

Thorugh implementation of these logical steps in preparing for surgery, this thesis aims to improve the rate on which training has to be done 'live' on patients.
By providing the VR environment to visualize and train on virtual patients, the hope is to improve the time which has to be spent on training on real patients in the surgical environment.
\newline
A key part of this thesis is to advance the area of visualisation of medical imaging.
Even though techniques exist, operations are still planned on two-dimensional images for three-dimensional surgical sites.
With the emergence of new HMD such as the Oculus Rift S last year, HMDs become more and more affordable.
Based on this trend, this thesis will be developed with consumer VR HMDs and will be compatible with currently available HMD to utilize the immersive power of stereoscopic imaging.
Input device for interactions in VR will be the Valve Index Controller \cite{ValveIndex}, however all consumer input devices will be compatible with minimal adjustments to the software.
As development environment, Unity3D will be utilized, as it is the predominantly used in the in Section \ref{sec::RelatedWork} described works.

For OMF surgeons, especially the imaging of voluminous objects is mentally demanding and this thesis hopes to eliminate this problem completely by providing realistic 3D medical imaging in virtual reality.
This thesis is part of an applied virtual and augmented reality workflow for oral and maxillofacial surgery using head mounted displays (HMDs) as described in Figure \ref{fig::ProjectPlan}.

The main goal of this thesis is to create a pre-operative assistance tool in VR with HMDs for oral and maxillofacial surgery.
This means that while the main goal is surgery simulation, the focus is not on realistic physical behaviour of tissue.
In the context of the research project by the department of oral and maxillofacial surgery in the university hospital of the RWTH Aachen and the VR Group of RWTH Aachen University, the results of the pre-operative planning might be used intra-operatively to provide assistance via Augmented Reality (AR) as described in the workflow (Figure \ref{fig::ProjectPlan}).
To provide a useful preparational tool, it is critical to simulate individual operation steps.
To achieve this, visualization and planning tools for OMFS with HMDs will be provided.
In an immersive virtual operating theatre, users will be able to plan and train surgical procedures.
Planned steps will be storable in a format in which they can be loaded and viewed in both the virtual and augmented reality applications bi-directionally as described (Figure \ref{fig::ProjectPlan}).
By planning the operational steps in virtual reality, planned procedures can easily be shown to other staff involved.
Naturally, it will be of uttermost importance that we have medical equipment and an appropriate virtual environment recreated remarkably close to reality.
With the help of VR, this thesis is aiming to investigate whether stress and action times of surgeons can be improved by planning and training in a safe and risk-free environment.