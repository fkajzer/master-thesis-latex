For this thesis, a number of problems which surgeons today face have been identified.
These problems can be categorized into planning and training for procedures.
\\ To prepare for medical procedures, a number of preparational options are at the surgeons disposal.
Each of those common practices will be described based on how well it is suited for the individuality of patients, the realism and the cost (time and resources) associated with preparing for the operation.
\\ A classical approach is the use of operational textbooks and videos.
They are not patient specific at all, but due to real operations being depicted the realism is generally low to medium \cite{Ellis.2005}.
\\ Another common technique is the mental simulation in which the surgeon uses his imagination to get a mental image of the surgical site.
Surgeons draw out treatment plans in a papersimulation to complement the mental exercise.
This is solely dependant on the experience and spatial imagination of the surgeon. 
Hence it can be very patient specific, which is the goal in such an excercise.
This is usually little patient specific and not realistic, however the costs are very low too.
Since the costs are dependant on the expercience and skill of the surgeon himself, experienced surgeons however have an advantage here.
\\ Another technique is the phantom simulation.
Surgeons use a "phantom" of the patient to simulate the procedure.
Phantoms are real world representations of the patients anatomy and pathology, which can be e.g. 3D-printed.
This allows surgeons to more realisticly model the planned procedure and even try out certain procedures like drilling and milling, although the phantom will be corrupted in the process.
Since patient data is used here, it is an highly patient specific and highly realistic simulation.
However, recreating specific structures is prone to error, and once created, phantoms cannot be easily scaled or modified \cite{TejoOtero.2020}.
\\ They can be produced cost-effectively and timely though, since corresponding materials are commonly available.
One drawback is the non reproducible of procedures \cite{richardson2015cost}.
\\ The last technique presented are planning software on traditional 2D monitors.
Several tools already exist, reaching from pre-operational planning tools to intra operation navigation and assitance \cite{HASSFELD20012}.
A drawback of such tools is that they are often complex software systems, which are in discrepandy with the requirements of the surgeon.
Hassfeld et. al. point out the need for user friendly and ergomic software solutions for surgeons.
Proposed advantages are precise and fast entry of the planned surgical procedure in the planning and simulation phase \cite{HASSFELD20012}.
\\ Each of theses techniques has major disadvantages.
They all lack critical spatial perception, which is up to the surgeons imagination.
Additionally, most techniques are not patient specific in regards to anatomy and pathology.
Also, the 3D medical imaging, which gets viewed on 2D computer screens, has to be translated back onto the patient via the surgeons imaginative power.
Often, complex software systems are used for viewing patient data, which are in discrepandy with the requirements of the surgeon.
Hassfeld et. al. point out the need for user friendly and ergomic software solutions for surgeons.
Proposed advantages are precise and fast entry of the planned surgical procedure in the planning and simulation phase \cite{HASSFELD20012}.
\\ As described, there does not seem to be a trivial technique which all surgeons should use.
Surgeons will often use a combination of mentioned techniques to get the best results.
In the preparation stage, it is crucial that the operator gets a well defined mental image by 3D medical imaging data of the patient's anatomy and pathology.

Surgical training has to be done on patients in the operating room.
An analysis of the scientific literature shows that this can not be fully mitigated \cite{mcgaghie2011does}.
\\ Since patient procedures are non-reproducible, trainees have limited opportunities to train for specific procedures.
In VR, through saving states when training and being able to reproduce specific procedures, trainees have the ability to repeat specific procedures as often as needed.
\\ Another disadvantage of non-reproducibility comes in the form of difficulty of measurability.
By saving planned procedures, trainees can redo and peers can review the procedures.
Saving has to be implemented with reproducibility in mind.
In each step of the procedure, how steps are to be performed has to be clear.

Lastly, surgical intervention itself and the thereto related operating room time is expensive \cite{Barber.2020}.
Often times, surgeries are immediate.
The surgeon finds something and must act immediately and quickly.
For theses reasons, the surgical environment is stressful especially for inexperienced surgeons \cite{schuetz2008three}.