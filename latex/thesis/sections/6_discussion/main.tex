The results of the study show that the potential advantages of VR 
presented in Section \ref{chap::Introduction} 
and investigated via related works in Section \ref{chap::RelatedWork}
could be successfully implemented in the context of this thesis. 
\\ Although the abstract simulation of 
procedures was lacking, they were nevertheless perceived as being 
useful tools for planning and training in OMFS.
Especially the visualization of patient-specific 
anatomy and pathology was a successful aspect of this thesis.
However, users performing the medical imaging acquisition have to make expert 
decisions about which techniques should be used in order to highlight the 
patient specific tissue, which cannot be influenced inside of the application.
The high standard deviation of the SUS score could be attributed
to the subjective experience of the participants of the study and difference in familiarity with the technology,
whilst performing the study tasks. It was observed, 
that for some, the interactions and usage of system components 
came naturally and others struggled to navigate through VR.
It seems, that technology affine people in general had a better
experience while using the system. 
It was suprising, that one of the participants felt the interaction with
the GUI to be strange. Although haptic feedback in the form of 
the Valve Index Controllers vibrating in the hand of participant 
signal that the GUI was touched, the participant mentioned that
he did not feel any feedback when interacting with the GUI.
Additionally, the virtual hands cannot pass through the GUI, which
means that the GUI acts as a virtual wall. For the majority of
participants, it seems that this kind of signaling was sufficient, 
however, it seems that there is room for improvement for the
interaction with the GUI here.
\\ Another important aspect, which was aspired to be achieved through
the design of the virtual OT, was user presence. 
This was a huge success, since participants rated their presence and 
immersion as well as the layout of the virtual OT as overwhelmingly positive. 
Also, the robustness of the system has been satisfactory, as
there were no system failures and the study proceeded smoothly.
\\ Unfortunately, voice feedback and control were only sufficient, 
although they were an essential part of the navigation of 
project cases. There is definitely room for improvement here, 
especially since they play an important role in the context of the workflow.
\\ For this thesis, the abstract simulation of procedures is considered
to be a success. In the context of visualizing and memorizing the individual steps necessary
for specific procedures, the usefulness of these abstract simulations, even
though the majority of study participants criticized that they 
were not realistic, could be shown. This means, that for some 
aspects of pre-operative planning and training, abstract 
simulations could be enough. Especially in the case of visualization and 
learning processes of specific procedures, they should be considered.
\\ The realistic simulation of procedures, as mentioned by one of 
the participants of the user study, in the form of cutting and moving
skin in a realistic way should definitely be aspired to. This 
was the major point of criticism for the application and cannot
be ignored. However, in the scope of this thesis, and the VR-AR-based
workflow, this was simply not practicable. For now, the focus was to
prove that a pre-operative planning and intraoperative guiding workflow, combining
AR glasses and HMDs was feasable. The evaluation of the study shows,
that the VR part of the workflow can definitely be considered a success, 
even though, there are some points of criticism.