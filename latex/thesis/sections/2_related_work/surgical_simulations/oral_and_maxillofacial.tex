VR Surgery allows trainees to virtually participate in a surgical procedure and interact with the patient's anatomy \cite{RN6}.
There is a huge emphasis to depict real surgical procedures and all related circumstances, such as a crowded operating table, as close as possible.
The idea for VR Surgery came out of the global need for safe and affordable surgery.
It is predicted that the global demand for surgeons can not be met with the current methods of surgical education \cite{RN68}.

Pulijala et. al describe how technical and non-technical skills have to be acquired in surgical training.
Traditional means of surgical education though hands-on-practice has been around for more than a century.
It was found that four out of ten novice surgeons are not confident in performing major procedures.
VR Surgery aims to provide cognitive training for oral and maxillofacial surgeons \cite{RN68}.

After analysing existing methods of surgical training and identifying the need for their transformation,
Pulijala et. all concluded that a simulation based learning tool can be benefitial for trainees.
In VR Surgery, surgical trainees view pre-recorded stereoscopic 3D videos of real surgery in an OT and are able to interact with patient's anatomy via Oculus Rift and Leap Motion.
The surgical procedure depicted in the videos is Le Fort I surgery.
The software was designed in Unity 3D (TODO QUELLE).

Content of VR Surgery is split into four steps:
\begin{compactenum}[label=(\alph*)]
    \item Preoperative preparation
    \item Soft tissue incision and exposure
    \item Bone cuts, disimpaction and mobilisation
    \item Bone fixation and suturing
\end{compactenum}

The 3D videos provide depth perception and realistic view of the surgical procedure, with other surgeons performing the procedure.
This gives trainees the impression to be present during real procedures. It was found that such realistic scenarios improved learning.
Users can zoom in and touch the 3D model of the patient's anatomy to visualise spatial relationshipts between the anatomy.
Trainees get feedback through questions and tasks about procedures.

Pulijala et. al conclude how surgical education needs a major reform to meet future needs.
Through continuously advancing commercially available hardware, VR became an affordable platform for high quality surgical simulations.
They predict that applications like VR Surgery will provide an additional way of learning and in the best case reduce training time of surgical novices.

VR Surgery highlights how important realism is for training simulations.
This is precisely why this thesis proposes a novel approach to existing methods.
Since commercially available hardware today is able to produce high fidelity visuals and an immersive, stereoscopic view of virtual objects,
the natural conclusion is to try out innovative approaches to existing problems with them.
It also highlights how important a 1:1 real world scale is for visualizing the spatial relationships in patient's anatomy and pathology.



