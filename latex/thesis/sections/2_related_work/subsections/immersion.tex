\textbf{Immersion}: In the related works, the level of immersion which is aimed to achieve varies a lot.
Some do not pay any attention to immerse users into the VR at all, while others make great efforts to completely immerse users into the VR.
\newline
In Pulijala et als (2017) "VR Surgery", 3D videos provide depth perception and realistic view of the surgical procedure, with other surgeons performing the procedure.
This gives trainees the impression to be present during real procedures.
It was found that such realistic scenarios improved learning.
Users can zoom in and touch the 3D model of the patient's anatomy to visualise spatial relationships between the anatomy.
Trainees get feedback through questions and tasks about procedures.
\newline
Parham et al (2019) present an immersive surgical simulator in which the surgical environment is modeled as closely to reality as possible \cite{Parham.2019}.
The authors focused on high-quality visuals for immersion.
The virtual OT consists of the open surgical area including organs of the patient, a tray for surgical instruments and a monitor displaying simulated patient vitals and procedure instructions.
It was modeled after a real world OT located inside the University Teaching Hospital in Lusaka, Zambia.
The assets were modeled after recieving reference photos and videos of locations and instruments and researching the female anatomy.
It was argued that even though 3D scans of real human organs exists, they are too inefficient to run in real time VR \cite{RN52}.
3D models of the human female pelvic anatomy and pathalogy including organs, veins, peritoneum and connective tissue were created.
The accuracy of the simulation in regard to real life counterparts was the main focus of this works.
Immersing trainees in the simulator was crucial so that they can focus on learning and practise without distractions.
Putting them in a realistic operating scenario helps reducing anxiety and selfconsciousness before first time operations \cite{Parham.2019}.
\newline
Barber et al (2020) propose a fusion of 3D-intraoperative video with immersive VR as a teaching tool for lateral skull base surgery.
Interactive anatomic models are utilized in an virtual OT while watching 360-degree video as a learning tool.
This VR simluation situates users in the lead surgeon's chair while simultaneously being able to pause, stop and rewind the procedure.
Additionally, users have the 3D interactive anatomic model in their hands to utilize stereoscopic visualization of the surgical site \cite{Barber.2020}.
\newline
Usage of immersive video is also utilized broadly.
While some do not utilize video, others used filmed surgical procedures as a method to prepare trainees for real surgery and reduce stress in the real operation environment.
\newline
Parham et al (2019) used an approach in which filmed procedures in 360 degree are utilized together with an immersive virtual operating theatre.
Videos of surgical procedures were captured during real surgical procedures via six specially arranged GoPro 360-degree cameras and were used to caputure the OT \cite{Parham.2019}.
Additionally, close-ups of the surgical procedures were captured with stereoscopic Sony 3D cameras.
\newline
Captured videos were then played inside of the VR OT during the training of the procedure.
The training consisted of online didactic lectures, readings and seven modules, with embedded videos that correspond to the surgical procedure.
These could be viewed from withing the virtual OT, so that trainees would watch procedures and then simulate them via the tools present in the OT \cite{Parham.2019}.