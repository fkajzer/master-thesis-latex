\textbf{Devices}: Even though a wide range of devices are summarized as VR HMDs, there are just a few immersive HMDs, which allow the visualization of 3d models and interaction with their environment and the user thorugh technologies such as hand tracking or controllers.
\newline
Two particularly widespread devices are Oculus Rift \cite{OculusRift} and HTC Vive \cite{Vive}.
They are the first consumer HMDs widely known for their interaction and immersion compatibilities, bringing immersive VR to the consumer market around 2016.
For surgical purposes, HMDs like the Oculus Rift, including the developement kits DK 1 and 2 \cite{Parham.2019, Pulijala.2017,Sampogna.2017}, HTC Vive \cite{.2017, Barber.2020}, but also more recent devices such as Oculus Rift S (Quellen) were used.
\newline
One reason Oculus Rift head-mounted display was selected was its availability, cost and efficiency at the time of research \cite{Pulijala.2017}.
\newline
In the sense of augmented virtuality \cite{milgram1994taxonomy}, there exist different technologies for representation of the users hands in VR.
One such technologie is Leap Motion, which captures the users hands through the use of two cameras \cite{LeapMotion}.
These were applied in the surgical field to immerse users in the simulation \cite{Pulijala.2017, Sampogna.2017}
\newline
However, the use of consumer HMDs together with more specialized, non consumer haptic feedback devices such as the Geomagic Touch were also described in studies \cite{VenkataS.Arikatla.2018}.
Here, haptic feedback devices are used to improve the realism of surgical simulations. 
Advantages of such devices are a more realistic representation of the surgical procedure, while effectively having a higher barrier of entry.