In the scientifc literature about VR surgery applications, the level of presence which is aimed to achieve varies a lot.
Some do not pay any attention to immerse users into the VR at all, i.e. they simply show a 3D model of the patients anatomy and pathalogy, while others make great efforts to completely immerse users into the VR \cite{Vaughan.2016}.
In some cases, VR is used solely as a novel visualization tool with novel control schemes \cite{.2017}.
However, other works make a great effort to give the user a level of presence, thorugh modeling a virtual OT after real life locations as close as possible.
It was shown that a higher level of presence can reduce stress and anxiety from real operations through VR training \cite{Pulijala.2017,Pulijala.2018,Pulijala.2018b}.
\\ In general, one of two approaches is utilized.
The first kind of approach is already depicted in Section \ref{sec::VRBasedApplications}.
Here applications make a great effort to simulate all aspects of the surgical environment, including patient, procedures with their respective instruments, and the virtual OT.
\\ This way, trainees feel present in the OT and focus on performing a surgical procedure.
In the other approach, which will be presented in the following, only some aspects of the surgical setting are present in the virtual OT.
360 degree video is used complimentary to VR to give stereoscopic instructional videos of certain procedures, while trainees have the option to pause video and visualize the patients anatomy via a 3D model of the patient.
\\Pulijala et al. present a novel visualization and training tool for trainees as well as senior surgeons.
An immersive learning experience for surgical trainees through pre-recorded stereoscopic 3D videos of surgery and interactive models of patient’s anatomy was created \cite{Pulijala.2017}.
The depicted surgical procedure was "Le Fort 1" surgery, a type of maxillofacial surgery, performed to correct lower midface deformities \cite{Pulijala.2017}
\\ 3D videos are used to provide depth perception and realistic view of the surgical procedure, with other surgeons performing the procedure.
This gives trainees the impression to be present during real procedures.
It was found that such realistic scenarios improved learning.
Users can zoom in and touch the 3D model of the patient's anatomy to visualise spatial relationships between the anatomy.
Trainees get feedback through questions and tasks about procedures \cite{Pulijala.2017,Pulijala.2018}.

Parham et al. used an approach in which filmed procedures in 360 degree are utilized together with an immersive virtual OT.
The virtual surgical environment is modeled as closely to reality as possible \cite{Parham.2019}.
The authors focused on high-quality visuals for immersion.
\\ Videos of surgical procedures were captured during real surgical procedures via six specially arranged GoPro 360-degree cameras and were used to caputure the OT.
Additionally, close-ups of the surgical procedures were captured with stereoscopic Sony 3D cameras \cite{Parham.2019}.
\\ Captured videos were then played inside of the virtual OT during the training of the procedure.
The training consisted of online didactic lectures, readings and seven modules, with embedded videos that correspond to the surgical procedure.
These could be viewed from withing the virtual OT, so that trainees would watch procedures and then simulate them via the tools present in the OT \cite{Parham.2019}.
\\ The virtual OT consists of the open surgical area including organs of the patient, a tray for surgical instruments and a monitor displaying simulated patient vitals and procedure instructions.
It was modeled after a real world OT located inside the University Teaching Hospital in Lusaka, Zambia.
The assets were modeled after recieving reference photos and videos of locations and instruments and researching the female anatomy.
It was argued that even though 3D scans of real human organs exists, they are too inefficient to run in real time VR \cite{RN52}.
3D models of the human female pelvic anatomy and pathalogy including organs, veins, peritoneum and connective tissue were created.
The accuracy of the simulation in regard to real life counterparts was the main focus of the aformentioned study.
Immersing trainees in the simulator was crucial so that they can focus on learning and practise without distractions.
Putting them in a realistic operating scenario helps reducing anxiety and selfconsciousness before first time operations \cite{Parham.2019}.

Barber et al. propose a fusion of 3D-intraoperative video with immersive VR as a teaching tool for lateral skull base surgery.
Interactive anatomic models are utilized in an virtual OT while watching 360-degree video as a learning tool.
This VR simluation situates users in the lead surgeon's chair while simultaneously being able to pause, stop and rewind the procedure.
Additionally, users have the 3D interactive anatomic model in their hands to utilize stereoscopic visualization of the surgical site \cite{Barber.2020}.