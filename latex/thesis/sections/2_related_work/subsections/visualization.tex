Advancements in computers and imaging, especially over the last 10 years, have permitted the adoption of 3-dimensional imaging protocols in the health care field.
\newline
Virtual patients can be created by patient-specific anatomic reconstruction (PSAR), which can then be studied and used to develop and simulate treatment protocols.
Schendel et. al. propose image fusion, which involves combining images from different imaging modalities tocreate a virtual record of an individual called a patient-specific anatomic reconstruction \cite{schendel2009three}.
By using the combination of different image acquisition techniques, a treatment was planned for a young woman with mandibular retrusion.
It was shown that using generated 3D models of the patients specific anatomy and pathalogy brought huge improvements in planning treatments over traditional methods \cite{schendel2009three}.

In another related work, Computet Tomography (CT) and Magnetic Resonance (MR) were compared on the diagnostic potential in the aassessment of Synovia Chondromatosis (SR) of the Temporo-Mandibular Joint (TMJ).
Testaverde et al compared softtissueinvolvement (disk included), osteostructural alterations of the joints, loose bodies and intra-articular fluid of eight patients with symptoms of SC \cite{testaverde2011ct}.         
Results showed that CT scan is excellent to define bony surfaces of the articular joints and flogistic tissue but it fails in the detection of loose bodies when these are not yet calcified.
Optimally, a PSAR approach as proposed by Schendel et al \cite{schendel2009three} should be used.