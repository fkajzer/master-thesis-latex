\textbf{Visualization}: Segmentation of patient specific DICOM data is predominantly done by using 3D Slicer \cite{Barber.2018,Barber.2020}.
Postprocessing of generated data can be done by utilizing tools such as Blender3D or Autodesk Maya \cite{Barber.2020,Parham.2019,Sampogna.2017}.
\newline
Sampogna et al (2017) achieve patient specific 3D models by acquiring medical imaging of different modalities, such as CT and MRT \cite{Sampogna.2017}.
To better visualize patient specific anatomy and pathalogy, the acquired DICOM data was segmented in 3D Slicer and postprocessed in Blender.
Additionally, generated 3D models were printed, so that a virtual and real life 3D model exist.
The finalized 3D models were evaluated by a radiologist.
In VR, the users used Leap Motion for VR interaction to prepare for sugery.
With the use of hand and finger gestures, users could rotate the target anatomy and change the point of view.