\textbf{Interactions}: Kim et al (2017) present an interactive environment, in which users can grab and cut a craniopagic patients model \cite{.2017}.
Manipulation of the presented model is user friendly, as each of the two implemented interaction is accesed via the use of a single button.
\newline
Parham et al (2019) created a VR simulation, in which traineee stand in the virtual OT with an operating table, tray of surgical instruments and the surgery awaiting patient with cervical cancer.
The procedure is closely modeled after an actual surgical procedure, meaning the surgical site is exposed while the rest of the patient is covered.
Instructions are given via a monitor above the operating table and audio feedback is given to guide the trainee through the simulation.
The simulation lasts roughly 20 minutes and provides feedback on the trainees accuracy on various postions of the procedure and an overall score.
To compare traditional surgical training versus VR, the trainees are assessed by expert surgeon-mentors \cite{RN52}.

Pulijala et al \cite{Pulijala.2017} presented "VR Surgery", which allows trainees to virtually participate in a surgical procedure and interact with the patient's anatomy \cite{RN6}.
There is a huge emphasis to depict real surgical procedures and all related circumstances, such as a crowded operating table, as close as possible.
The users could interact with the annotated 3D models of the skull by zoom in and out and touch the 3D model.
Different structures could also be isolated to better inspect anatomic relationships \cite{Pulijala.2017}.
