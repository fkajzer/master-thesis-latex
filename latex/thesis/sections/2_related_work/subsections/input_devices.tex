In the sense of Augmented Virtuality (AV) \cite{milgram1994taxonomy}, there exist different technologies for representation of the users hands in VR.
One such technology is Leap Motion, which captures the users hands through the use of two cameras \cite{LeapMotion}.
These were applied in the surgical field to immerse users in the simulation \cite{VenkataS.Arikatla.2018,Sampogna.2017}.
In VR, Leap Motion is used as an interaction tool to prepare for surgery.
The user's hands are tracked in real time and a representation of the user's hands is projected into VR.
With the use of hand and finger gestures, users could rotate the target anatomy and change the point of view \cite{Sampogna.2017}.
\\ In another approach, Kim et al. \cite{kim2017virtual} present an interactive environment, in which users can grab and cut a craniopagic patient's model.
Using a representation of the input device, in this case a standard first generation input device named Vive Controller, the user gets a sense of where 
his hands are in the virtual world, although they are not represented as virtual hands. 
Manipulation of the presented model is user friendly, as each of the two implemented interaction is accessed via the use of a single button, namely the 
trigger button for cutting and grip button for grabbing the patient's model \cite{.2017}.
\\ However, the use of consumer HMDs together with more specialized, non-consumer haptic feedback devices, such as the Geomagic Touch were also described in studies \cite{VenkataS.Arikatla.2018}.
Here, haptic feedback devices are used to improve the realism of surgical simulations.
Advantages of such devices are more realistic representation of the surgical procedure, while effectively having a higher barrier of entry.