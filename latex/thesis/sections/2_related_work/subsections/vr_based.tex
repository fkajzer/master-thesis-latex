Cervical cancer is one of the world’s leading causes of death \cite{Parham.2019}.
Parham et. al present a novel approach for a low cost surgical simulation to train novice
surgeons for cervical cancer surgery \cite{Parham.2019}. The simulator aims to create pre-trained novices
in resource-challenged countries. 
\\ By developing this low cost simulation using commercially available VR software and the
Oculus Rift HMD, they have successfully helped surgeons to prepare for radical abdominal
hysterectomy surgery procedures. A near identical representation of an operating theatre (OT)
using 1:1 scale was used for immersion. 3D replica of human female pelvic anatomy and
pathology including organs, veins, peritoneum and connective tissue was created. A
great focus while developing was to recreate reality as accurately as possible.
Immersing trainees in the simulator is crucial so that they can focus on learning and practicing without
distractions. Putting them in a realistic operating scenario helps reduce anxiety and
self-consciousness before first time operations \cite{Parham.2019}.
\\ The proposed simulator creates pre-trained novices by providing new ways to acquire the psycho-motor skills,
sensory acuity and cognitive planning abilities needed for surgery.
Virtual reality based training is already proven to reduce the time to acquire surgical proficiency
\cite{RN61,RN62}. In randomized control studies, VR trained trainees performing laparoscopic cholecystectomy, made fewer errors and were faster \cite{RN63,RN64}. VR trained trainees required only
half the time to reach the skill level of intermediately skilled surgeons compared to standard training.
Hence, it is proven that skills acquired in simulations can successfully be translated to the OT \cite{RN63,RN64}.
\\ Parham et. al focus on high-quality visuals for immersion. All assets are represented
in 1:1 scale and the correct visual reproduction of organs, tools and hand positions is
ensured through thorough analysis of real world counterparts. Object materials are
physics-based and organic materials approximated by a physics engine. The lighting
emerging from medical equipment is based on their specifications. Lastly, the software
was developed with future expansion into other medical fields in mind \cite{Parham.2019}.
\\ The virtual OT consists of the open surgical area including organs of the patient, a tray
for surgical instruments and a monitor displaying simulated patient vitals and procedure
instructions. It was modeled after a real world OT located inside the University Teaching
Hospital in Lusaka, Zambia. The assets were modeled after receiving reference photos
and videos of locations and instruments and researching the female anatomy. It was
argued that even though 3D scans of real human organs exist, available hardware cannot
run them in real time VR due to performance limitations \cite{Parham.2019}.
\\ In the virtual reality simulation, the trainee stands in the virtual OT with an operating
table, tray of surgical instruments and the surgery awaiting patient with cervical cancer.
The procedure is closely modeled after an actual surgical procedure, meaning the surgical
site is exposed while the rest of the patient is covered. Instructions are given via a
monitor above the operating table and audio feedback is given to guide the trainee
through the simulation. The simulation lasts roughly 20 minutes and provides feedback
on the trainees accuracy on various points of the procedure and an overall score. To
compare traditional surgical training versus VR, the trainees are assessed by expert
surgeon-mentors \cite{Parham.2019}.
\\ Parham et al. recognize a need for clinical testing to establish VR efficacy, since there is
a lack of research for VRs clinical utility \cite{RN59}. However, VR was already considered as 
an important addition to surgical training over 10 years ago, with a prediction for even
more relevance in the future \cite{RN60}.
\\ It is mentioned how commercial VR will advance significantly in the future, allowing
for an even better adoption of VR simulations in surgery. Surgical training enhanced
with AR and VR will have wide applications according to Parham et
al. However, such technology has to be carefully built and clinically tested. VR and
AR have the potential to help train the workforce and to ensure higher quality standards
\cite{Parham.2019}.

Another example for virtual reality training tools is the Regional Anaesthesia Simulator
(RASim) \cite{RN70}. RASim focuses on physical accuracy using soft-tissue simulation and a haptic device to improve realism of interactions.
It was developed out of the lack of training opportunities. Even though mannequins exist
for training, they are too limited as they do not reflect the diversity of different patients
and they wear from repeated use \cite{RN70}. RASim allows for relevant tissue to be viewed. Tissue is segmented into skin, fat,
muscle, blood vessels and bones. The simulation consists of two steps: First, the user
performs a palpation using an extended index finger to sense the right puncture site for
the procedure on the patients skin. Secondly, the user can switch the interaction mode
to control a virtual needle. The needle can be moved freely until the patient’s tissue is
penetrated. When penetrating, only the depth of the needle can be varied. The user can
trigger a virtual aspiration at any time to check whether the needle tip penetrated any
blood vessel. If the needle tip is in emission range of any virtual nerve cord, according
muscular motor responses are displayed in real-time. An interesting feature of RASim is
the event logger, which remembers any given interaction step and thus allows to replay
the whole procedure. This allows for further assesment by users themselves, but even
moreso allows trainees to view the log of an experienced surgeon for learning.
\\ A user study compromised of ten residents and consultants with experience between one
and twenty years has been conducted by Ullrich et al. \cite{RN72}. The overall sentiment towards RASim was
positive. Realism of the anatomy and identification of landmarks were highly rated. The
majority of participants agreed that training with the simulator will be helpful to gain
more confidence and increase the rate of successful nerve blocks. However, the majority
of participants stressed that the haptic feedback has to be as sophisticated as possible \cite{RN72}.

Venkata et al. created a surgical simulator with tissue simulation of the 
bony structures of patients using iMSTK \cite{VenkataS.Arikatla.2018}.
An oscillating saw and a burr tool were successfully simulated 
with the use of HMDs and haptic feedback device for force feedback during 
bilateral sagittal split osteotomy (BSSO).
When performing the procedure, surgeons rely heavily on visual and haptic cues. 
Therefore,
the authors focussed on cutting the mandible in a specific 
area of the jaw as accurately as possible. This was realized utilizing 
the Geomagic Touch as a haptic device and the Occulus Rift as a 
display. The authors highlight the years of training, which are 
necessary to aquire the physical finesse and skill in order to 
perform the complex surgical procedure of BSSO. With the proposed 
method, crucial parts of the surgery
were succesfully simulated in a prototype application.
However, according to the authors, the real value of such a haptic feedback simulation must be
further evaluated in the future \cite{VenkataS.Arikatla.2018}.

A closely related work is the bilateral sagittal split osteotomy 
simulator (BSSOSim) \cite{RN69}.
The aforementioned long time to reach proficiency for BSSO surgery
is not only limited by the time it takes to get the necessary motor
skills, but also due to the reason that 
training opportunities are rare and can be harmful for
patients. Similar to RASim, this simulator compromises of 
realistic haptic feedback and tissue simulation.
In contrast to the aforementioned simulator, 
here, the focus is on drilling and breaking, which are a subset of 
the operation steps in BSSO.
\\ The main goal of the BSSOSim is the training of necessary motor 
abilities for a successful
outcome of the operation. To achieve the required realism to train motor 
abilities, the Geomagic Touch is once again utilized.
This allows users to
get a feel for the exerted force needed in the operation. 
In contrast to RASim, the immersive display L-Bench was used.
L-Bench is a room-mounted display. 
This way, the users could see their own hands on the haptic
feedback device while performing simulated procedures,
which can be seen as an advantage over the blocked vision
of an HMD in this case \cite{RN69}.

NeuroSimVR is another training tool utilizing haptic devices \cite{RN71}. This simulator allows
novice surgeons to learn and practice a spinal pedicle screw insertion (PSI). Mostafa
et al. \cite{RN71} recognize that most similar simulators perceive limited adoption by some medical
experts. In contrast to the aforementioned simulators utilizing haptic feedback, NeuroSimVR aims to optimize user interaction
and user experience. The goal is to support medical experts with a training and learning
tool that better satisfies their needs and expectations \cite{RN71}.
Mostafa et al. highlight that users, particularly novices, need an considerable amount
of training before they can use and operate many of the existing simulators. Hence,
NeuroSimVR has unique educational features and a simplified interface for ease of use.
A key problem with NeuroSimVR was that the simulation lacked realistic haptic feedback,
as noted by numerous participants of the user study conducted \cite{RN71}.
Based on the feedback of their user study, Mostafa et al. concluded a number of key components which should be implemented in future works.
\\ First, a flexible and simplified interaction is very important. Users especially liked the
real-time x-ray visualization of different segmented tissue. Post simulation performance
measures were also highly rated. Mostafa et al.
strongly suggested simplifying the design of interactions when building surgical simulation as an important step towards providing more individualized independent learning.
\\ The ability to adjust the visualization of each anatomy part should be supported. This
includes giving the user the option to hide or show various tissues as well as varying their
opacity. All participants of the conducted study stressed that performance measures
beyond a simple numerical score provided by the simulation are important. Integrating
feedback in a timely manner can be useful especially whenever something goes wrong.
In NeuroSimVR this was realized by visual blinking if mistakes were made \cite{RN71}.