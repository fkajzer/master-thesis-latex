Most research is utilizing freely available software.
Segmentation of patient specific DICOM data is predominantly done by using the open-source 3D Slicer \cite{Barber.2018,Barber.2020}.
Postprocessing of generated data can be done by utilizing tools, such as open-source Blender3D or commercially avaliable software such as Autodesk Maya \cite{Barber.2020,Parham.2019,Sampogna.2017}.
\\ A popular choice for development environment, utilized in a great number of research, is Unity3D \cite{Parham.2019,Pulijala.2017,Sampogna.2017}.
Even though a game engine at its core, Unity3D has been proven to be feasable to be used as an development environment for surgical simulations.
Unity3D utilizes the easy to use scripting language C\# and has integration with most consumer devices such as Oculus Rift, HTC Vive etc. \cite{wang2010new}.
\\
Other research utilized Unreal Engine, while tools for generating models for patients and operating room remained similar \cite{Barber.2020}.
\\
A lot of researchers use generic software, i.e., the game engines. Other researchers use their own in-house developments, e.g., ViSTA VR Toolkit \cite{RN70}, which often 
have specialized features for their projects.
Another specialized software, which was chosen, when more complex surgical simulations are planned together with more niche haptic feedback devices, is 
the interactive medical simulation toolkit (iMSTK) \cite{VenkataS.Arikatla.2018}.
iMSTK provides tools to realistically simulate tissue. 
The framework contains advanced algorithms for modeling of tool-tissue interactions that capture the behavior of 
aggregated tissue and organs at their boundaries or interfaces \cite{VenkataS.Arikatla.2018}.