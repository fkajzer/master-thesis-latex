In 2009, Swennen et al. discuss several improvements for three-dimensional treatment planning over conventional methods \cite{swennen2009three}.
Cost reduction and better patient outcome were achieved with three-dimenstional treatment planning, even though the planning was still conducted on conventional computers with 2D screens.
Additionally, experts all over the world can be consulted since treatment plans can be send via electronic mail.
Especially the diagnosis, treatment planning and treatment communication were improved \cite{swennen2009three}.

Virtual reality would offer advantages such as 3D stereoscopic vision, scalability, and repeatability over traditional methods such as described in \ref{chap::Introduction}.
As mentioned by Hassfeld et al \cite{HASSFELD20012}, developed software has a strong need to be easily accesible.

Haptic feedback devices, as described in \ref{sec::RelatedWork} are very specific to the procedure and very costly in all cases.
Additionally, some users have difficulties adjusting to the input devices, since they have a learning curve.

Kim et al (2017) presented that immersive VR combined with 3D printing models improved visualization and interaction with complex anatomy, surgical rehearsal, customization, and precise communication between medical specialists \cite{.2017}.
Furthermore, the authors argue that VR systems will improve model manipulation and provide more anatomical information for each stage of the surgical procedure over traditional techniques \cite{.2017}.

Pulijala et al (2017) concluded that when the context of simulation closely resembles a real-life model, such as an operating room environment, learning was found to be better \cite{Pulijala.2017}.
The authors stress that Oculus Rift- and HTC Vive-like devices brought high-quality surgical simulations into common man’s reach \cite{Pulijala.2017}.
\newline
Pulijala et. al describe how technical and non-technical skills have to be acquired in surgical training.
Traditional means of surgical education though hands-on-practice has been around for more than a century.
It was found that four out of ten novice surgeons are not confident in performing major procedures.
VR Surgery aims to provide cognitive training for oral and maxillofacial surgeons \cite{RN68}.

Parham et al (2019) presented how VR can be used as a low-cost training tool.
By developing a low cost simulation, using commercially available VR software and the Oculus Rift HMD, they have succesfully helped surgeons to prepare for radical abdominal hysterectomy surgery procedures in Zambia \cite{Parham.2019}.
The authors presented a simulator which succesfully created pre-trained novices by providing new ways to acquire the psycho-motor skills, sensory acuity and cognitive planning abilities needed for sugery.
Virtual reality based training has proven to reduce the time to acquire surgical proficiency \cite{RN61,RN62}. 
In randomized control studies, VR trained trainees performing laparoscopic cholecystectomy, made fewer errors and were faster \cite{RN63,RN64}.
VR trained trainees required only half the time to reach the skill level of intermediately skilled surgeons compared to standard training.
Hence, it is proven that skills acquired in simulations can successfully be translated to the operting theatre (OT) \cite{RN63,RN64}.
Parham et al recognize a need for clinical testing to establish VR efficacy, since there is a lack of research for VRs clinical utility \cite{RN59}. 

Swennen et al (2009) mention how one of the biggest disadvantage of this technology, having a powerful enough workstation to power the software, will soon be eliminated \cite{swennen2009three}.
Today, we already see the trend to affortable, consumer friendly HMDs.
Standalone HMD such as the Oculus Quest also propose an interesting application area for surgery, where cost of high-end computer hardware is eliminated. 
However, as of now, this technology is not powerful enough for surgical simulation.

Sampogna et al (2017) confirm the feasability of preoperative and intraoperative guidance by virtual 3D models \cite{Sampogna.2017}.
For patient specific models however, good-quality radiologic imaging is a concern.
Additionally, the process of segmentation can be time consumer, although it depends on the proficiency with available tools such as 3D Slicer.
The authors demonstrated the feasability and clinicians appreciation for VR in surgery \cite{Sampogna.2017}.

Commercial VR will advance significantly in the future, allowing for an even better adoption of VR simulations in surgery.
Surgical training enhanced with augmented and virtual reality will have wide applications according to Parham et al \cite{RN61,RN62}.
However, such technology has to be carefully build and clinically tested.
VR and AR has the potential to help train the workforce and to ensure higher quality standards \cite{RN52}.

Barber et al (2020) mention the steep learning curve of developing VR applications as a disadvantage.
The point out that even though VR is an established technology as of now, there is still no definition on how to effectively use tools and features to create educational applications \cite{Barber.2020}.
Even though the aforementioned drawback, the authors conclude that VR simulation using game engines is feasible and will be established in future studies.
This thesis confirms the feasibility of using game engines, in this case Unity3D, for bringing educational value through creating surgical simulations.

The related works highlight how important realism is for training simulations.
A combined approach of clincal examination together with 3D inspection of the patient has an unprecedented potential toward the diagnosis of the patient with a maxillofacial deformity \cite{swennen2009three} by providing a virtual inspection of the patient’s anatomy.
This is precisely why this thesis proposes a novel approach to existing methods.
Since commercially available hardware today is able to produce high fidelity visuals and an immersive, stereoscopic view of virtual objects,
the natural conclusion is to try out innovative approaches to existing problems with them.
It also highlights how important a 1:1 real world scale is for visualizing the spatial relationships in patient's anatomy and pathology.
Additionally, not only the surgical environment, but also the interactions, usability and the users representation in the VR have to be considered.