\textbf{Development Environment}: Most research is utilizing open source software as an development environment.
A popular choice, utilized in a great number of research, is Unity3D \cite{Parham.2019,Pulijala.2017,Sampogna.2017}.
Even though a game engine at its core, Unity3D has been proven to be feasable to be used as an development environment for surgical simulations.
Unity3D utilizes the easy to use scripting language C\# and has integration with most consumer devices such as Oculus Rift, HTC Vive etc \cite{wang2010new}.
\newline
Other research utilized Unreal Engine \cite{Barber.2020}, while tools for generating models for patients and operating room remained similar.
\newline
When more complex surgical simulations are planned, together with the use of more niche haptic feedback devices for example, the interactive medical simulation toolkit (iMSTK) was chosen \cite{VenkataS.Arikatla.2018}.
iMSTK provides tools to to realistically simulate tissue. VR is mainly used as an immersive display.
iMSTK contains advanced algorithms for modeling of tool-tissue interactions that capture the behavior of aggregated tissue and organs at their boundaries or interfaces \cite{VenkataS.Arikatla.2018}.