There are a wide range of devices summarized as HMDs.
However, there are some devices, with which interaction capacities are very limited.
\\ Early devices, such as the Samsung Gear VR or Google Cardboard, have limited interaction capacities.
However, they still allow for visualization of 3D models or 360-degree video of surgeries and were considered for surgical training applications \cite{sararit2017vr, gomez2019techniques}.
\\ On the other hand, there are more immersive HMDs, which allow the visualization of 3D models and interaction with the virtual environment 
through technologies such as hand tracking or controllers.
Two particularly widespread HMDs are Oculus Rift \cite{OculusRift} and HTC Vive \cite{Vive}.
They are the first consumer HMDs widely known for their interaction and immersion compatibilities, bringing immersive VR to the consumer market around 2016.
For surgical purposes, HMDs like the Oculus Rift, including the developement kits DK 1 and 2 \cite{Parham.2019, Pulijala.2017,Sampogna.2017}, HTC 
Vive \cite{.2017, Barber.2020}, were used.
One reason Oculus Rift HMD was selected was its availability, cost and efficiency at the time of research \cite{Pulijala.2017}.
\\ Another possibility are room-mounted displays, such 
as the L-Bench utilized by Knott et al. \cite{RN69}, which do not fully
cover the vision of the user. This means that users can 
benefit from immersive qualities of such a display, while 
still being able to see their own hands. However,
they are generally not as available to consumers.