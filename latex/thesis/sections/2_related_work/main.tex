A key part of this thesis is to advance the area of visualisation of medical imaging.
Even though techniques exist, operations are still planned on two-dimensional images for three-dimensional surgical sites.
In 2009, Swennen et al. discuss several improvements for three-dimensional treatment planning over conventional methods \cite{swennen2009three}.
Cost reduction and better patient outcome were achieved with three-dimenstional treatment planning, even though the planning was still conducted on conventional computers with 2D screens.
Especially the diagnosis, treatment planning and treatment communication were improved.
Additionally, experts all over the world can be consulted since treatment plans can be send via electronic mail.
As VR technology advances and entry costs are reduced, VR is considered more and more for a wide spectrum of applications.
In surgery, the main focus of research is on surgical training with pre-modeled patient anatomy and pathology.
By using state-of-the-art medical imaging and virtual reality technology and head mounted displays, this thesis aims to advance medical visualisation and surgery planning for patient specific procedures.
In Section \ref{sec::General}, a number of VR-based surgical simulations in different medical fields will be presented.
In Section \ref{sec::OralAndMaxillofacial}, the OMFS specific training tool "VR Surgery" will be presented.
VR Surgery was developed as a visualisation aid for senior surgeons and as a practice tool for novices.

