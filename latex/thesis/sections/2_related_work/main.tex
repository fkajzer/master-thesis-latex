As VR technology advances and entry costs are reduced, VR is considered more and more for a wide spectrum of applications \cite{Ayoub.2019, Berni.2020}.
However, VR was already considered over 10 years ago as an important addition to surgical training, with a prediction for even more relevance in the future \cite{RN60}.
In training and planning for surgery, the main focus of research is on surgical training with pre-modeled patient anatomy and pathology.

By using medical imaging in combination with VR, advances in medical visualization and surgery planning have already been achieved, which will be outlined in this section.
With the use of room-mounted displays, advantages of VR for the medical application have already been shown.
By using HMDs, costs can be reduced and integration into clinical routines is easier.
VR applications for surgery are distinguishable by their used display device, the kind of interaction, i.e. input device and the used development platform \cite{Vaughan.2016}.

\section{\label{sec::DisplayDevices}Common Display Devices}
There are a wide range of devices summarized as HMDs.
However, there are some devices, with which interaction capacities are very limited.
\\ Early devices, such as the Samsung Gear VR or Google Cardboard, have limited interaction capacities.
However, they still allow for visualization of 3D models or 360-degree video of surgeries and were considered for surgical training applications \cite{sararit2017vr, gomez2019techniques}.
\\ On the other hand, there are more immersive HMDs, which allow the visualization of 3D models and interaction with the virtual environment 
through technologies such as hand tracking or controllers.
Two particularly widespread HMDs are Oculus Rift \cite{OculusRift} and HTC Vive \cite{Vive}.
They are the first consumer HMDs widely known for their interaction and immersion compatibilities, bringing immersive VR to the consumer market around 2016.
For surgical purposes, HMDs like the Oculus Rift, including the developement kits DK 1 and 2 \cite{Parham.2019, Pulijala.2017,Sampogna.2017}, HTC 
Vive \cite{.2017, Barber.2020}, were used.
One reason Oculus Rift HMD was selected was its availability, cost and efficiency at the time of research \cite{Pulijala.2017}.
\\ Another possibility are room-mounted displays, such 
as the L-Bench utilized by Knott et al. \cite{RN69}, which do not fully
cover the vision of the user. This means that users can 
benefit from immersive qualities of such a display, while 
still being able to see their own hands. However,
they are generally not as available to consumers.

\section{\label{sec::InteractionDevices}Common Interaction Devices}%(also mention showing the hands as body avatars here)
In the sense of Augmented Virtuality (AV) \cite{milgram1994taxonomy}, there exist different technologies for representation of the users hands in VR.
One such technology is Leap Motion, which captures the users hands through the use of two cameras \cite{LeapMotion}.
These were applied in the surgical field to immerse users in the simulation \cite{VenkataS.Arikatla.2018,Sampogna.2017}.
In VR, Leap Motion is used as an interaction tool to prepare for sugery.
The users hands are tracked in real time and a representation of the users hands is projected into VR.
With the use of hand and finger gestures, users could rotate the target anatomy and change the point of view \cite{Sampogna.2017}.
\\ In another approach, Kim et al. (2017) present an interactive environment, in which users can grab and cut a craniopagic patients model.
Using a representation of the input device, in this case a standard, first generation input device named Vive Controller, the user gets a sense of where his hands are in the virtual world, although they are not represented as virtual hands. 
Manipulation of the presented model is user friendly, as each of the two implemented interaction is accesed via the use of a single button, namely the trigger button for cutting and grip button for grabbing the patients model \cite{.2017}.
\\ However, the use of consumer HMDs together with more specialized, non consumer haptic feedback devides such as the Geomagic Touch were also described in studies \cite{VenkataS.Arikatla.2018}.
Here, haptic feedback devices are used to improve the realism of surgical simulations.
Advantages of such devices are more realistic representation of the surgical procedure, while effectively having a higher barrier of entry.

\section{\label{sec::FrameworksAndTools}Common Frameworks and Tools}
Most research is utilizing freely available software.
Segmentation of patient specific DICOM data is predominantly done by using the open-source 3D Slicer \cite{Barber.2018,Barber.2020}.
Postprocessing of generated data can be done by utilizing tools, such as open-source Blender3D or commercially avaliable software such as Autodesk Maya \cite{Barber.2020,Parham.2019,Sampogna.2017}.
\\ A popular choice for development environment, utilized in a great number of research, is Unity3D \cite{Parham.2019,Pulijala.2017,Sampogna.2017}.
Even though a game engine at its core, Unity3D has been proven to be feasable to be used as an development environment for surgical simulations.
Unity3D utilizes the easy to use scripting language C\# and has integration with most consumer devices such as Oculus Rift, HTC Vive etc. \cite{wang2010new}.
\\
Other research utilized Unreal Engine, while tools for generating models for patients and operating room remained similar \cite{Barber.2020}.
\\
A lot of researchers use generic software, i.e., the game engines. Other researchers use their own in-house developments, e.g., ViSTA VR Toolkit \cite{RN70}, which often 
have specialized features for their projects.
Another specialized software, which was chosen, when more complex surgical simulations are planned together with more niche haptic feedback devices, is 
the interactive medical simulation toolkit (iMSTK) \cite{VenkataS.Arikatla.2018}.
iMSTK provides tools to realistically simulate tissue. 
The framework contains advanced algorithms for modeling of tool-tissue interactions that capture the behavior of 
aggregated tissue and organs at their boundaries or interfaces \cite{VenkataS.Arikatla.2018}.

\section{\label{sec::SurgicalApplications}Surgical Applications}
In this section, a number of endevours for surgical simulations in VR will be presented.
This section gives a broad overview of previous VR-based simulations for surgeons in training.
Additionally, approaches where the usage of 360-degree or stereoscopic video are utilized together with VR will be presented.
\subsection{\label{sec::VRBasedApplications}VR-based Applications}
Parham et. al present a novel approach for a low cost surgical simulation to train novice
surgeons for cervical cancer surgery \cite{Parham.2019}. The simulator aims to create pre-trained novices
in resource-challenged countries. Cervical cancer is one of the world’s leading causes of
death \cite{Parham.2019}.
\\ By developing a low cost simulation, using commercially available VR software and the
Oculus Rift HMD, they have successfully helped surgeons to prepare for radical abdominal
hysterectomy surgery procedures. A near identical representation of an OT
using 1:1 scale was used for immersion. 3D replica of human female pelvic anatomy and
pathalogy including organs, veins, peritoneum and connective tissue was created. A
huge focus while developing was to recreate reality as accurately as possible.
Immersing trainees in the simulator is crucial so that they can focus on learning and practicing without
distractions. Putting them in a realistic operating scenario helps reducing anxiety and
selfconsciousness before first time operations \cite{Parham.2019}.
\\ Parham et al. recognize a need for clinical testing to establish VR efficacy, since there is
a lack of research for VRs clinical utility \cite{RN59}. However, VR was already considered over
10 years ago as an important addition to surgical training, with a prediction for even
more relevance in the future \cite{RN60}.
\\ The proposed simulator creates pre-trained novices by providing new ways to acquire the psycho-
motor skills, sensory acuity and cognitive planning abilities needed for surgery. Virtual
reality based training is already proven to reduce the time to acquire surgical proficiency
\cite{RN61,RN62}. In randomized control studies, VR trained trainees performing laparoscopic chole-
cystectomy, made fewer errors and were faster \cite{RN63,RN64}. VR trained trainees required only
half the time to reach the skill level of intermediately skilled surgeons compared to stan-
dard training. Hence, it is proven that skills acquired in simulations can successfully be
translated to the OT \cite{RN63,RN64}.
\\ Parham et. al focus on high-quality visuals for immersion. All assets are represented
in 1:1 scale and the correct visual reproduction of organs, tools and hand positions is
ensured through thorough analysis of real world counterparts. Object materials are
physics-based and organic materials approximated by an physics engine. The lighting
emerging from medical equipment is based on their specifications. Lastly, the software
was developed with future expansion into other medical fields in mind \cite{Parham.2019}.
\\ The virtual OT consists of the open surgical area including organs of the patient, a tray
for surgical instruments and a monitor displaying simulated patient vitals and procedure
instructions. It was modeled after a real world OT located inside the University Teaching
Hospital in Lusaka, Zambia. The assets were modeled after recieving reference photos
and videos of locations and instruments and researching the female anatomy. It was
argued that even though 3D scans of real human organs exists, they are too inefficient
to run in real time VR \cite{Parham.2019}.
\\ In the virtual reality simulation, the trainee stands in the virtual OT with an operating
table, tray of surgical instruments and the surgery awaiting patient with cervical cancer.
The procedure is closely modeled after an actual surgical procedure, meaning the surgical
site is exposed while the rest of the patient is covered. Instructions are given via a
monitor above the operating table and audio feedback is given to guide the trainee
through the simulation. The simulation lasts roughly 20 minutes and provides feedback
on the trainees accuracy on various postions of the procedure and an overall score. To
compare traditional surgical training versus VR, the trainees are assessed by expert
surgeon-mentors \cite{Parham.2019}.
\\ It is mentioned, how commercial VR will advance significantly in the future, allowing
for an even better adoption of VR simulations in surgery. Surgical training enhanced
with AR and VR will have wide applications according to Parham et
al. However, such technology has to be carefully build and clinically tested. VR and
AR have the potential to help train the workforce and to ensure higher quality standards
\cite{Parham.2019}.

Another example for virtual reality training tools is the regional anaesthesia simulator
(RASim) \cite{RN70}. RASim focuses on physical accuracy using soft-tissue simulation and a haptic device to improve realism of interactions.
It was developed out of the lack of training opportunities. Even though mannequins exist
for training, they are too limited as they do not reflect the diversity of different patients
and they wear from repeated use \cite{RN70}.
\\ RASim allows for relevant tissue to be viewed. Tissue is segmented into skin, fat,
muscle, blood vessels and bones. The simulation consists of two steps: First, the user
performs a palpation using an extended index finger to sense the right puncture site for
the procedure on the patients skin. Secondly, the user can switch the interaction mode
to control a virtual needle. The needle can be moved freely until the patient’s tissue is
penetrated. When penetrating, only the depth of the needle can be varied. The user can
trigger a virtual aspiration at any time to check whether the needle tip penetrated any
blood vessel. If the needle tip is in emission range of any virtual nerve cord, according
muscular motor responses are displayed in real-time. An interesting feature of RASim is
the event logger, which remembers any given interaction step and thus allows to replay
the whole procedure. This allows for further assesment by users themselves, but even
moreso allows trainees to view the log of an experienced surgeon for learning.
\\ A user study compromised of ten residents and consultants with experience between one
and twenty years has been conducted by Ullrich et al. \cite{RN72}. The overall sentiment towards RASim was
positive. Realism of the anatomy and identification of landmarks were highly rated. The
majority of participants agreed that training with the simulator will be helpful to gain
more confidence and increase the rate of successful nerve blocks. However, the majority
of participants stressed that the haptic feedback has to be as sophisticated as possible \cite{RN72}.

Venkata et al. created a surgical simulator with tissue simulation of the 
bony structures of patients using iMSTK.
An oscillating saw and a burr tool were successfully simulated 
with the use of HMDs and haptic feedback device for force feedback during 
bilateral sagittal split osteotomy (BSSO) \cite{VenkataS.Arikatla.2018}.
When performing the procedure, surgeons rely heavily on visual and haptic cues. 
Therefore,
the authors focussed on cutting the mandible in a specific 
area of the jaw as accurately as possible. This was realized utilizing 
the Geomagic Touch as a haptic device and the Occulus Rift as a 
display. The authors highlight the years of training, which are 
necessary to aquire the physical finesse and skill in order to 
perform the complex surgical procedure of BSSO. With the proposed 
method, crucial parts of the surgery
were succesfully simulated in a prototype application.
However, according to the authors, the real value of such a haptic feedback simulation must be
further evaluated in the future \cite{VenkataS.Arikatla.2018}.

A closely related work is the bilateral sagittal split osteotomy 
simulator (BSSOSim) \cite{RN69}.
The aforementioned long time to reach proficiency for BSSO surgery
is not only limited by the time it takes to get the necessary motor
skills, but also due to the reason, that 
training opportunities are rare and can be harmful for
patients. Similar to RASim, this simulator compromises of 
realistic haptic feedback and tissue simulation.
In contrast to the aforementioned simulator, 
here, the focus is on drilling and breaking, which are a subset of 
the operation steps in BSSO.
\\ The main goal of the BSSOSim it the training of necessary motor 
abilities for a successful
outcome of the operation. To achieve the required realism to train motor 
abilities, the Geomagic Touch is once again utilized.
This allows users to
get a feel for the exerted force needed in the operation. 
In contrast to RASim, the immersive display L-Bench was used.
L-Bench is a room-mounted display. 
This way, the users could see their own hands on the haptic
feedback device while performing simulated procedures,
which can be seen as an advantage over the blocked vision
of an HMD in this case \cite{RN69}.

NeuroSimVR is another training tool utilizing haptic devices \cite{RN71}. This simulator allows
novice surgeons to learn and practice a spinal pedicle screw insertion (PSI). Mostafa
et al. \cite{RN71} recognize that most similar simulators perceive limited adoption by some medical
experts. In contrast to the aforementioned simulators utilizing haptic feedback, NeuroSimVR aims to optimize user interaction
and user experience. The goal is to support medical experts with a training and learning
tool that better satisfies their needs and expectations \cite{RN71}.
\\ Mostafa et al. highlight that users, particularly novices, need an considerable amount
of training before they can use and operate many of the existing simulators. Hence,
NeuroSimVR has unique educational features and a simplified interface for ease of use.
A key problem with NeuroSimVR as noted by numerous participants of the user study
conducted \cite{RN71}, was that the simulation lacked realistic haptic feedback. Based on the
feedback of their user study, Mostafa et al. concluded a number of key components which should be implemented in future works.
\\ First, a flexible and simplified interaction is very important. Users especially liked the
real-time x-ray visualization of different segmented tissue. Post simulation performance
measures were also highly rated. Mostafa et al.
strongly suggested simplifying the design of interactions when building surgical simulation as an important step towards providing more individualized independent learning.
\\ The ability to adjust the visualization of each anatomy part should be supported. This
includes giving the user the option to hide or show various tissues as well as varying their
opacity. All participants of the conducted study stressed, that performance measures
beyond a simple numerical score provided by the simulation are important. Integrating
feedback in a timely manner can be useful especially whenever something goes wrong.
In NeuroSimVR this was realized by visual blinking if mistakes were made \cite{RN71}.

\subsection{\label{sec::Others}Others}%for your 3D video
In the scientific literature about VR surgery applications, the level of presence which is aimed to achieve varies a lot.
Some do not pay any attention to immerse users into the VR at all, i.e. they simply show a 3D model of the patients anatomy and 
pathalogy, while others make great efforts to completely immerse users into the VR \cite{Vaughan.2016}.
In some cases, VR is used solely as a novel visualization tool with novel control schemes \cite{.2017}.
However, other works make a great effort to give the user a level of presence, through modeling a virtual OT after real life locations as close as possible.
It was shown, that a higher level of presence can reduce stress and anxiety from real operations through VR training \cite{Pulijala.2017,Pulijala.2018,Pulijala.2018b}.
\\ In general, one of two approaches is utilized.
The first kind of approach is already depicted in Section \ref{sec::VRBasedApplications}.
Here applications make a great effort to simulate all aspects of the surgical environment, including patient, procedures with their respective instruments, and the virtual OT.
\\ This way, trainees feel present in the OT and focus on performing a surgical procedure.
In the other approach, which will be presented in the following, only some aspects of the surgical setting are present in the virtual OT.
360-degree video is used complimentary to VR to give stereoscopic instructional videos of certain procedures, while trainees have the option to pause video 
and visualize the patients anatomy via a 3D model of the patient.

Pulijala et al. \cite{Pulijala.2017} present a novel visualization and training tool for trainees as well as senior surgeons.
An immersive learning experience for surgical trainees through pre-recorded stereoscopic 3D videos of surgery and interactive models of patient’s anatomy was created \cite{Pulijala.2017}.
The depicted surgical procedure was "Le Fort 1" surgery, a type of maxillofacial surgery, performed to correct lower midface deformities \cite{Pulijala.2017}
\\ 3D videos are used to provide depth perception and realistic view of the surgical procedure, with other surgeons performing the procedure.
This gives trainees the impression to be present during real procedures.
It was found that such realistic scenarios improved learning.
Users can zoom in and touch the 3D model of the patient's anatomy to visualise spatial relationships between the anatomy.
Trainees get feedback through questions and tasks about procedures \cite{Pulijala.2017,Pulijala.2018}.

Parham et al. \cite{Parham.2019} used an approach in which filmed procedures in 360-degree are utilized together with an immersive virtual OT.
The immersive virtual OT was already described in detail in Section \ref{sec::VRBasedApplications}. Here, the focus will be on the utilization of immersive 
video for training purposes.
\\ Videos of surgical procedures were captured during real surgical procedures using six specially arranged GoPro 360-degree cameras and were used to caputure the OT.
Additionally, close-ups of the surgical procedures were captured with stereoscopic Sony 3D cameras \cite{Parham.2019}.
\\ Captured videos were then played inside of the virtual OT during the training of the procedure.
The training consisted of online didactic lectures, readings and seven modules, with embedded videos that correspond to the surgical procedure.
These could be viewed from withing the virtual OT, so that trainees would watch procedures and then simulate them via the tools present in the OT \cite{Parham.2019}.

Barber et al. \cite{Barber.2020} propose a fusion of 3D-intraoperative 
video with immersive VR as a teaching tool for lateral skull base surgery.
Interactive anatomic models are utilized in an virtual OT while watching 
360-degree video as a learning tool.
This VR simluation situates users in the lead surgeon's chair while 
simultaneously being able to pause, stop and rewind the procedure.
Additionally, users have the 3D interactive anatomic model in their 
hands to utilize stereoscopic visualization of the surgical 
site. A user study to validate the 
methology was not conducted \cite{Barber.2020}.


