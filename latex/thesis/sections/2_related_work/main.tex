As VR technology advances and entry costs are reduced, VR is considered more and more for a wide spectrum of applications \cite{Ayoub.2019, Berni.2020}.
\newline
However, VR was already considered over 10 years ago as an important addition to surgical training, with a prediction for even more relevance in the future.
Low cost VR applications could be especially considered for resource-challenged countries where there is a lack of skilled workforce, mainly because of cost \cite{RN60}.
\newline
In surgery, the main focus of research is on surgical training with pre-modeled patient anatomy and pathology.
By using medical imaging and VR HMDs, advances in medical visualisation and surgery planning for patient specific procedures have been achieved, which will be outlined in this section.
VR HMD applications for surgery are distinguishable by their used VR HMD, the kind of interaciton i.e. input device and the used development platform.

\textbf{Devices}: Even though a wide range of devices are summarized as VR HMDs, there are just a few immersive HMDs, which allow the visualization of 3d models and interaction with their environment and the user thorugh technologies such as hand tracking or controllers.
\newline
Two particularly widespread devices are Oculus Rift \cite{OculusRift} and HTC Vive \cite{Vive}.
They are the first consumer HMDs widely known for their interaction and immersion compatibilities, bringing immersive VR to the consumer market around 2016.
For surgical purposes, HMDs like the Oculus Rift, including the developement kits DK 1 and 2 \cite{Parham.2019, Pulijala.2017,Sampogna.2017}, HTC Vive \cite{.2017, Barber.2020}, but also more recent devices such as Oculus Rift S (Quellen) were used.
\newline
One reason Oculus Rift head-mounted display was selected was its availability, cost and efficiency at the time of research \cite{Pulijala.2017}.
\newline
In the sense of augmented virtuality \cite{milgram1994taxonomy}, there exist different technologies for representation of the users hands in VR.
One such technologie is Leap Motion, which captures the users hands through the use of two cameras \cite{LeapMotion}.
These were applied in the surgical field to immerse users in the simulation \cite{Pulijala.2017, Sampogna.2017}
\newline
However, the use of consumer HMDs together with more specialized, non consumer haptic feedback devices such as the Geomagic Touch were also described in studies \cite{VenkataS.Arikatla.2018}.
Here, haptic feedback devices are used to improve the realism of surgical simulations. 
Advantages of such devices are a more realistic representation of the surgical procedure, while effectively having a higher barrier of entry.

\textbf{Development Environment}: Most research is utilizing open source software as an development environment.
A popular choice, utilized in a great number of research, is Unity3D \cite{Parham.2019,Pulijala.2017,Sampogna.2017}.
Even though a game engine at its core, Unity3D has been proven to be feasable to be used as an development environment for surgical simulations.
Unity3D utilizes the easy to use scripting language C\# and has integration with most consumer devices such as Oculus Rift, HTC Vive etc \cite{wang2010new}.
\newline
Other research utilized Unreal Engine \cite{Barber.2020}, while tools for generating models for patients and operating room remained similar.
\newline
When more complex surgical simulations are planned, together with the use of more niche haptic feedback devices for example, the interactive medical simulation toolkit (iMSTK) was chosen \cite{VenkataS.Arikatla.2018}.
iMSTK provides tools to to realistically simulate tissue. VR is mainly used as an immersive display.
iMSTK contains advanced algorithms for modeling of tool-tissue interactions that capture the behavior of aggregated tissue and organs at their boundaries or interfaces \cite{VenkataS.Arikatla.2018}.

Advancements in computers and imaging, especially over the last 10 years, have permitted the adoption of 3-dimensional imaging protocols in the health care field.
\newline
Virtual patients can be created by patient-specific anatomic reconstruction (PSAR), which can then be studied and used to develop and simulate treatment protocols.
Schendel et. al. propose image fusion, which involves combining images from different imaging modalities tocreate a virtual record of an individual called a patient-specific anatomic reconstruction \cite{schendel2009three}.
By using the combination of different image acquisition techniques, a treatment was planned for a young woman with mandibular retrusion.
It was shown that using generated 3D models of the patients specific anatomy and pathalogy brought huge improvements in planning treatments over traditional methods \cite{schendel2009three}.

In another related work, Computet Tomography (CT) and Magnetic Resonance (MR) were compared on the diagnostic potential in the aassessment of Synovia Chondromatosis (SR) of the Temporo-Mandibular Joint (TMJ).
Testaverde et al compared softtissueinvolvement (disk included), osteostructural alterations of the joints, loose bodies and intra-articular fluid of eight patients with symptoms of SC \cite{testaverde2011ct}.         
Results showed that CT scan is excellent to define bony surfaces of the articular joints and flogistic tissue but it fails in the detection of loose bodies when these are not yet calcified.
Optimally, a PSAR approach as proposed by Schendel et al \cite{schendel2009three} should be used.

[Instrumente usw.]

[Echter OP Raum]

\textbf{Multi User Capability}:

\textbf{Augmented Virtuality}: Augmented virtuality is when the real world is "augmented" using virtual reality \cite{Milgram.1994}.
Immersive VR HMDs in combination with 3D printed models are being considered \cite{.2017,Sampogna.2017,Barber.2018}.
Models are printed at 1:1 scale for visualization, and additionally imported into the VR to make us of VRs interactive features \cite{.2017}.
\newline
Venkata et al (2018) have created a surgical simulator with tissue simulation of the bony structures of patients \cite{VenkataS.Arikatla.2018}.
An oscillating saw and a burr tool were successfully simulated with the use of HMDs and haptic feedback device for force feedback during BSSO.