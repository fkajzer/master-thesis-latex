As VR technology advances and entry costs are reduced, VR is considered more and more for a wide spectrum of applications \cite{Ayoub.2019, Berni.2020}.
\newline
However, VR was already considered over 10 years ago as an important addition to surgical training, with a prediction for even more relevance in the future.
Low cost VR applications could be especially considered for resource-challenged countries where there is a lack of skilled workforce, mainly because of cost \cite{RN60}.
\newline
In surgery, the main focus of research is on surgical training with pre-modeled patient anatomy and pathology.
By using medical imaging and VR HMDs, advances in medical visualisation and surgery planning for patient specific procedures have been achieved, which will be outlined in this section.
VR HMD applications for surgery are distinguishable by their used VR HMD, the kind of interaciton i.e. input device and the used development platform.

\textbf{Devices}: Even though a wide range of devices are summarized as VR HMDs, there are just a few immersive HMDs, which allow the visualization of 3d models and interaction with their environment and the user thorugh technologies such as hand tracking or controllers.
\newline
Two particularly widespread devices are Oculus Rift \cite{OculusRift} and HTC Vive \cite{Vive}.
They are the first consumer HMDs widely known for their interaction and immersion compatibilities, bringing immersive VR to the consumer market around 2016.
For surgical purposes, HMDs like the Oculus Rift, including the developement kits DK 1 and 2 \cite{Parham.2019, Pulijala.2017,Sampogna.2017}, HTC Vive \cite{.2017, Barber.2020}, but also more recent devices such as Oculus Rift S (Quellen) were used.
\newline
One reason Oculus Rift head-mounted display was selected was its availability, cost and efficiency at the time of research \cite{Pulijala.2017}.
\newline
In the sense of augmented virtuality \cite{milgram1994taxonomy}, there exist different technologies for representation of the users hands in VR.
One such technologie is Leap Motion, which captures the users hands through the use of two cameras \cite{LeapMotion}.
These were applied in the surgical field to immerse users in the simulation \cite{Pulijala.2017, Sampogna.2017}
\newline
However, the use of consumer HMDs together with more specialized, non consumer haptic feedback devices such as the Geomagic Touch were also described in studies \cite{VenkataS.Arikatla.2018}.
Here, haptic feedback devices are used to improve the realism of surgical simulations. 
Advantages of such devices are a more realistic representation of the surgical procedure, while effectively having a higher barrier of entry.

\textbf{Development Environment}: Most research is utilizing open source software as an development environment.
A popular choice, utilized in a great number of research, is Unity3D \cite{Parham.2019,Pulijala.2017,Sampogna.2017}.
Even though a game engine at its core, Unity3D has been proven to be feasable to be used as an development environment for surgical simulations.
Unity3D utilizes the easy to use scripting language C\# and has integration with most consumer devices such as Oculus Rift, HTC Vive etc \cite{wang2010new}.
\newline
Other research utilized Unreal Engine \cite{Barber.2020}, while tools for generating models for patients and operating room remained similar.
\newline
When more complex surgical simulations are planned, together with the use of more niche haptic feedback devices for example, the interactive medical simulation toolkit (iMSTK) was chosen \cite{VenkataS.Arikatla.2018}.
iMSTK provides tools to to realistically simulate tissue. VR is mainly used as an immersive display.
iMSTK contains advanced algorithms for modeling of tool-tissue interactions that capture the behavior of aggregated tissue and organs at their boundaries or interfaces \cite{VenkataS.Arikatla.2018}.

\textbf{Visualization}: Segmentation of patient specific DICOM data is predominantly done by using 3D Slicer \cite{Barber.2018,Barber.2020}.
Postprocessing of generated data can be done by utilizing tools such as Blender3D or Autodesk Maya \cite{Barber.2020,Parham.2019,Sampogna.2017}.
\newline
Sampogna et al (2017) achieve patient specific 3D models by acquiring medical imaging of different modalities, such as CT and MRT \cite{Sampogna.2017}.
To better visualize patient specific anatomy and pathalogy, the acquired DICOM data was segmented in 3D Slicer and postprocessed in Blender.
Additionally, generated 3D models were printed, so that a virtual and real life 3D model exist.
The finalized 3D models were evaluated by a radiologist.
In VR, the users used Leap Motion for VR interaction to prepare for sugery.
With the use of hand and finger gestures, users could rotate the target anatomy and change the point of view.

\textbf{Interactions}: Kim et al (2017) present an interactive environment, in which users can grab and cut a craniopagic patients model \cite{.2017}.
Manipulation of the presented model is user friendly, as each of the two implemented interaction is accesed via the use of a single button.
\newline
Parham et al (2019) created a VR simulation, in which traineee stand in the virtual OT with an operating table, tray of surgical instruments and the surgery awaiting patient with cervical cancer.
The procedure is closely modeled after an actual surgical procedure, meaning the surgical site is exposed while the rest of the patient is covered.
Instructions are given via a monitor above the operating table and audio feedback is given to guide the trainee through the simulation.
The simulation lasts roughly 20 minutes and provides feedback on the trainees accuracy on various postions of the procedure and an overall score.
To compare traditional surgical training versus VR, the trainees are assessed by expert surgeon-mentors \cite{RN52}.

Pulijala et al \cite{Pulijala.2017} presented "VR Surgery", which allows trainees to virtually participate in a surgical procedure and interact with the patient's anatomy \cite{RN6}.
There is a huge emphasis to depict real surgical procedures and all related circumstances, such as a crowded operating table, as close as possible.
The users could interact with the annotated 3D models of the skull by zoom in and out and touch the 3D model.
Different structures could also be isolated to better inspect anatomic relationships \cite{Pulijala.2017}.


\textbf{Immersion}: In the related works, the level of immersion which is aimed to achieve varies a lot.
Some do not pay any attention to immerse users into the VR at all, while others make great efforts to completely immerse users into the VR.
\newline
In Pulijala et als (2017) "VR Surgery", 3D videos provide depth perception and realistic view of the surgical procedure, with other surgeons performing the procedure.
This gives trainees the impression to be present during real procedures.
It was found that such realistic scenarios improved learning.
Users can zoom in and touch the 3D model of the patient's anatomy to visualise spatial relationships between the anatomy.
Trainees get feedback through questions and tasks about procedures.
\newline
Parham et al (2019) present an immersive surgical simulator in which the surgical environment is modeled as closely to reality as possible \cite{Parham.2019}.
The authors focused on high-quality visuals for immersion.
The virtual OT consists of the open surgical area including organs of the patient, a tray for surgical instruments and a monitor displaying simulated patient vitals and procedure instructions.
It was modeled after a real world OT located inside the University Teaching Hospital in Lusaka, Zambia.
The assets were modeled after recieving reference photos and videos of locations and instruments and researching the female anatomy.
It was argued that even though 3D scans of real human organs exists, they are too inefficient to run in real time VR \cite{RN52}.
3D models of the human female pelvic anatomy and pathalogy including organs, veins, peritoneum and connective tissue were created.
The accuracy of the simulation in regard to real life counterparts was the main focus of this works.
Immersing trainees in the simulator was crucial so that they can focus on learning and practise without distractions.
Putting them in a realistic operating scenario helps reducing anxiety and selfconsciousness before first time operations \cite{Parham.2019}.
\newline
Barber et al (2020) propose a fusion of 3D-intraoperative video with immersive VR as a teaching tool for lateral skull base surgery.
Interactive anatomic models are utilized in an virtual OT while watching 360-degree video as a learning tool.
This VR simluation situates users in the lead surgeon's chair while simultaneously being able to pause, stop and rewind the procedure.
Additionally, users have the 3D interactive anatomic model in their hands to utilize stereoscopic visualization of the surgical site \cite{Barber.2020}.
\newline
Usage of immersive video is also utilized broadly.
While some do not utilize video, others used filmed surgical procedures as a method to prepare trainees for real surgery and reduce stress in the real operation environment.
\newline
Parham et al (2019) used an approach in which filmed procedures in 360 degree are utilized together with an immersive virtual operating theatre.
Videos of surgical procedures were captured during real surgical procedures via six specially arranged GoPro 360-degree cameras and were used to caputure the OT \cite{Parham.2019}.
Additionally, close-ups of the surgical procedures were captured with stereoscopic Sony 3D cameras.
\newline
Captured videos were then played inside of the VR OT during the training of the procedure.
The training consisted of online didactic lectures, readings and seven modules, with embedded videos that correspond to the surgical procedure.
These could be viewed from withing the virtual OT, so that trainees would watch procedures and then simulate them via the tools present in the OT \cite{Parham.2019}.

\input{sections/2_related_work/subsections/multi_user.tex}

\textbf{Augmented Virtuality}: Augmented virtuality is when the real world is "augmented" using virtual reality \cite{Milgram.1994}.
Immersive VR HMDs in combination with 3D printed models are being considered \cite{.2017,Sampogna.2017,Barber.2018}.
Models are printed at 1:1 scale for visualization, and additionally imported into the VR to make us of VRs interactive features \cite{.2017}.
\newline
Venkata et al (2018) have created a surgical simulator with tissue simulation of the bony structures of patients \cite{VenkataS.Arikatla.2018}.
An oscillating saw and a burr tool were successfully simulated with the use of HMDs and haptic feedback device for force feedback during BSSO.