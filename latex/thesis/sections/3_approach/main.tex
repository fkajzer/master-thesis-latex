The related works highlight how important presence is for training simulations.
A combined approach of clincal examination together with 3D inspection of the patient has an unprecedented potential toward the diagnosis of the patient with a maxillofacial deformity \cite{swennen2009three} by providing a virtual inspection of the patient’s anatomy.
This is precisely why this thesis proposes a novel approach to existing methods.
Since commercially available hardware today is able to produce high fidelity visuals and an immersive, stereoscopic view of virtual objects,
the natural conclusion is to try out innovative approaches to existing problems with them.
It also highlights how important a 1:1 real world scale is for visualizing the spatial relationships in patient's anatomy and pathology.
Additionally, not only the surgical environment, but also the interactions, usability and the users representation in the VR have to be considered.
\\ Based on the related work described in Section \ref{sec::RelatedWork} and the goal of the thesis outlined in Section \ref{chap::Introduction}, a requirement analysis has been conducted.
The requirement analysis is divided in non-functional and functional requirements.
Afterwards, based on the functional and non-functional requirements, a concept is then finalized and described.

\section{\label{sec::RequirementAnalysis}Requirement Analysis}
As previously described, the user interface consists of a GUI, VUI and natural hand interactions such as grabbing.
In the following, how users interact with the different kind of interfaces will be described. 

\subsection{\label{sec::GraphicalUserInterface}Graphical User Interface}
The first graphical interface which the user will see when pressing the button is depicted in Figure \ref{fig::UIProjectCase}
\begin{figure}[ht]
    \centering
    \includegraphics[width=200px]{images/implementation/user_interface/project_cases.png}
    \caption{\label{fig::UIProjectCase}User Interface for the Selection of Project Cases}
\end{figure}
The buttons for loading a case are created dynamically and depend on the number of project case JSON files found in the specified folder.
As will be explained later, the evaluation consists of five distinct scenarios in which distinct procedures have to be performed via the available tools.
By pressing the button by selecting it with the representation of the users hand in the virtual operating room, the project case will be loaded and the patient and information written down in the project case will be display from within the virtual operting room.
On the left side, different submenus can be selected.
Visualization tools can be selected via clicking on 'patient', as depicted in Figure \ref{fig::UIPatient}.
Here, tools which are the same for any project case can be used for visualization purposes.
Users have the option to scale the patient, look at the unprocessed patient for comparison and the ability to explode the 3D-Model to help with visualization.
Project cases can also be saved from this menu (Figure \ref{fig::UIPatient}).

\begin{figure}[h]
    \centering
    \begin{minipage}{.5\textwidth}
      \centering
      \includegraphics[width=0.95\linewidth]{images/implementation/user_interface/patient.png}
      \caption{\label{fig::UIPatient}User Interface for Patient Visualization Tools}
    \end{minipage}%
    \begin{minipage}{.5\textwidth}
      \centering
      \includegraphics[width=0.95\linewidth]{images/implementation/user_interface/segments.png}
      \caption{\label{fig::UIPatientSegments}User Interface for Patient specific Segments}
    \end{minipage}
  \end{figure}
\subsection{\label{sec::VoiceUserInterface}Voice User Interface}
\begin{center}
    \label{table::VoiceCommands}
    \begin{tabular}{ | l | p{4cm} | p{8cm} |}
    \hline
    Command & Action & Comment \\ \hline
    "Start" & Show first step of procedure & This command also disables the "Show all" command. \\ \hline
    "Show all" & Show all steps enumerated chronologicly & Is disabled via "Start" command. \newline
    Disables the use of 'step next' or 'step back'. \\ \hline
    "Step next" & Show next step of the procedure & \\ \hline
    "Step back" & Show previous step of the procedure & \\ \hline
    "Undo" & Reverts current step & Last undid step gets cached and can be redone via "Redo" command. \\ \hline
    "Redo" & Redo last reverted step & Only the last reverted step can be redone. \\ \hline
    "Train start" & Start train mode & Start at step 1 and navigate steps by following instructions and performing procedures in the correct manner.
    It also disables every other command except "Train stop" \\ \hline
    "Train stop" & Stop train mode & \\ \hline
    \end{tabular}
\end{center}

Table \ref{table::VoiceCommands} shows all Voice Commands.
\section{\label{sec::Concept}Concept}
As previously described, the user interface consists of a GUI, VUI and natural hand interactions such as grabbing.
In the following, how users interact with the different kind of interfaces will be described. 

\subsection{\label{sec::GraphicalUserInterface}Graphical User Interface}
The first graphical interface which the user will see when pressing the button is depicted in Figure \ref{fig::UIProjectCase}
\begin{figure}[ht]
    \centering
    \includegraphics[width=200px]{images/implementation/user_interface/project_cases.png}
    \caption{\label{fig::UIProjectCase}User Interface for the Selection of Project Cases}
\end{figure}
The buttons for loading a case are created dynamically and depend on the number of project case JSON files found in the specified folder.
As will be explained later, the evaluation consists of five distinct scenarios in which distinct procedures have to be performed via the available tools.
By pressing the button by selecting it with the representation of the users hand in the virtual operating room, the project case will be loaded and the patient and information written down in the project case will be display from within the virtual operting room.
On the left side, different submenus can be selected.
Visualization tools can be selected via clicking on 'patient', as depicted in Figure \ref{fig::UIPatient}.
Here, tools which are the same for any project case can be used for visualization purposes.
Users have the option to scale the patient, look at the unprocessed patient for comparison and the ability to explode the 3D-Model to help with visualization.
Project cases can also be saved from this menu (Figure \ref{fig::UIPatient}).

\begin{figure}[h]
    \centering
    \begin{minipage}{.5\textwidth}
      \centering
      \includegraphics[width=0.95\linewidth]{images/implementation/user_interface/patient.png}
      \caption{\label{fig::UIPatient}User Interface for Patient Visualization Tools}
    \end{minipage}%
    \begin{minipage}{.5\textwidth}
      \centering
      \includegraphics[width=0.95\linewidth]{images/implementation/user_interface/segments.png}
      \caption{\label{fig::UIPatientSegments}User Interface for Patient specific Segments}
    \end{minipage}
  \end{figure}
\subsection{\label{sec::VoiceUserInterface}Voice User Interface}
\begin{center}
    \label{table::VoiceCommands}
    \begin{tabular}{ | l | p{4cm} | p{8cm} |}
    \hline
    Command & Action & Comment \\ \hline
    "Start" & Show first step of procedure & This command also disables the "Show all" command. \\ \hline
    "Show all" & Show all steps enumerated chronologicly & Is disabled via "Start" command. \newline
    Disables the use of 'step next' or 'step back'. \\ \hline
    "Step next" & Show next step of the procedure & \\ \hline
    "Step back" & Show previous step of the procedure & \\ \hline
    "Undo" & Reverts current step & Last undid step gets cached and can be redone via "Redo" command. \\ \hline
    "Redo" & Redo last reverted step & Only the last reverted step can be redone. \\ \hline
    "Train start" & Start train mode & Start at step 1 and navigate steps by following instructions and performing procedures in the correct manner.
    It also disables every other command except "Train stop" \\ \hline
    "Train stop" & Stop train mode & \\ \hline
    \end{tabular}
\end{center}

Table \ref{table::VoiceCommands} shows all Voice Commands.