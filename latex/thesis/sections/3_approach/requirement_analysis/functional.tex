As highlighted in the related works in Section \ref{chap::RelatedWork}, surgical simulations are generally specialized to a certain procedure and a certain display device.
In this thesis, the aim is to abstract procedures and HMD in such a way that in principle, any procedure can be done with any HMD in the future.
Although for now, the focus will be on OMFS, with first procedures implemented being modelled after the workflow in the UHA.
In addition to the evidence from the related works, optimally current best practices in the field of VR development would be investigated from examplary applications from i.e. Oculus or 
Steam to know what to expect from such a system to complement the previous knowledge.
In this interdisciplinary thesis, the following requirements have been worked out to deliver best results for this project. 

\begin{compactenum}[label=(\textbf{\alph*})]
    \item\label{req::F1} \textbf{Virtual operating room}: The virtual operating room should be based on real locations in the University Hospital RWTH Aachen (UHA).
    If possible, it shall be designed via photogrammetry approach. The goal is to give the user the feeling of presence by utilizing familiar locations.
    \item\label{req::F2} \textbf{Locomotion}: The user should be able to freely move around.
    Additionally, some sort of teleportation should be implemented to complement the natural walking, to accomodate for smaller room setups.
    A simple turning option should also be implemented, so that users have the option, if they prefer that over natural turning.
    Avoiding nausea and discomfort should be prioritized at any times.
    \item\label{req::F3} \textbf{Interactions}: The user interface should allow for natural interactions as found commonly in modern commercial VR applications.
    The following interactions to aid with visualizing, planning, and training for OMFS should be provided, while being as easy-to-use as possible:
    \begin{compactenum}[label=(\textbf{\alph*})]
        \item \label{req::F3.1}Adapt the size, rotation and position of patient models.
        This way, users should be able to get the perfect viewpoint.
        Additionally, users should be able to reset changes to the default position.
        \item \label{req::F3.2}Project a copy of the unchanged model for comparison, so that changes made to the model do not have to be undone to compare it to the intital state.
        \item \label{req::F3.3}Explode segments and interact with each segment to get a better understanding of individual tissue.
        \item \label{req::F3.4}Transparency and completely hiding of segments so that hidden anatomy and pathology can be viewed, i.e. an x-ray view of the model.
        \item \label{req::F3.5}Measure relative distances so that spatial relationships between tissue can be investigated.
        \item \label{req::F3.6}Simulation of medical procedures which are commonly used in OMFS.
        This includes drilling, chiseling, sawing, milling, marking, and placing osteosynthesis plates.
        \item \label{req::F3.7}A mechanism to plan, view and go through procedures in relation to the patient should be implemented.  
    \end{compactenum}
    \item \label{req::F4}\textbf{Instruments}: Users should be able to interact with surgical instruments in a natural way.
    Surgical instruments should be realistic virtual objects, which have been designed after real physical instruments and materials which are used in the OMF department of the UHA.
    They should be able to be used for their intended operational procedures.
    \item \label{req::F5}\textbf{Screen Capture}: Users should be able to make captures of the current scene whenever they like to.
    By using photos of the left or right eye view of the user, planned procedures could also be shown outside of the VR application. 
    \item \label{req::F6}\textbf{Workflow}: An interface to the AR application should be provided.
    Procedures, which have been made should be able to be saved, reproduced and undid so that procedures can be understood, and modified if needed.
    Users could collaborate on individual procedures, improve them incrementally or train by going through procedures as often as needed.
    \item \label{req::F7}\textbf{Information}: Information about necessary patient data must be displayed.
    To understand, what kind of procedure is planned for which patient, this information should include patient name, finding, a case number and additional informational text.
    Additionally, procedure specific information about the current procedure step, which is to be performed, should be displayed inside the VR application, so that instructional guidance can 
    be provided at any step.
\end{compactenum}