Whereas non-functional requirements serve as constraints or restrictions on the design of a system, functional requirements describe concrete behaviour or functions.
Since the aim is to build a new, extensible system with an AR- and VR-Workflow in mind, many recommendations made by authors as stated in Section \ref{chap::RelatedWork} are simply not applicable.

TODO, Related Work referenzieren, probleme aufzählen.

In general, we want to use the prior knowledge of the discussed related works as a basis for our considerations. However, as we are aiming for a completely novel approach with our system, these deliberations might not be enough by themselves.
For example, they do not take the use of commercialy available soft- and hardware into account.
Also, for this thesis the abstraction of HMD hardware and controllers is of importance, so that it does not matter which hardware is currently available to users.
Additionally, the use of HMDs in the medical field, especially the use of more recent hardware such as the Valve Index and the ability for natural hand gestures, is a novel apporach.
Therefore, in addition to the evidence from the related works, optimally we would want to investigate current best practices in the field of VR development from big players such as Oculus or Steam to know what they expect from such a system to complement the previous knowledge.
In this interdisciplinary thesis, with the help surgeons of the UHA, the following requirements have been worked out to deliver best results for this project. 

\begin{compactenum}[label=(\alph*)]
    \item \textbf{(1) Virtual operating room}: The virtual operating room will be based on real locations in the UHA.
    If possible, it shall be designed via photogrammetry appraoch. The goal is to immerse the user with familiar locations.
    \item \textbf{(2) Locomotion}: The user should be able to freely move around.
    Additionally, some sort of teleportation will be implemented to complement the natural walking.
    \item \textbf{(3) Interactions}: Users should be able to interact with the virtual model of the patient.
    They shall be able to magnify and minify the patient and reset him to the original size.
    Another copy of the model should optionally be projected for comparison.
    The user should be able to set cutting planes.
    The user should be able to measure relative distances on the patient.
    Individual segments of the patient shall optionally be hidden, and transparency shall be able to be set.
    \item \textbf{(4) Instruments}: Users should be able to grab surgical instruments.
    Surgical instruments should be realistic virtual objects, which have been designed after real phisical isntruments and materials which are used in the oral and maxillofacial department of the UHA.
    A Mechanism to plan and view procedures in relation to the patient shall be implemented.
    Surgical instrumented should be able to be used for their intended operational procedures (drilling, sawing etc.)
    Procedures which have been made should be able to be saved, reproduced and undid.
    \item \textbf{(5) Screen Capture}: Users should be make screen captures whenever they like to.
    \item \textbf{(6) Workflow}: An interface to the AR application shall be provided.
    \item \textbf{(7) Information}: Information about necessary patient data must be displayed.
\end{compactenum}

Since there will be a lot of things to consider after all, some use cases will be prioritized over others in the end, and then finally implemented.