The first step in realizing these requirements is to develop a software architecture with an interface to the AR application in mind.
Since this is a bi-directional data exchange which does not need to be real time, one obvious approach would be to use a simple object notation language such as JSON.
A humanly readable format will be used so that operation plans can even be constructed without the need of an HMD.
Different options will be explored and decided upon in the beginning phase of the thesis.

After deciding on a software architecture, the next step is to create a virtual operating room and first surgical instruments.
The operation room will be designed after real operation rooms inside of the UHA.
A photogrammetry approach will be evaluated in the hope to give the most realistic experience as possible for OMF surgeons.
Virtual objects will be combined with a scan of the real world to give an interactive and immersive experience.
After the operating room, the focus will be on developing a traversing mechanism which allows for free exploration of the operating room.
Since operating rooms are not too large in general, it should be possible to traverse them in room-scale VR.
However, a teleport function will be added for convenience.
The surgical instruments will either be created in a 3D modelling software or exported in a similar way to how we obtain the segmented patient models from medical imaging.
The full functionality of the instruments will be developed in a later stage, since we will first work on the more important planning tools.

The planning tools will be the most critical part of the thesis, since they have to behave as expected.
There will be a mix of planning tools which are represented by virtual surgical instruments and basic features which will be mostly visualisation assistance.
At this stage, it will be crucial that the user inteface (UI) is as intuitive as possible and does not distract in any way.
Improving the UI however will be a continious effort throughout the thesis, and will hopefully become as intuitive and assisting as possible.

Each of the surgical instruments will have its own planning operations which can be recorded and saved.
Since at this stage, the architecture and format will already be decided, there should not be too much to worry about when implementing this feature.

To evaluated the feasibility in the context of OMFS, an expert review by working surgeons will be conducted, evaluating the usability and percieved realism of the tool.
Additionally, a small questionnaire and think-aloud-protocol of the participants will be used to determine if the new tool is preferred over conventional methods.


\subsection{\label{sec::ApproachAvailableTools}Available Tools}

Hier könnte zb stehen, was es für tools gibt.
Unreal VS Unity, SteamVR vs Oculus etc.
Headsets vergleichen, controller vergleichen.
Alles vorstellen, vergleichen, abwägen.
%\input{sections/3_approach/concept/available_hardware.tex}
%\input{sections/3_approach/concept/available_software.tex}
\subsection{\label{sec::ApproachAcquisition}Medical Imaging Acquisition}
Ein bisschen geschwafel über den process der imaging acquisition.
Bis zu dem step in blender ist alles im rahmen des forschungsprojektes der UHA.
Trotzdem würde ich hier gerne ein bisschen darauf eingehen, wie genau die USE CASES
dieser thesis entstanden sind. hierfür bräuchte ich aber deine Hilfe.
Zusätzlich würde ich hier Blender kurz vorstellen, und welche tools wir davon benutzt haben.
Darauf eingehen, worauf man beim improt in Unity achten muss.
%\input{sections/3_approach/concept/medical_imaging.tex}
\subsection{\label{sec::ApproachUserInterface}User Interface}
Verschiedene approaches vergleichen.
Warum ist das 3-Component UI gewählt worden.
Warum nicht einfach GUI?
Warum nicht einfach nur buttons?
Warum Voice?
Warum natural gestures?
Wodurch werden diese einzelnen componenets überhaupt ermöglicht?
%\input{sections/3_approach/concept/user_interface.tex}