This thesis aims to improve medical imaging and pre-operational planning by using a modern approach with head-mounted displays.
By using acquired 3D medical imaging in a virtual reality application, a very patient specific, highly realistic and only moderately costly technique with which surgeons can prepare for operations is pursued.
\newline
In 2009, Swennen et al. discuss several improvements for three-dimensional treatment planning over conventional methods \cite{swennen2009three}.
Cost reduction and better patient outcome were achieved with three-dimenstional treatment planning, even though the planning was still conducted on conventional computers with 2D screens.
Additionally, experts all over the world can be consulted since treatment plans can be send via electronic mail.
Especially the diagnosis, treatment planning and treatment communication were improved \cite{swennen2009three}.

This thesis aims to achieve the following advantages over conventional methods:
\begin{compactenum}[label=(\alph*)]
    \item Familiarize the operator with the patient specific anatomy and pathalogy before operating
    \item The ability to simulate important operation steps
    \item Allow revision of the virtual operation as often as needed
    \item Recording and analysis of users and others virtual operations
    \item Test out procedures
\end{compactenum}

Thorugh implementation of these logical steps in preparing for surgery, this thesis aims to improve the rate on which training has to be done 'live' on patients.
By providing the VR environment to visualize and train on virtual patients, the hope is to improve the time which has to be spent on training on real patients in the surgical environment.
Virtual reality would offer advantages such as 3D stereoscopic vision, scalability, and repeatability over traditional methods such as described in \ref{chap::Introduction}.
As mentioned by Hassfeld et al \cite{HASSFELD20012}, developed software has a strong need to be easily accesible.
With the emergence of new HMD such as the Oculus Rift S last year, HMDs become more and more affordable.
The development environment can be learned quickly with tools like Unity3D, so that modifications to the developed system are also an option.

For OMF surgeons, especially the imaging of voluminous objects is mentally demanding and this thesis hopes to eliminate this problem completely by providing realistic 3D medical imaging in virtual reality.
This thesis is part of an applied virtual and augmented reality workflow for oral and maxillofacial surgery using head mounted displays (HMDs) as described in Figure \ref{fig::ProjectPlan}.

The main goal of this thesis is to create a pre-operative assistance tool in VR with HMDs for oral and maxillofacial surgery as highlighted in Figure \ref{fig::ProjectPlan}. 
In addition, the results of the pre-operative planning might be used intra-operatively to provide assistance via Augmented Reality (AR) as described in the workflow.
To provide a useful preparational tool, it is critical to simulate individual operation steps.
Planned steps will be storable in a format in which they can be loaded and viewed in both the virtual and augmented reality applications bi-directionally as described.
By planning the operational steps in virtual reality, planned procedures can easily be shown to other staff involved.
Naturally, it will be of uttermost importance that we have medical equipment and an appropriate virtual environment recreated remarkably close to reality.
With the help of VR, this thesis is aiming to investigate whether stress and action times of surgeons can be improved by planning and training in a safe and risk-free environment.

The remainder of this Thesis is structured as follows:
\newline
First, related work in the field of VR applied to surgical workflows will be presented.
Used hard- and software, as well as input devices (with and without haptic feedback) will be presented.
Different visualization techniques and levels of immersion will be presented via the related work.
\newline
After presenting the state of the art in VR in the surgical field, the scope of this thesis and the implementation details of the developed VR thesis will be extensively described.
\newline
An evaluation with five surgeons of the OMFS department of UHA and the results will be presented.
\newline
Lastly, an thorough discussion followed by a the conclusion will follow.