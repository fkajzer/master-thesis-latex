The study was conducted in the OMFS office of the UHA.
The participants were actively practicing OMF surgeons of UHA.
The VR and AR part of the workflow (Section \ref{sec::Workflow}) were evaluated consecutively.
The prototype application was run on the same computer, that this 
thesis was developed on.
This computer had the following specifications:

\begin{compactenum}[label=(\alph*)]
    \item Graphics Card: Nvidia GTX 1070
    \item CPU: Intel i7-6700k
    \item Memory (RAM): 16GB
\end{compactenum}

This way, a smooth running application was guaranteed.
\\ Before playing through scenarios, participants were allowed 5 to 10 minutes familiarize themselves with the HMD and the application.
The participants were asked to keep their thoughts and feedback in mind during playthrough of the scenarios.
Auditory feedback was given through a generic speaker system, so that everyone involved could follow.
Participants were allowed to ask questions about system components, such as how certain instruments worked or how to use the GUI and VUI.
The participants were monitored through their right eye view on the computer monitor, so that certain aspects of user behaviour could be captured.
Participants were asked to use the implemented features for visualization and each instrument at least once, so that they could give proper feedback on system components.
Rather than explaining the training aspects of the application beforehand, participants were asked to activate traning mode (Section \ref{fig::TrainingMode}) after they had performed a number 
of operating steps.
Having to go through a previously planned procedure for training purposes at least once was directly incorporated, and was done as the last part of the evaluation.
Planning and training were clearly seperated in this evaluation, so that users could first get an impression of how they liked the planning options, and could afterwards focus on the training aspects.
\\ Before starting the scenarios, users were asked to fill out a Simulator Sickness Questionaire (SSQ) as described by Kennedy et al. \cite{kennedy1993simulator}.
After completion of the scenarios, the users were asked to once again fill out an SSQ.
Additionally, users were asked to fill out a System Usability Scale (SUS) \cite{brooke1996sus} questionnaire, as well as an additional questionnaire to assess impressions of individual system components.
Participants could also give their opinions and general thoughts on the system in designated text areas.
\\ The SSQ had the standarized options: None, slight, moderate and severe.
The SUS was used with a Likert Scale from from one (strongly disagree) to five (strongly agree).
For the system component related questions, a Likert Scale from one (strongly disagree) to seven (strongly agree).
At the top of each questionnaire, all options were first described so that participants knew what the points of the Likert Scale represent.
The full questionnaires are attached in Appendix \ref{appendix::A}.
Additionally, a preliminary questionnaire was used to collect data about the participants and their prior knowledge of virtual reality, which will be presented in the following section.
The questionnaires were printed out and each participant was identified by a unique code, which was written down on the questionnaires.
Answers to all questions were required, and questionnaires could not be submitted prior to responding to all questions.