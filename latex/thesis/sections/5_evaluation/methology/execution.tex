As previously mentioned, there were three distinct scenarios, meaning prepared project cases (Section \ref{sec::Workflow}), to choose from.
Participants could choose between generic patient models of a head, an arm or a leg.
They were chosen since they are common body parts to operate on in OMFS, arm and leg are commonly used for transplanting bone or skin tissue.
Each of these models was segmented and prepocessed as required by the workflow (Section \ref{sec::Workflow}).
The evaluation consisted of the following steps, in order:

\begin{compactenum}[label=(\alph*)]
    \item Introduction to VR and the system
    \item Familiarization with the interactions, locomotion, GUI and VUI
    \item Selecting a project case
    \item Using surgical instruments and visualization tools for planning procedures
    \item Training the planned procedure 
    \item Questionnaires
\end{compactenum}

\paragraph{\textbf{Procedure}}

The first phase served as an introduction to VR basics and what to expect from the system, since participants were generally unfamiliar with the technology.
The second part of the process served as a preparation and familiarization in VR first and foremost.
Therefore, users were asked to put on the HMD and familiarize themselves with the virtual environment.
Also, existing uncertainties and questions from the participants about the system, i.e. how GUI, VUI and surgical instruments are used, could be cleared up beforehand, so that the process of 
planning the procedure could be made with fewer interruptions.
After the initial phase of familiarization with VR and the application, participants would be instructed to choose a project case via the GUI.
\\ When participants felt familiar enough by signaling it to the instructor or after a maximum of 10 minutes, the goal of the usability study was explained.
To loosen the situation, it was first signified that the usability is being evaluated and not the performance of the users.
Afterwards, the participants were given time to plan any procedure they wanted.
The project cases where labeled in such a way that it was clear which body part they would be inspecting by selecting the case.
Participants could freely choose which case to start with.
To enable the participants to try out all surgical instruments and visualiation features, they were not asked to perform a specific procedure.
In general, participants tried out the several instruments and visualization features on their own, if not, they were asked to try certain features.
After trying out the features, participants were asked to perform two to three procedure steps.
Then, they were asked to use the voice command to start the train mode.
Since instructions for train mode are present in the virtual OT, no further instructions were given.
It was investigated, whether instructions for training (visual and textual) are enough, so that procedure processes could be reproduced.

\paragraph{\textbf{Feedback}}

When training the planned procedure was finished, participants would take off the HMD and could give their first impressions of the system.
Afterwards, participants were asked to fill out the questionnaires, and to write down wishes or improvements for the system at the last page of the questionnaire.
\\ The first questionnaire handed after the study was the second SSQ, to assess changes in well-being of participants.
The SUS questionnaire was the second, followed by the questionnaire containing system specific questions and feedback (Appendix TODO).
