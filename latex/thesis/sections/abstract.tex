\textbf{Introduction}: Virtual reality (VR) technology is first depicted in the scientific literature as early as 1963.
The upcoming of head-mounted devices (HMDs) is described since the early 1960s.
With the emergence of consumer HMDs such as the Oculus Rift in 2013, a widespread use of VR HMDs has followed.
The use of VR in surgical training is described since 1995 and has proven to bring advantages over classic surgical training.
However, the use of VR-based surgical scenarios is still limited duo to two reasons.
First, most current scenarios are application-specific instead of generalizable.
This requires costly and customized development for each medical procedure simulated in VR.
Second, most applications focus on expensive room-mounted displays and custom input devices instead of affordable HMDs.
Through the use of HMDs, an immersive and cost effective solution for VR surgical training could be provided.
By using consumer devices, the barrier of entry for such VR surgial training software is rather low.
Furthermore, there exist no surgical workflow for VR-based surgical planning and training and augmented reality (AR) glasses based intra-operational navigation.
By choosing a common data format, exchanging data from the planning process in VR can be reused for intra-operational guidance with AR glasses.
In the course of this master thesis, an open-source software application was developed for an VR scenario with HMDs embedded in an VR/AR surgical workflow in the field of oral and maxillofacial surgery (OMFS). 
(\textbf{TODO ANDREA}: Explain what OMFS is and what the goal of the medical procedures is. Remember: the abstract is the first thing readers read.)


\textbf{Material and Methods}: Unity 3D was used together with the Open VR Software Development Kit to develop an open-source software for a VR-scenario with the HTC Vive.
The software was coded in C\#. 
Additionally, the recently released Valve Index controller, which allow for tracking of the user hands in VR, were used for a more natural and immersive experience.
A real operation room was captured with a 360 degree camera and used for more immersion in the developed VR software.
Corresponding 3D models of true surgical instruments and a wide variety of surgical material like osteosynthesis plates were implemented.
The developed VR scenario was evaluated by 5 OMFS trainees in a system usability scale study.

\textbf{Results}: (\textbf{TODO ANDREA}: ...Here i would expect statements of how well your 5 participants could handle the system, whether they liked the options, whether they were able to plan the procedures for their cases etc.)
Furthermore, training with VR was percieved XXX. (TODO results of study)

\textbf{Conclusion}: The use of an open-source VR software with VR HMDs from the consumer market is a cost-effective application in the field of OMFS.
