\textbf{Introduction}: Virtual reality (VR) technology is first depicted in the scientific literature as early as 1963.
The upcoming of head-mounted displays (HMDs) is described since the early 1960s.
Since the emergence of consumer HMDs such as the Oculus Rift in 2013, HMDs are considered more frequently in several research areas.
The use of VR in surgical training is described since 1995 and has proven to bring advantages over classic surgical training.
However, the use of VR-based surgical scenarios is still limited due to two reasons.
First, most current scenarios are application-specific instead of generalizable.
This requires costly and customized development for each medical procedure simulated in VR.
Second, most applications focus on expensive room-mounted displays and custom input devices instead of affordable HMDs.
Through the use of HMDs, an immersive and cost-effective solution for VR surgical training could be provided.
By using consumer devices, the barrier of entry for such VR surgical training software is rather low.
Furthermore, no workflows exist for VR-based surgical planning and training in combination with augmented reality (AR) glasses based intra-operational navigation.
By choosing a common data format, exchanging data from the planning process in VR can be reused for intra-operational guidance with AR glasses.
In the course of this thesis, an application was developed for a VR scenario with HMDs embedded in a VR/AR surgical workflow in 
the field of oral and maxillofacial surgery (OMFS).
OMFS is a diverse surgical discipline, as it has to handle 
a complex arrangement of bones, teeth, vessels, cartilage, nerves, muscles,
skin and gland tissue. This thesis aims to investigate
whether surgical planning and training can be improved by using
abstract simulation in OMFS context.

\textbf{Material and Methods}: Unity 3D was used together with the Open VR Software Development Kit (SDK) in order to develop an application for a VR-scenario with the HTC Vive.
The standard language for Unity development C\# was used. 
Additionally, the recently released Valve Index controllers, which allow for tracking of the user's hands in VR, were used for a more natural and immersive experience.
A real operation room was captured with a 360-degree camera and used so that users felt present in the virtual environment.
Corresponding 3D models of true surgical instruments and a wide variety of surgical material like osteosynthesis plates were implemented.
The developed VR scenario was evaluated by 5 OMFS trainees in a user study.

\textbf{Results}: The user study confirms that using abstract simulations of surgical procedures for pre-operative training and planning is feasable.
Participants felt that the interactions with the virtual environment were intuitive.
The several visualization options and the presented surgical instruments were well received.
All users agreed that they would like to use VR technology in the surgical field in the future and that they would recommend colleagues to try out the presented
application. 

\textbf{Conclusion}: The use of commercially avaliable HMDs are a cost-effective application in the field of OMFS.