\textbf{Introduction}: Virtual reality technology is first depicted in the scientific literature as early as XXX.
The upcoming of head-mounted devices is described since the early 1960s.
With the emergence of consumer HMDs such as the Oculus Rift in XXXX, a widespread use of VR HMDs has followed.
The use of VR in surgical training is described since XXX and has proven to bring advantages over classic surgical training.
However, this is faced with two problems.
First the developed surgical VR scenarios for HMDs describe specific applications.
Second, the existing  open-source frameworks for surgical VR do not support HMDs.
Through the use of HMDs, an immersive and cost effective solution for VR surgical training could be provided.
By using consumer devices, the barrier of entry for such VR surgial training software is rather low.
Furthermore, the surgical VR training is strongly seperated from the intraoperative use of AR HMDs.
In the course of this master thesis an open-source software application was developed for an VR scenario with HMDs embedded in an AR/VR surgical workflow in the field of OMFS. 

\textbf{Material and Methods}: Unity 3D was used together with the Open VR Software Development Kit to develop an open-source software for a VR-scenario with the HTC Vive.
The software was coded in C\#. 
Additionally, the recently released Valve Index controller, which allow for tracking of the user hands in VR, were used for a more natural and immersive experience.
A real operation room was captured with a 360 degree camera and used for more immersion in the developed VR software.
Corresponding 3D models of true surgical instruments and a wide variety of surgical material like osteosynthesis plates were implemented.
The developed VR scenario was evaluated by 5 OMFS trainees in a system usability scale study.

\textbf{Results}: The development of an open-source software for the use in an VR-scenario with an VR-HMDs in the field of OMFS is feasible.
Furthermore, training with VR was percieved XXX. (TODO results of study)

\textbf{Conclusion}: The use of an open-source VR software with VR HMDs from the consumer market is a cost-effective application in the field of OMFS.
