In this thesis, a preoperative planning and training tool utilizing
commercially available HMDs, namely HTC Vive and Valve Index Controllers
has been developed. The feasability of the usability of 
abstract simulations of procedures in the context of preoperative planning
and training in OMFS could preliminarily be shown with the 
conducted user study. 
A combination of natural hand gestures, GUI and VUI have been
successfully utilized in such a way that the usability
of the system have been percieved as overwhelmingly positive.
Even though system was somewhat complex due to the amount of features 
and available tools, users did not feel confused while using the 
system.
The accessability of the developed software, 
as mentioned to be an essential part of developed applications by
Hassfeld et al. \cite{HASSFELD20012}, has been especially considered,
as per requirements depicted in Section \ref{chap::Approach}.
Through the overall concept of the application, as depicted in Section 
\ref{sec::Concept} and the workflow depicted in Section \ref{sec::Workflow},
the extensibility of the application should be enabled. 
The efforts made to provide users with an immersive experience,
as depicted in Section \ref{sec::VitualOperatingRoom},
resulted in a high level of presence in the virtual OT felt by the users.
\\ The biggest criticism of the application was the abstract 
simulation of procedures. Unfortunately, haptic feedback devices are still
not commercially available and somewhat expensive input devices. Since
the aim in this theis was to provide a broadly available, cost-effective
tool, they were not considered here. However, as VR technology progresses,
haptic feedback devices will surely be introduced to the comsumer market.
Especially then, they should be utilized in such applications to combat
the major criticism recieved in the user study.
On the software side, one would also have to make sure that 
the anatomy of the 3D models behaves realistically. This is, as 
presented in the related works (Section \ref{chap::RelatedWork}), 
not an easy task. 
Because the focus was an abstraction of procedures by the mere provision 
of instruments, a large number of procedures should be made possible.
In this context, a realisticly behaving surgical procedure is
especially difficult to realize.
In conclusion, it can be said that abstract simulations have a valuable
place as a preoperative planning and training tool. However, they 
can only complement the pre-existing methods and not replace them.
 
\section{Future Work}
From the results of the user study, an interesting prospect for
future improvements would be the integration of 
haptic input devices. 
To be consistent with the broad availability, which is
aimed to be achieved here, this should be postponed
until haptic feedback devices are available
in the consumer market. 
\\ In this proof-of-concept prototype, project
cases which are the essential part of the underlying VR-AR-based
surgical workflow were stored locally on the user's 
computer. This could be improved by integrating a remote server as 
a shared storage device. 
This would enable a better collaborative process as well as a 
better integration into the post planning step with AR glasses.
A special focus on the security of the stored data is then 
important, as this data is patient-specific and therefore 
contains sensitive information.
\\ Another interesting prospect would be the integration of multi-user
capabilities, such as proposed by Bashkanov et al. \cite{RN43} in their
multi-user conference room for surgical planning. 
This way, users could more actively collaborate in the process 
of surgical planning as well as point out 
important points of the patient specific anatomy in real time.
This should especially further improve the training aspects of 
the developed application. 
\\ In the future, the application will be published in the scientific community to enable further development of the here presented system for surgical planning and training.
The license under, which the application will be published, will be decided at a later date.





