From the results of the user study, an interesting prospect for
future improvements would be the integration of 
haptic input devices. 
To be consistent with the broad avaliability, which is
aimed to be achieved here, this should be postponed
until haptic feedback devices are available
in the consumer market. 
\\ In this proof-of-concept prototype, project
cases, which are the essential part of the underlying AR-VR-based
surgical workflow, were stored locally on the user's 
computer. This could be improved by integrating a remote server as 
a shared storage device. 
This would enable a better collaborative process as well as a 
better integration into the post planning step with AR glasses.
A special focus on the security of the stored data is then 
important, as this data is patient-specific and therefore 
contains sensitive information.
\\ Another interesting prospect would be the integration of multi-user
capabilities, such as proposed by Bashkanov et al. \cite{RN43} in their
multi-user conference room for surgical planning. 
This way, users could more actively collaborate in the process 
of surgical planning as well as point out 
important points of the patient specific anatomy in real time.
This should especially further improve the training aspects of 
the developed application. 
\\ In the future, this application will be published to the 
scientific community under the GNU General Public License v3.0 \cite{gpl}
to allow for further 
development of the here presented software for surgical planning 
and training. 