Worldwide, less than 25 percent of people with cancer requiring surgery have access to safe, affordable and timely surgery \cite{RN52}.
One way of increasing the surgical capacity is to reduce time and cost of training novices.
Parham et. al (2019) present a novel approach for training novice surgeons \cite{RN52}.
By developing a low cost simulation, using commercially available VR software and the Oculus Rift HMD, they have succesfully 
helped surgeons to prepare for radical abdominal hysterectomy surgery procedures.
A near identical representation of an operating theatre using 1:1 scale was used for immersion.
3D replica of human female pelvic anatomy and pathalogy including organs, veins, peritoneum and connective tissue was created.
A huge focus while developing was to simulate reality as accurately as possible. 
Immersing trainees in the simulator is crucial so that they can focus on learning and practise without distractions.
Putting them in a realistic operating scenario helps reducing anxiety and selfconsciousness before first time operations \cite{RN52}.

Parham et. al recognize a need for clinical testing to establish VR efficacy, since there is a lack of research for VRs clinical utility \cite{RN59}. 
However, VR was already considered over 10 years ago as an important addition to surgical training, with a prediction for even more relevance in the future \cite{RN60}.
These low cost VR applications could be especially considered for resource-challenged countries where there is a lack of skilled workforce, mainly because of cost.

This simulator creates pre-trained novices by providing new ways to acquire the psycho-motor skills, sensory acuity and cognitive planning abilities needed for sugery.
Virtual reality based training is already proven to reduce the time to acquire surgical proficiency \cite{RN61,RN62}. 
In randomized control studies, VR trained trainees performing laparoscopic cholecystectomy, made fewer errors and were faster \cite{RN63,RN64}.
VR trained trainees required only half the time to reach the skill level of intermediately skilled surgeons compared to standard training.
Hence, it is proven that skills acquired in simulations can successfully be translated to the operting theatre (OT) \cite{RN63,RN64}.

Parham et. al focus on high-quality visuals for immersion.
All assets are represented in 1:1 scale and the correct visual reproduction of organs, tools and hand positions is ensured through thorough analysis of real world counterparts.
Object materials are physics-based and organic materials approximated by an physics engine.
The lighting emerging from medical equipment is based on their specifications.
Lastly, the software would be developed with future expansion into other medical fields in mind \cite{RN52}.

The virtual OT consists of the open surgical area including organs of the patient, a tray for surgical instruments and a monitor displaying simulated patient vitals and procedure instructions.
It was modeled after a real world OT located inside the University Teaching Hospital in Lusaka, Zambia.
The assets were modeled after recieving reference photos and videos of locations and instruments and researching the female anatomy.
It was argued that even though 3D scans of real human organs exists, they are too inefficient to run in real time VR \cite{RN52}.

In the virtual reality simulation, the trainee stands in the virtual OT with an operating table, tray of surgical instruments and the surgery awaiting patient with cervical cancer.
The procedure is closely modeled after an actual surgical procedure, meaning the surgical site is exposed while the rest of the patient is covered.
Instructions are given via a monitor above the operating table and audio feedback is given to guide the trainee through the simulation.
The simulation lasts roughly 20 minutes and provides feedback on the trainees accuracy on various postions of the procedure and an overall score.
To compare traditional surgical training versus VR, the trainees are assessed by expert surgeon-mentors \cite{RN52}.

It is mentioned how commercial VR will advance significantly in the future, allowing for an even better adoption of VR simulations in surgery.
Surgical training enhanced with augmented and virtual reality will have wide applications according to Parham et. al.
However, such technology has to be carefully build and clinically tested.
VR and AR has the potential to help train the workforce and to ensure higher quality standards \cite{RN52}.

The goal of this thesis is to apply virtual reality in oral and maxillofacial surgery treatment planning.
In contrast to the presented work, this thesis aims to provide a useful tool not only for learning, but also for planning patient specific surgery.
Because of technology advancing rapidly, VR and AR become more and more affordable options.
Mentioned limitations for patient specific 3D models are not an issue anymore.
By using patient anatomy and pathology specific models, surgeons can profit from the mentioned benefits of VR simulation not only as a learning tool.
Moreover, the goal is to give trained surgeons a useful tool which adds to the arsenal of existing planning techniques.
However, trainees can also benefit immensely by studying and reproducing planned procedures of experts.
Another aspect of VR is the ability to communicate on a global level. 
In fields where expects are rare, planned procedures by such experts could be viewed and studied to get greater insight.
