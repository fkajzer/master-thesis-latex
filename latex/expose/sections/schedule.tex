The written part of the thesis will be worked on continuously throughout the whole six months period of the thesis, either in form of notes or directly written passages. What follows is a broad monthly schedule for the thesis:

\textbf{January:}
Designing a software architecture to allow the implementation of the requirements mentioned in Section \ref{sec::Features} and designing the structure of the "project cases" to save and load planned procedues.

\textbf{February:}
Start of the development phase of the thesis, beginning with implementing the a basic interaction system in which users can navigate and interact with objects.
This is important as a first step so that we can fully concentrate on the planning tools and the realistic operating room afterwards.

\textbf{March:}
Designing and implementing the virtual operating room and surgical instruments. Since an interaction system is already in place, the tools will already be interactable and base functionality provided.

\textbf{April / May:}
Implement planning operations (and save/load) and functionality for intended use of the surgical instruments (drill, saw...).
As mentioned in Section \ref{sec::Features}, it will be not important to have complex physics, but the necessary functionality of the instruments has to be implemented so that they are useful as planning tools.

\textbf{June:}
Finish the written part of the thesis.




